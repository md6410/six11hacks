% in the ``tabular'' environment, indicate columns separated by pipe
% characters. options are:
%   l           left aligned column
%   c           center aligned column
%   r           right aligned column
%   p{width}    paragraph column, text at the top
%   m{width}    ''                ''   in the middle
%   b{width}    ''                ''   at the bottom

\begin{table}%[h] % [h]ere [t]op [b]ottom [p]age
\centering
\begin{tabular}{p{1.5cm}| p{1.5cm} | p{12cm}}
\textbf{Param.} & \textbf{Value} & \textbf{Remark} 
\\ \hline
 & &
\\
$k$ &

1 &

Number of points in heading vector window. For higher resolution input
surfaces $k$ should be larger because the input points will be much
closer together.

\\

$min_{sd}$ & 

20 &

Minimum curvilinear distance between sample points. Smaller values
allow users to make smaller erase gestures, but might also introduce
false positives. 

\\

$t$ &

100ms &

Limits the sample history so only the most recent samples are
used. Samples older than this may be discarded.

\\

$min_\theta$ &

$\pi/2$ rad. &

Minimum corner angle between a recent sample heading and the current
sample heading. 

\\

$min_c$ &

5 &

Minimum number of corners required for a stroke to be an erasure.

\\

$C$ &

$70$ &

A percentage value in the range [0..100] describing how strictly to
choose erasure targets. A low value indicates few targets. Using zero
means only the most specific segment is erased; 100 means all segments
that intersect the erase gesture would be erased.

\end{tabular}
\caption{Parameters involved in detecting erase gestures.}
\label{tab:erase-params}
\end{table}
