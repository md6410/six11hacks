\begin{landscape}
\begin{table}
\begin{tabular}{ p{4cm} | p{4.5cm} | p{5.5cm} }
\textbf{``What'' to recognize} & 
\textbf{Examples} & 
\textbf{Remark} \\ 
\hline \hline

Genre &
Mathematical graph, architectural floor plan, web site layout, circuit design &

It may be sufficient to recognize a sketch is of a certain kind
without asking the user. The program could assume the sketch is
in a certain domain.\\ \hline

Characters (writing) & 
Alphanumerics, math symbols & 
Usually with other characters, in words and sentences. \\ \hline

Geometric shapes & 
Dots, lines, rectangles, blobs &
Geometric shapes are often drawn in relation to others. \\ \hline

Spatial features &
A is contained in, is above, is larger than B &
Spatial relations among elements may influence recognition. \\ \hline

Entities &
Domain-specific notation such as diodes, transistors &
Contextual clues can help disambiguate semantics of domain symbols. \\ \hline

Artistic nuance &
Shadows, textures, color &
Ink that modifies an existing element, perhaps suggesting 3D shape. \\ \hline

Commands &
Object selection, delete, copy, move, etc. &
Command ink specifies operations on the drawing. \\ \hline

Intention &
Drawing's function or behavior, e.g. \textit{circuit breaker} & 
Requires detailed domain knowledge and reasoning. \\ \hline

\end{tabular}
\caption[Elements to Recognize]{Kinds of elements to be recognized}
\label{tab:what}
\end{table}
\end{landscape}
