\chapter{Related Work}

summarize first: will talk about sketch-based interaction in context
of design, and how design sketching can be supported by computation if
only we figure out how interaction ought to work.

\section{Traditional Design Sketching}

things people do with pencils on paper, markers on whiteboards, and
why we borrow the waitress's pen to draw on bar napkins.

\section{Sketch Recognition}

overview of sketch recognition as an AI field. introduces concepts of
ambiguity, domain, isomorphism, and computational techniques for
performing recognition.

\section{Sketch-based Interaction Techniques}

Individual techniques for supporting conversation between a human and
the system. there are lots of isolated techniques, but few examples of
systems that integrate them.

\section{Sketch-based Interfaces and Modeling (SBIM)}

While there has been a SBIM conference for years, we have yet to see
useful and usable sketch-based design tools outside of labs. Actually,
if you can find one \textbf{in} a lab, let me know because I'd like to
see it.

Note: the conference (and apparently the field) is sketch based
\textbf{interfaces} and modeling, but it is really much more about
interface. It's about interaction. Maybe the name was badly chosen by
chance, or maybe it belies a fundamental lack of understanding by many
of the central players who purport to be experts.



