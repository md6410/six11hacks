\chapter{Related Work}

This chapter discusses two high level topics. First I discuss design,
paying close attention to rapid fabrication and laser cutting. Second
I provide background on research pertaining to sketch based interaction.

\section{Design for Rapid Fabrication}

A growing community of self-described \textit{makers} design and build
many kinds of physical things~\cite{gershenfeld-fab}. Some are
electronic or robotic gizmos, while others are made from traditional
material. These ``new makers''~\cite{gross-new-makers} are empowered
by rapid fabrication machines like 3D printers and laser cutters.

It is possible that we are beginning to see a shift from an economy
based on mass-production (in factories) to one that includes
mass-customization (in homes, schools, and community
centers)~\cite{economist-fab}. Rapid fabrication machines continue to
decline in price while improving in quality. A new sector of
businesses use rapid fabrication to cater to the needs of hobbyist
designers as well as people that need highly customized
goods~\cite{paulos-citizenscience}. For example, companies such as
Ponoko fabricate and send users physical output based on digital
models uploaded over the web.

Rapid prototying machines are used in many domains: mechanical
engineering, architecture, craftwork, industrial design, to name a
few. But users of these machines are often professional, but the
growing population of Makers are not necessarily trained or schooled
to design and build things. Instead, members of this group might be
better described as hobbyists or semi-professionals. They are smart
and motivated, but do not necessarily have a great abundance of time
or money.

\subsection{Rapid Fabrication Machines}

The type, quality, and composition of the output depends not only on
the machine, but also on the designer's skill in creating suitable
instructions. A computer controlled mill, for example, could be used
to create a customized office desk. But in order to build the parts
comprising the desk, somebody must give the machine a digital model
that indicates not only where the mill will cut, but how the cut will
be made.

A mill may several axes: it can move in a 2D plane, up and down
(z-axis), and some models allow the tool head to rotate. The tool path
is therefore an important consideration when designing for most types
of computer controlled manufacturing machines. In many (but not all)
cases, software tools can compute tool paths automatically. Designers
must consider a machine's capabilities.

SIMI targets design for laser cutters. This tool was chosen because
tool path concerns are nearly non-existant. Laser cutters are among
the more popular rapid fabrication machines. They can be thought of as
a very fast, strong, and precise automated razor blade, cutting
through flat material (paper, wood, plastic, metal, etc.) from
directly above. Many items can be made entirely with a laser cutter,
aside from the occasional screw or
glue. Figure~\ref{fig:laser-example} shows several examples of useful
items made with laser cutters. % TODO get good examples

The user places material on the laser cutter's \textit{cut
  bed}. Typical sizes for cut beds are in the range of 12''x 18'' to
48'' x 72'' (30.5 x 46 cm to 122 x 183 cm). A laser is directed
through a series of mirrors that are mounted on robotic arms that move
to allow the laser to reach any location on the cut bed. A 40 watt
machine can cut $\frac{1}{4}''$ (6mm) thick wood; a 100 watt machine
can cut up to 1'' (25mm) plywood.

\subsection{Current Design Tools for Laser Cutting}

Today, designers can choose among several modeling tools for laser
cutter projects. Adobe Illustrator, a general-purpose vector graphics
editor, is the most commonly used tool. Illustrator is full-featured
and has an interface new users find familiar. However, participants in
our formative study had a hard time using Illustrator quickly and
effectively because they spent a great deal of effort looking for
appropriate functions among the many features that are irrelevant to
laser cutters. Specialized CAD tools like Rhino or SolidWorks are
perhaps more appropriate for this kind of modeling but they also have
a substantial learning curve. If rapid fabrication is to become
common, appropriate modeling tools must be made accessible to ordinary
users~\cite{lipson-homefactory}.


%% \begin{itemize}
%% \item Software modeling tools (e.g. CAD)
%% \item Physical modeling tools (e.g. clay, wood, foam, paper)
%% \item Rapid fabrication machines
%% \end{itemize}

%% Among the rapid fab machines, we have:

%% \begin{itemize}
%% \item 3D printing
%% \item CNC Mills
%% \item Water jet cutters
%% \item Laser cutters
%% \end{itemize}

%% The focus of this thesis is on a sketch-based tool for laser
%% cutting. But while the narrow domain in question is rapid fabrication
%% using laser cutters, it is situated in the broader context of digital
%% modeling tools, so it should have applications to any designer that
%% uses computers to aid design. This accounts for nearly all designers,
%% so the implications are very broad.

\section{Sketching}

%% Need a paragraph to talk about the relationship between design,
%% designers, and their tools. Sketching is a significant activity
%% here. 

\subsection{Traditional Sketching}

%% Copy in the part of the lit review on traditional sketching and edit
%% it down. Be sure to also talking about pragmatic vs. epistemic
%% actions, and the tentative, explorational nature of design sketching.

\subsection{Computational Support for Sketching}

%% Distinguish between ``faithful'' and ``faux'' sketching
%% systems. Faithful tools respect the tentativeness of physical
%% sketching where people don't have all the answers at the start. Faux
%% sketching systems are named so because the developers view their tools
%% as being fast, but they do not support sketching as on paper.

%% Within computational support for sketching there are a bunch of
%% technical areas to look at. Can crib a lot from the lit review. The
%% sections that are most important are:

%% \begin{itemize}
%% \item Ink Parsing (corner finding, segmentation)
%% \item Recognition
%% \item Domain modeling
%% \item Graphics (rectification, rendering)
%% \item Interaction techniques, interaction design
%% \end{itemize}

