\chapter{Conclusion}

A citation to let me include a (short) bibliography: \cite{gross-boe}

Start with a few paragraphs that recap everything. Basically just
summarize the main takeaway of the preceding sections.

\section{Implications to Related Areas}

The important part of the thesis is the contributions. This is the
spot to discuss the implications of SIMI for various interests. Talk
about implications for:

\begin{itemize}
\item Rapid (or even \textit{slow}) fab
\item Democratized design
\item New combinations of interaction:
  \begin{itemize}
  \item Sketching and touch
  \item Sketching and visual gesture (e.g. Kinect)
  \item Sketching and VR or AR
  \item Sketching everywhere (walls, desks, tablets): can go beyond PC-at-a-desk design space
  \end{itemize}
\item Sketch Interaction and SBIM. 
  \begin{itemize}
  \item Must move beyond isolated toy demonstrations
  \item Must move beyond ``works on my computer'' demos. This does not
    promote building on prior work due to technical difficulty in
    replicating other work
  \item Interaction is one of the glaring missing links
  \item SIMI gives and example of a useful and usable
    \textit{system}. Its value is its coherence.
  \item If we have a platform for research, a common starting point
    for building things, we explore what this mode of interaction can
    do. But we have to be brave and have vision. Have to move beyond
    slavish attendance to statistics and laboratory user studies.
  \end{itemize}
\end{itemize}

\section{Future Work}

\begin{itemize}
\item Commercialization is likely
\item Most immediate next steps would be:
  \begin{itemize}
  \item Support 'doodle' ink: not recognized but may be important
  \item Handwriting recog. for dimension values
  \item More constraint types (e.g. parallel, centered)
  \item Way to visualize how 2D parts compose to make 3D thing
  \end{itemize}
\item Machine learning
  \begin{itemize}
    \item Learning user preferences, drawing styles
    \item Learning ad-hoc syntax
    \item Learning organizational preferences (e.g. cultural norms in
      company/domain)
  \end{itemize}
\item Domain-specific improvements (for laser cutting)
  \begin{itemize}
  \item Kerf
  \item Have a library of common joint types
  \item Optimize use of material
  \item Versioning of parts
  \end{itemize}
\item Possibiliities beyond laser cutting:
  \begin{itemize}
  \item 3D modelling
  \item 2D graphic design (illustrations or cartooning)
  \item Other design domains: architecture, mechanical engineering,
    software development
  \item Classroom or academic uses: mathematical graphs, physics
    simulation, box-and-arrow diagrams
  \item Essentially any domain that makes use of structured or
    semi-structured diagrams
  \end{itemize}
\end{itemize}
