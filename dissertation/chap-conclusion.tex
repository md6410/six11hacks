\chapter{Conclusion}

Rapid fabrication---3D printing, laser cutting, and many other
processes---can support designers to create in ways that were not
possible even ten years ago. But to realize this promise, machine
access is not enough. Users need effective and appropriate design
tools.

Sketching is cited (often blithely) as a common and necessary part of
the design process in various domains. Beginning in 1963 with
Sutherland's Sketchpad system~\cite{sutherland-sketchpad}, researchers
have developed sketching systems in many such areas, like mechanical
engineering~\cite{lipson-correlation}, web site
design~\cite{lin-denim}, and furniture
production~\cite{oh-fab,saul-sketch-chair}. In this time there has
been a great deal of work on sketch recognition and on isolated sketch
based interaction techniques. But there has been little effort in
exploring how all of this fundamental work can be integrated and
presented as a coherent and useful whole to take on a real-world
design problem.

This thesis explored the role of sketch based interaction techniques
as a basis for rapid fabrication design tools. I bridged the gap
between an academic treatment of sketch based design and a useful and
usable system. This work has included a literature review in design
and computational support for sketching (Chapter~\ref{sec:rw} and the
survey with my
committee~\cite{johnson-sketch-review}). Chapter~\ref{sec:formative}
explored the domain of design for laser cutting by studying both
artist and the artifact. Last,
Chapters~\ref{sec:overview}~and~\ref{sec:details} detailed my
\textit{Sketch It, Make It} system that provides a solid example of
how a sketch based system can present a set of sketch recognizers and
(most importantly) interaction techniques to enable real people to do
real work in a way that was previously not actualized.

While I have made a prototype that many observers find compelling,
there is still more work that could be done. It raises interesting
ideas about what else is possible with this form of interaction,
and questions remain unresolved.

In this chapter I discuss how my work with SIMI can serve as a
starting point for further work. First I discuss the implications SIMI
has to research on sketch-based interaction and modeling. Next I will
describe the most obvious next steps that I would like to take.

\section{Implications to Related Areas}

%% \item Rapid (or even \textit{slow}) fab
%% \item Democratized design

Laser cutters, 3D printers, CNC mills and so forth are called
\textit{rapid} prototyping machines because they facilitate fast
machine-work. Often, software design tools are the
bottleneck. Experienced professionals may have overcome the learning
curve and mastered the software to use it quickly, but the majority of
current (and certainly potential) users have not. Sketch based
interaction might be one part of a new kind democratized design
software that lets hobbyists and amateurs design and make with rapid
fabrication machines.

Inexpensive, widely available, and \textit{easy-to-use} rapid
fabrication hardware and software has interesting
consequences. Economically, it suggests a future where everyday people
are capable of simply ``printing'' a new object (such as a door knob)
when one is needed---which is obviously a topic of some concern for
the hardware store. Socially, it gives regular people the ability to
play an active role in the design, function, and production of
everyday things---which is concerning to professional designers and
engineers. While it is difficult to envision a world without hardware
stores (or designers and engineers), desktop design and fabrication
will have serious (and unpredictable) consequences.

Sketch based interaction may be paired with other input paradigms to
achieve unique results that would not be possible in isolation. While
sketching, the user's non-dominant hand is free to perform other
tasks. This can be leveraged to develop new ways to design. 

Touch input is a current topic of importance, with touch-sensitive
devices like tablets and smart phones dominating the consumer
electronics market. Sketching, paired with touch, has recently been
shown to give compelling results~\cite{hinckley-pen-touch}. New
technologies, such as Microsoft's distance-sensing Kinect, offer
additional avenues that might be appropriate to mix with sketch-based
input. Virtual reality hardware---special glasses that simulate 3D
projection, volumetric displays, \textit{etc.}---could be used to
immerse designers in augmented reality design tools that lets people
draw anything anywhere in physical space~\cite{jung-lightpen}. These
topics have been covered to various degrees, but the sketch
interaction component has always been ad-hoc. I've given a solid
example of one way to support such interaction, so we might have
something to gain by revisiting some of these ideas. The input
paradigms that pair sketching with something else (touch, gesture, VR)
have hardly been explored enough to consider them known topics.

Currently, sketch interaction is limited to devices that explicitly
support pen interaction. But ongoing advances in sensing technology
(e.g. Touch\'e~\cite{sato-touche}) may allow arbitrary surfaces to
accept sketch and touch input. Further, projection and display
technology continues to advance. When every available wall and
tabletop can display graphics and recognize sketch input, that has
serious consequences for the future of interaction and design.


%% \begin{itemize}

%% \item Sketch Interaction and SBIM. 
%%   \begin{itemize}
%%   \item Must move beyond isolated toy demonstrations
%%   \item Must move beyond ``works on my computer'' demos. This does not
%%     promote building on prior work due to technical difficulty in
%%     replicating other work
%%   \item Interaction is one of the glaring missing links
%%   \item SIMI gives and example of a useful and usable
%%     \textit{system}. Its value is its coherence.
%%   \item If we have a platform for research, a common starting point
%%     for building things, we explore what this mode of interaction can
%%     do. But we have to be brave and have vision. Have to move beyond
%%     slavish attendance to statistics and laboratory user studies.
%%   \end{itemize}
%% \end{itemize}

\section{Future Work}

The most interesting research is that which leaves more questions than
it answered. Having built a useful and useful system for laser cut
designs, much of the feedback I receive focuses on additional
features. Observers commonly would like to see a tool like SIMI for
some other domain, such as for 3D modeling.

\subsection{Common Feature Requests}

It is likely that Sketch It, Make It (or something like it) will be
made into a commercial product soon. We have accumulated a long list of feature
requests that can serve as a starting point for the next set of
additional features. 

Users like the paper-like feel of the tool, and would prefer to have
the ability to add ink that is not recognized. This enables designers
to explore freely, as they would on paper, while still having the
ability to use these unrecognized portions in future work.

Handwriting recognition is another commonly requested feature. This
was intentionally not included in the current prototype because it
would have required a substantial time investment without a
correspondingly substantial pay-out. Handmade numbers and text could be
applied to give dimensions and labels to parts of a drawing.

Users also request additional constraints, for example making two
lines parallel, or making an object appear at the midpoint between two
others. This would allow a wider and richer range of output.

Another feature that would likely make SIMI much more useful is the
ability to assemble 3D constructions from 2D parts. This would let
users see the relationship between the 2D parts they have designed and
the final 3D output. This can be useful, for example, to identify
stylistic or assembly problems before spending time and material on
the laser cutter.

\subsection{Machine Learning Improvements}

SIMI was built on the idea that it is generally better to give users
interaction techniques to state their intentions, rather than using
machine learning routines to deduce what the user wants. However,
there are a number of cases where machine learning could be used to
make the overall user experience better.

Each user has a drawing style that is different from any other. The
differences can be subtle, or might be substantial. In keeping with
the paper-like interface, an intelligent agent could watch the user
draw and learn their particular style and preferences. For example, if
a user consistently fails to make the erase gesture correctly, their
preferred erasure gesture could be learned and applied (without
asking). The user could also teach the system new gestures entirely,
providing new syntax directly.

Sometimes companies or other organizations adopt conventions. These
conventions may be common within that community but not typically
found outside. SIMI could learn these social norms, and understand
when to apply them.

\subsection{Domain Specific Improvements}

Laser cutters and similar machines (water jet, plasma, CNC mills)
share common properties. If the software could account for these
properties, the user could incorporate them into their designs. An
obvious property of laser cutters is the \textit{kerf}---the width of
the cut path. The laser cutter used throughout SIMI's development
typically leaves a kerf of approximately 0.4mm. When designing
press-fit notches for rigid material that is 3mm thick, the kerf size
is a significant concern. If the tool understood that certain geometry
was sensitive to kerf thickness, the software could save the user time
and material costs by making more accurate cuts.

The press-fit notch just mentioned is just one type of
joint. Currently SIMI does not include a concept of joints. If it did,
the user would be able to indicate where joints should exist, between
which parts, what the material properties are---and the software would
take care of the rest.

Material use is another important consideration, both because it costs
money, and because supplies are often limited. For example, Ponoko
sells 15''x15'' squares of acrylic for between \$8 and \$28 depending
on how thick it is. To make more efficient use of material, the
software could perform intelligent cutfile layout with a shape-packing
algorithm. If the system knows where a piece of material has holes,
the algorithm could also place new parts on used material.

Assemblies often have a lot of parts, and it can frustrate users when
trying to determine which part goes where. The tool could etch part
numbers (or joint labels) into the material to aid assembly.

\subsection{Beyond Laser Cutting}

In this work, I set out to demonstrate how a set of interaction
techniques designed to be used together to make a useful and usable
system, and show that sketch-based interaction should be taken
seriously, particularly as a paradigm for doing design. To do this I
picked a specific domain---laser cutting---but there are other domains
that would have been appropriate as well. 

3D modeling is an obvious choice. Physical objects are necessarily
three dimensional, and designers often sketch them in
perspective. There are interesting challenges associated with ``3D
sketching'', and many others are working on these research topics.

2D applications will be easier to write based on SIMI's current
implementation. Graphic design applications like illustrations,
diagrams, logos, T-shirt designs, or cartooning are areas that SIMI's
interaction would likely be appropriate.

Teachers famously draw on chalk boards or whiteboards to illustrate
concepts in many areas, from physics to mathematics. A sketch-based
education system might be a useful tool for teaching STEM topics at
all levels---primary, secondary, and post-secondary education.

The techniques presented in this dissertation could potentially be used
in any domain that makes use of structured or semi-structured
diagrams. It could be used to think about ideas, generate concepts. Or
it could be a communication tool used to work with others in ongoing
design projects. Or sketch-based design could simply be used as a
fast, efficient method of specifying the designer's intent to create
finished output.


