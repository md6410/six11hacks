\chapter{Introduction}

Mass production enabled industrial economies to grow, allowing
higher-quality products to be manufactured and delivered efficiently
and inexpensively to consumers. While this process has clearly
improved people's quality of life in many ways, it also has brought
monotony. 

Nearly everything around us has been \textit{designed} and
\textit{built}. With mass production, a single design can be
replicated many times. For example, a company might design a coat rack
and build 200,000 copies. This model of coat rack is likely to be good
enough for its users, and while they might prefer this or that to be
different, it lets them hang up their coats and get on with their
lives---all for less than \$30. But what if the user \textit{could}
indicate exactly what they want for the same price?

Machines like 3D printers, CNC mills, and laser cutters are beginning
to give us a glimpse of what the future might hold. Today, a
knowledgable person can design and ``print'' parts for a coat rack on
rapid prototyping machines. And if the designer finds they want to
change something, they can simply print a revised model. While this is
not as cost-effective as buying an ``almost good enough''
mass-produced coat rack at a store, it does enable the consumer to
directly engage themselves with the design and production of the item.

Rapid prototyping is a new technological phenomenon with economic and
social consequences. The machinery that enables regular people to
``print'' objects at home was unavailable ten years ago: either it was
too expensive, or it had not been invented yet. The current crop of
rapid prototyping machinery is still too expensive for most people to
buy for personal use. And today's machines produce adequate (but not
typically compelling) output. Happily, this is quickly changing as
price decreases while quality improves. It is interesting to see a new
technology that enables new activities---some that we can predict, but
others that we can't yet clearly see. It is conceivable that we are
witnessing the beginning of a shift from an economy entirely based on
\textit{mass production} to one that includes \textit{mass
  customization}.

Rapid fabrication gives many people the opportunity to design and
build things when there was no opportunity before. In the mass
production model, people are merely consumers. Rapid fabrication
enables people to play an active role in designing and constructing
the world around them. But the machines, alone, are insufficient: a
human must tell them what to make. To support this, people need
adequate design tools.

This dissertation addresses the observation that people need adequate
design tools if they are to effectively use rapid fabrication
machines. Current modeling software targets users who go to school to
learn how to design and use design software. However, most people can
not dedicate that much time to learn the intricacies and gotchas for
design software.

While someone might a hard time using professional design software, it
is likely that person can \textit{draw} using pencil and paper to
communicate their ideas with other people. Even if the freehand sketch
is rough, it is an effective method to express ideas about the shape
of objects and how they relate to one another.

Researchers have tried to leverage sketching as a computational medium
for at least half a century, and despite many interesting prototypes,
we have yet to see this effort move beyond academic laboratories. I
believe the critical element missing from this body of work is the
lack of a \textit{system of interaction design} for sketch-based
tools.

In this dissertation, I fill this missing space by describing a
\textit{useful} and \textit{usable} modeling tool that exemplifies a
\textit{coherent set of sketch-based interaction techniques}. The
software, \textit{Sketch It, Make It (SIMI)}, lets people design
precise items for laser cutters. 

\section{Intended Audience}

This work is aimed primarily at researchers interested in sketch-based
modeling. This includes people working on topics traditionally found
in computer science (e.g. artificial intelligence or computer
graphics) as well as those interested in human-computer interaction
and interaction design. Beyond motivating the work (sketch-based
interaction) and domain (designing for laser cutters), a reader could
implement any of the techniques described herein. The target audience
may be in academia, but it is hoped that commercial software tools can
start to incorporate some of the novel interaction exemplified by my
tool.

Further, members of the physical hacking or tangible interaction
worlds may be interested in learning how their trade can benefit from
improved design tools.

\section{Motivation}

This work is motivated by academic and practical perspectives. From an
academic perspective, we have seen system after system that
demonstrates some sketching technology. These include recognition
algorithms, corner-finding methods, and interaction techniques, among
other topics. But as I have said, sketch-based tools don't exist
outside research labs.

%% Talk about motivation from both academic and practical
%% perspectives. From the acad perspective we have an interest in
%% exploring ways to bring together disparate sketch-based interaction
%% techniques to form a coherent whole. From a practical perspective we
%% want more ``natural'' tools that allow us to think more freely,
%% without being tied down to what some structured tool demands.

\section{Thesis Structure}

%% Summarize the rest...

