\documentclass{article}

\usepackage{times}
\usepackage{uist}

\begin{document}

% --- Copyright notice ---
\conferenceinfo{UIST'11}{October 16-19, 2011, Santa Barbara, CA, USA}
\CopyrightYear{2011}
\crdata{978-1-4503-0716-1/11/10}

% Uncomment the following line to hide the copyright notice
\toappear{}
% ------------------------

\bibliographystyle{plain}

\title{Sketch It, Make It: Freehand Drawing for Precision Rapid Fabrication}

\author{
\parbox[t]{9cm}{\centering
	     {\em Author One}\\
	     Institution Name\\
             City, ST, USA\\
	     user@institution.net}
\parbox[t]{9cm}{\centering
	     {\em Author Two}\\
	     Institution Name\\
             City, ST, USA\\
	     user@institution.net}
}

\maketitle

% TODO: change this
\abstract Abstract goes here. 

\classification{I.3.5 [Computational Geometry and Object Modeling]: Modeling packages}

% TODO: change this
\terms{Design, Human Factors}

\keywords{sketching, rapid fabrication, design tools, constraints}

\tolerance=400 % prevent words from sticking out in the margin

%% \begin{figure}[tb]
%% \vspace{1.9in}
%% \caption{A figure caption.  It is set in 9 point Helvetica type, with a
%% 0.5 cm wider margin on both left and right sides.} 
%% \label{fig-example}
%% \end{figure}

\section{INTRODUCTION}
* Personal fabrication

  - introduce ponoko/thingiverse because they are used later

* Laser cutting

* examples

* why is design for laser cutting hard?

  - tools mismatch

  - allude to Illustrator study

  - common tasks in this little domain:

    + make notches

    + use precise measurements

    + angle and length matching

    + items sometimes must join together

* made tool SIMI to explore solutions to these problems

  - sketch tool. why is this the right paradigm?

  - contributions

\section{RELATED WORK}
Related work on ...

* tools for design (various domains like UI design, graphic design, CAD)

* fabrication

* sketching

* informal interfaces 

\section{USER OBSERVATIONS}

\subsection{Formative Study}

* illustrator

* describe things made in study

* list problems we saw: x y z

\subsection{Artifact Analysis}

* examined 60ish laser-cut objects from ponoko and thingiverse

* common properties from analysis: (examples only)
  - symmetry

  - angle

  - material

  - joint types

\section{SKETCH IT, MAKE IT}

* recall common tasks described in

  - intro

  - formative study

  - ponoko analysis

* lots of screenshots

* map problems from earlier (x y z) to solutions in simi

* overview of interaction

\subsection{Implementation}

* Stylus with offhand button

* Recognizes syntax and gestures

* Syntax: lines, elliptical arcs, splines, circles, ellipses

* Gestures:

  - Latch (3 kinds) 

  - Erase 

  - undo/redo

  - right angle

  - same-length 

  - flow-selection

  - guide points

  - select stencils

* Constraints

  - introduce what they are

  - list types supported: right angle, same length, co-terminate, specific length

  - visual appearance

* Stencils (with or without holes)

\section{EVALUATION}

* screenshots/photos from user study

* other results from user study...

\section{ACKNOWLEDGMENTS}

% TODO: remove
Every paper should cite \cite{sutherland-sketchpad}, if only to placate BibTeX.

% TODO: fill this in later. Leave left blank for blind review.

\bibliography{simi}

\end{document}
