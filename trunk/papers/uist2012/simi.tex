\documentclass{article}


\usepackage{times}
\usepackage{uist}
%\usepackage[config, font=small, labelfont={sf,bf}, textfont=sf]{caption,subfig}
%\usepackage{setspace}
%\doublespace
\usepackage[config, font=small, labelfont={sf,bf}, textfont=sf]{caption}
\usepackage{subfig}
\usepackage{graphicx}

\begin{document}

% --- Copyright notice ---
%\conferenceinfo{UIST'11}{October 16-19, 2011, Santa Barbara, CA, USA}
%\CopyrightYear{2011}
%\crdata{978-1-4503-0716-1/11/10}

% Uncomment the following line to hide the copyright notice
\toappear{}
% ------------------------

\bibliographystyle{plain}

\title{Sketch It, Make It: Freehand Drawing for Precision Rapid Fabrication}

\author{
\parbox[t]{9cm}{\centering
	     {\em Author One}\\
	     Institution Name\\
             City, ST, USA\\
	     user@institution.net}
\parbox[t]{9cm}{\centering
	     {\em Author Two}\\
	     Institution Name\\
             City, ST, USA\\
	     user@institution.net}
}

\maketitle

% TODO: change this
\abstract Abstract goes here. 

\classification{I.3.5 [Computational Geometry and Object Modeling]: Modeling packages}

% TODO: change this
\terms{Design, Human Factors}

\keywords{sketching, rapid fabrication, design tools, constraints}

\tolerance=400 % prevent words from sticking out in the margin

%% \begin{figure}[tb]
%% \vspace{1.9in}
%% \caption{A figure caption.  It is set in 9 point Helvetica type, with a
%% 0.5 cm wider margin on both left and right sides.} 
%% \label{fig-example}
%% \end{figure}

\section{INTRODUCTION}

A growing community of self-described \textit{makers} design and build
many kinds of physical things~\cite{gershenfeld-fab}. Some are
electronic or robotic gizmos, while others are made from traditional
material. These ``new makers''~\cite{gross-new-makers} are empowered
by rapid fabrication machines like 3D printers and laser cutters.

It is possible that we are beginning to see a shift from an economy
based on mass-production (in factories) to one that includes
mass-customization (in homes, schools, and community
centers)~\cite{economist-fab}. Rapid fabrication machines continue to
decline in price while improving in quality. A new sector of
businesses use rapid fabrication to cater to the needs of hobbyist
designers as well as people that need highly customized
goods~\cite{nyt-rapidfab}.  For example, companies such as Ponoko
fabricate and send users physical output based on digital models
uploaded over the web.

Laser cutters are among the more popular rapid fabrication
machines. They can be thought of as a very fast, strong, and precise
automated razor blade, cutting through flat material (paper, wood,
plastic, metal, etc.) from directly above. Many items can be made
entirely with a laser cutter, aside from the occasional screw or glue.
Figure~\ref{fig:laser-example} shows several examples of useful items
made with laser cutters.

\begin{figure}[h]
\centering 
\subfloat[] {
  \label{fig:laser-example-a} 
  \includegraphics[width=0.4\linewidth]{img/flat-b.jpg}
}
\subfloat[] {
    \label{fig:laser-example-b}
    \includegraphics[width=0.4\linewidth]{img/flat-b.jpg}
}
\caption{Laser cut items.}
\label{fig:laser-example}
\end{figure}

Today, designers can choose among several modeling tools for laser
cutter projects. Adobe Illustrator, a general-purpose vector graphics
editor, is the most commonly used tool. Illustrator is full-featured
and has an interface new users find familiar. However, participants in
our formative study had a hard time using Illustrator quickly and
effectively because they spent a great deal of effort looking for
appropriate functions among the many features that are irrelevant to
laser cutters. Specialized CAD tools like Rhino or SolidWorks are
perhaps more appropriate for this kind of modeling but they also have
a substantial learning curve. If rapid fabrication is to become
common, appropriate modeling tools must be made accessible to ordinary
users~\cite{lipson-homefactory}.

In this paper, we present a modeling tool called ``Sketch It, Make
It'' (SIMI).  It is based on recognizing short sequences of sketched
input with a stylus. By using freehand drawn input, SIMI enables the
designer to iteratively and incrementally create laser cut models that
fit precise specifications.

We are inspired by the potential of freehand drawing as a basis for
our tool for several reasons. Sketching is quick and can be easily
learned. The technology is simple.  Unlike structured editing
software, a designer does not need to set the pencil's mode to line,
circle, or anything else. A freehand sketch can provide enough
information that others can translate it into a digital model. 

\subsection{Contributions}

This paper presents the results of a formative study on how people
design items for laser cutting. We also provide an analysis of
artifacts from Maker community web sites that reveals specific
properties common to many laser cutter projects.

The primary contribution of this work is a system that embodies a new
way of designing precise 2D shapes by incremental sketch recognition
and rectification. We conducted a user study on how this tool can be
used to better support design for laser cutter projects.

\section{BACKGROUND}

Laser cut items are composed of parts cut from solid, flat pieces of
material. The primary concern of the designer is to define the path
taken by the laser cutter. Parts may be put together in various ways.
Pieces can be glued or screwed together in layers. Alternately the
user can design joints so the parts fit together. Most joints have
small margins of error. It is important that lengths, angles, and
relative position be indicated with precision for the parts to fit
together as they should.

\section{RELATED WORK}

While ``the design process'' has important differences between
domains, there are two common properties to nearly all design efforts,
including design for laser cutters. First, designers sketch throughout
the process, particularly in the beginning. Second, a computer tool is
used to formalize the design. This is confirmed by observations of
graphic designers~\cite{wong-rr-prototypes}, automotive
engineers~\cite{kara-styling}, and software
developers~\cite{dekel-improvised-notation}. Freehand drawing is a
powerful means for thinking and specifying design intent, but it is
disconnected from the computer aided portion of the design
process~\cite{company-sketching-in-engineering}.

Computer support for fabrication design has been a topic of interest
for several decades, alternately called computer aided design (CAD) or
computer aided manufacturing (CAM). While current interaction is
mostly performed with a keyboard and a mouse, this was not always the
case. For example, SketchPad~\cite{sutherland-sketchpad} users
controlled the computer by setting modes and parameters with one hand
and drew on the screen with a light pen in the other. 

More recently, novel interfaces have been developed that enable users
to model items for fabrication by sketching. For example,
Plushie~\cite{mori-plushie} lets people design soft objects like
stuffed animals. Users create 3D models of bulbous objects by
sketching in a manner similar to Teddy~\cite{igarashi-teddy}. The
program creates a set of 2D shapes that users can cut from fabric,
sew, and fill with stuffing. 

Sketch Chair is a more recent example of a tool that helps make design
for rapid fabrication more accessible~\cite{saul-sketch-chair}. It
lets users sketch the contours of a chair's seat and back rest, and
using a different drawing mode, add legs. The system includes a
sophisticated physical simulator to let the designer explore its
physicality (for example to determine if it will remain upright). It
also allows designers to change subtle properties of curves using
on-screen control handles.

Most work on sketch-based interfaces focuses on the early phases of
design when people are thinking about framing the problem and
solution. Sketch-based systems such as Plushie and Sketch Chair enable
people to make things they otherwise would not be able to, but the
designer hands off a great deal of control to the system. Further,
they support only a narrow class of artifacts.

The literature is not completely devoid of sketch-based systems that
support precision, however. ParSketch~\cite{naya-parsketch} enables
designers to create parametric 2D models by incrementally recognizing
sketched geometry and commands. It uses pressure to distinguish
between linework (high pressure) and constraint commands (lower
pressure). Pegasus~\cite{igarashi-pegasus} and Lineogrammer are
sketch-based drawing tools that lets users make clean, rectified
vector graphics. They work on the principle of interactive
beautification, which supports iterative sketch/rectify sequences.

\section{BACKGROUND STUDY}

To better understand the domain of laser cutter fabrication we
conducted two related empirical studies. We interviewed people with
experience designing these artifacts, and watched how they
worked. This helps form a picture of the kinds of tasks and problems
designers face when designing for laser cutters. We also analyzed
laser-cut items on community web sites to find common features.

\subsection{Formative Study on Designer Work Practices}
\label{sec:formative}

We interviewed six designers to find out more about their work
practices and to better understand how they use their tools. The
participants have substantially different backgrounds: all have
training in some form of design, including mechanical engineering,
graphic design, and architecture. The participants were all part of
our tool's target demographic: people who were interested in making
things with laser cutters as a hobby. 

Each session lasted approximately an hour and was split evenly between
the interview and implementation portions. Participants were asked to
describe their design process, and to show sketches or videos of their
work. While there are subtle (and some substantial) differences in
their process, each followed the following pattern.

They begin by thinking about a problem and making drawings by
hand. Some sketches are made to think about how to frame the project
(what it is for), while others help reason about how to make it (how
it works, how it fits together). Some designers explicitly noted that
sketching is a necessary part of the process; it would not be possible
to move forward without making freehand drawings. When the idea is
reasonably well-formed they will implement the model with a software
tool. It is common for this to involve translating a hand-made sketch
to a computer model (Figure~\ref{fig:translate}).

\begin{figure}[h]
  \centering
  \includegraphics[width=0.9\linewidth]{img/translate-sketch-to-computer.jpg}
  \caption{A common part of designing for laser cutters: translating a
    hand-made sketch to a computer modeling tool. The sketch includes
    a perspective drawing of the desired result, and 2D diagrams of
    individual parts with important dimensions indicated.}
  \label{fig:translate}
\end{figure}

Following the work practices interview, participants were asked to
implement the sketch shown in Figure~\ref{fig:interview-sketch} using
their software tool of choice. The purpose of this was to learn what
problems people encountered when executing the common task of
translating a sketch to a computer model. 

\begin{figure}[h]
\centering 
\subfloat[The part users set out to replicate.] {
  \label{fig:interview-sketch-1} 
  \includegraphics[width=0.9\linewidth]{img/laser-me-1.jpg}
}

\subfloat[Drawing of how the part is used in context.] {
    \label{fig:interview-sketch-2}
    \includegraphics[width=0.9\linewidth]{img/laser-me-2.jpg}
}
\caption{Participants were asked to implement the stencil at the top
  using modeling software.}
\label{fig:interview-sketch}
\end{figure}

Participants mostly chose to implement the sketch with Illustrator (5
users), while another chose Rhino. In all cases, a designer's strategy
involved common activities: creating or editing boundaries, aligning
or snapping items, using guide lines or reference points, measuring
distances, specifying or changing lengths and angles, and creating
finished ``cut files'' that will be sent to the laser cutter. These
tasks are in addition to selecting/deselecting on-screen elements, or
view port management like zooming and panning.

Participants in this experiment spent a good deal of time on operating
overhead. This includes (1) trying to find the appropriate tool for
the next task, (2) recovering from errors made when the wrong tool was
selected, or (3) when the tool behavior was inconsistent with the
user's intention. Approximately 50\% of the designer's time was spent
in this way.

For example, one user was aware of Illustrator's ``Path Finder'' tool
and wanted to use it. The user searched the program's menu structure
and slowly hovered over toolbar buttons to read tool tips before
finding it. Next, the designer invoked various functions of the Path
Finder, using the keyboard shortcut to undo after each attempt, as he
searched for the correct mode within the subcommand palette. This
process lasted approximately 80 seconds before finally being able to
continue.

Occasionally participants used features in rather strange ways to
achieve a desired outcome. In order to remove an unwanted segment of a
polyline, one participant (a graphic designer) chose to create an
opaque white rectangle to obscure it, rather than erase it. (``Don't
tell anyone I did this'', he said at the time). 

Similar episodes are common: a person \textit{should} know the
`correct' action, but takes an alternate approach. The alternative
might achieve the intended effect, but it might be less efficient
(more operations, longer execution time) or it might introduce
unwanted complexity (such as the invisible white rectangle).

To summarize, these are the common tasks and problems found during
this interview study:

\begin{itemize}
\item creating/editing boundaries
\item aligning/snapping items
\item using guide lines or reference points
\item measuring distances
\item specifying lengths/angles
\item creating cut files
\item selecting/deselecting
\item view port management
\item finding and entering tool modes
\item recover from error (tool or user error)
\item `correct' action is unknown or hard to do
\end{itemize}

\subsection{Artifact Analysis}

Ponoko and Thingiverse are two popular web sites for selling or
sharing items that can be made with rapid fabrication like laser
cutters and 3D printers. Ponoko has thousands of user-designed items
for sale, most of which are produced with laser cutters. Thingiverse
is a warehouse of digital models that mostly contains 3D objects but
has a fair number of designs for laser cutters. We selected 55
laser-cut projects from these two sites by browsing their catalog. We
then examined the project attributes to better understand what people are
really making with laser cutters. The results of this analysis are
shown in Figure~\ref{fig:ponoko}.

\begin{figure}[h]
  \includegraphics[width=0.9\linewidth]{img/ponoko-analysis.png}
  \caption{Frequency of certain attributes were present in laser-cut
    projects from Ponoko and Thingiverse.}
  \label{fig:ponoko}
\end{figure}

We used a set of ten properties to characterize each project. These
were chosen based on our own experience with items made with laser
cutters, as well as from observations from the formative study
discussed shortly. These properties are summarized now and described
in greater detail below:

\begin{itemize}
\item \textit{Right Angle}: Dominant edges meet at 90-degree angles.
\item \textit{Notch and Finger Joints}: Two parts come together using one of
  the joints illustrated in Figure~\ref{fig:joint}.
\item \textit{Single part}: Project is composed of a single, flat piece of
  material (e.g. a coaster).
\item \textit{Fasteners}: Clear use of glue, screws, or bolts.
\item \textit{Symmetry}: Radial or linear symmetry is a dominant feature.
\item \textit{Rounded Corners}: Right-angle corners are slightly blunt.
\item \textit{Raster etch}: Laser cutter etched patterns (e.g. words,
  images) rather than cutting through material.
%\item \textit{Non-stencil lines}: Instances where thin lines were cut.
\item \textit{Repeating geometry}: Linework is repeated several times.
\end{itemize}

\begin{figure}[h]
\centering 
\subfloat[Notch joints.] {
  \label{fig:joint-notch} 
  \includegraphics[width=0.4\linewidth]{img/joint-notch.png}
}
\subfloat[Finger joints, alternately called box joints.] {
    \label{fig:joint-finger}
    \includegraphics[width=0.4\linewidth]{img/joint-finger.png}
}
\caption{Two common joint types.}
\label{fig:joint}
\end{figure}

The purpose of single part projects were typically artistic, using
expressive, curvy linework. Among those objects composed of more than
one piece, nearly all used finger or notch joints
(Fig.~\ref{fig:joint}). The rest used fasteners.

The more professional-looking models were generally the ones that used
rounded corners. Raster etching was also used for artistic effect in
several cases. Repeating geometry was found in most models (joints
were excluded from consideration). These patterns involve sequences of
lines or curves with consistent length and angles. Some patterns were
quite ornate.

\section{SKETCH IT, MAKE IT}

Based on observation of designer work practices and the artifacts they
make, we have developed \textit{Sketch It, Make It} (SIMI), a
sketch-based tool for modeling laser cut items. We aim to address many
of the problems with current modeling systems enumerated above while
giving people a tool that specifically supports designing laser-cut
items. 

% * map problems from earlier (x y z) to solutions in simi

% * overview of interaction

SIMI users draw with a stylus and can use an offhand button for a few
actions. The system recognizes input as either geometric linework or
gestural commands. Linework includes straight lines, elliptical arcs,
splines, circles, and ellipses. 

Users can issue commands on linework by drawing gestures. Some
gestures are recognized and acted on immediately, such as the erase
(scribble) gesture. Others, such as commands to constrain segments be
the same length, are recognized after the user presses the button, or
after a timeout.

When the user makes a closed 2D path, the system recognizes it as a
stencil. Stencils are shapes that can be placed on the virtual laser
cutter plane. Several copies of a stencil can be added. The system
generates a vector file that can be sent to a laser cutter without
modification.

After laser cutting the stencils, the user can assemble them to their
final configuration. Several examples of projects made with SIMI are
shown in Figure~\ref{fig:laser-example}.

\subsection{Implementation}

\begin{figure}[h]
\centering \subfloat[Automatic: endpoints merge when endcaps intersect
  (illustrated in blue).] {
  \label{fig:latch-auto} 
  \includegraphics[width=0.43\linewidth]{img/latch-auto-endcaps.pdf}
}\hspace{5mm}
\subfloat[Latching endpoints.] {
  \label{fig:latch-endpoint} 
  \includegraphics[width=0.43\linewidth]{img/latch-manual-endpoint.pdf}
}
\\
\subfloat[Latching continuation.] {
    \label{fig:latch-continuation}
    \includegraphics[width=0.43\linewidth]{img/latch-manual-continuation.pdf}
}\hspace{5mm}
\subfloat[Latching T-Junction.] {
    \label{fig:latch-tjunct}
    \includegraphics[width=0.43\linewidth]{img/latch-manual-tjunct.pdf}
}
\caption{Automatic and manual latching used to bring segments together.}
\label{fig:latch}
\end{figure}

\begin{figure}[h]
  \centering
  \includegraphics[width=0.9\linewidth]{img/erase-all.pdf}
  \caption{Erase gesture: before, during, and after.}
  \label{fig:erase}
\end{figure}

\begin{figure}[h]
  \centering
  \includegraphics[width=0.9\linewidth]{img/constraints-all.pdf}
  \caption{Gestures for adding a right angle (left) and same-length.}
  \label{fig:constraints}
\end{figure}

\begin{figure}[h]
  \centering
  \includegraphics[width=0.9\linewidth]{img/guides-all.pdf}
  \caption{Reference points and guides. The first three panels show
    one, two, and three reference points and the guides that are
    displayed as a result. The designer uses the circular guide to
    create the circular arc shown in the final panel.}
  \label{fig:guides}
\end{figure}


%* Stylus with offhand button

The guiding principle we used when developing SIMI is that the
designer should never need to set down their pen. Most input is
provided entirely with a stylus. A single button used by the
non-dominant hand gives access to additional commands. The gestures
used to give commands or make constraints are summarized below.

\subsubsection{Latching}

It is usually desirable for adjacent linework to meet at a common
point. Latching is the process of combining adjacent segments (lines,
splines, arcs, \textit{etc.} so they meet at a common point. For
example, a square may be drawn with with four strokes, leaving a total
of eight unique pen-down/pen-up locations, but a square should have
only four corners.

SIMI provides two methods for latching segments together. One way is
automatic: the system analyzes new linework and tries to find cases
where the user probably meant their segments to join together, and
adjusts one or more segments to conform to the likely
outcome. Automatic latching can be problematic if it is too
zealous. Early approaches only used distance to find which segments
should be latched~\cite{herot-latch-corners}: if two endpoints were
within $x$ units of one another, latch them. This works well when
segment lengths are large relative to $x$. But when the segments are
short, the latcher will merge points when the designer did not want
them to join. 

SIMI's automatic latching process uses length and the segment
direction at the endpoint. It creates an \textit{endcap}: a line
segment centered at an end point that is a fraction of the line length
(currently 0.1) that follows the tangent at the end point. For nearby
segments to be automatically latched, the related endcaps must
intersect. SIMI can latch two or more segments in this manner.

The automatic latching process is intentionally conservative to avoid
causing user frustration. This means it is common for the automatic
process to not find all the places where the user wanted lines to
meet. To counter this, SIMI gives users a very simple method to latch
segments: simply draw a small circle around the endpoints to latch.

Most (or all) linework in SIMI is meant to compose stencils, which are
sequences of latched segments. It is therefore necessary for the
designer to be able to find segments are not latched. The system draws
a red marker at lonely endpoints to expose un-latched segments.

There are three ways segments can be manually latched together:
endpoint latching, continuation, and T-junctions. Endpoint latching is
what the automatic latcher tries to do. Continuation latching is when
the user brings together two segments that are close to the same
direction at the joined point. Continuation latching replaces two
segments with a single larger segment. A T-junction is when a segment
endpoint is latched to the middle of a second segment. This splits the
second segment in two.

\subsubsection{Erase}

Users may want to remove linework for various reasons: deleting
unwanted or accidental items, or as part of a deliberate strategy to
cut away geometry to allow new shapes to
emerge~\cite{zeleznik-lineogrammer}. Like latching, erasing is a
common task so it is given a simple, easy gesture.

Erasing is performed with a scribble gesture. During development we
tried various algorithms for detecting erasure. The first attempts
would were computationally intensive enough that they could only be
performed one time after the pen was released. However, test users
found that it was difficult to perform the gesture correctly. If it
was not done right, the input would be recognized as linework and
remain on the canvas, which would then need to be erased using the
same problem-prone algorithm. This caused considerable user
frustration.

Our new algorithm for detecting erasure is easy for the user to
perform, and It is efficient enough to run as the pen stroke is in
progress. When an erasure gesture is detected in mid-stroke, it
provides visual feedback that gives users confidence and avoids
frustration.

Erase (scribble) gestures are detected as follows. The first task is
to to assign each point $P_i$ with a time stamp $T_i$, a curvilinear
distance $D_i$, and a heading vector $H_i$. Curvilinear distance is
the path length along the stroke from the first point: $D_0$ is zero,
and the rest are $D_i = distance(P_{i-1}, P_i)$.

A pen stroke will not be considered an erasure if the pen has not
moved more than some minimum distance from the start point (we use 10
pixels).

The heading $H_i$ for point is a normalized vector from $P_{i-k}$ to
$P_{i+k}$, for a window size of some $k$ (we use $k=1$ but for more
sensitive input surfaces $k$ should be larger). The first $k$ points
use $H_k$ for their heading.

The next task is to add points to a list of samples $S$. If $D_i$ is
more than some threshold beyond the most recently added sample point,
$P_i$ is added to $S$. When a new sample point is added, it assembles
a sub-list $R$ of recent sample points that occurred within $t$
milliseconds (our implementation uses 100ms). If the angle between any
point in $R$ and the new point is greater than some value (we use $\pi$
radians), it increments a `corner' value for the current pen
stroke. When more than some number of corners is found for a stroke,
the system marks the current input as an erase scribble and stops
additional recognition until the pen is lifted.

Because the sample list depends on a relatively short time period, the
user must scribble with vigor to cause the erase gesture to
activate. If the user intends to erase but does not draw fast enough,
they quickly learn that they can just scribble a little faster and
wait for the visual feedback to tell them that their stroke is an
erasure.

\subsubsection{Undo and Redo}

Another way to recover from unwanted actions---particularly a sequence
of them---is to use undo. SIMI enables users to undo by holding the
offhand button down and dragging the pen to the left. Every 40 pixels
change in the x dimension causes an undo action. Redo is done by
dragging to the right. This lets the designer undo several actions by
simply dragging farther to the left. Both undo and redo actions can be
triggered by the same stroke by changing direction. This lets the
designer seek for a prior state by gesturing.

\subsubsection{Right Angle Constraints}

SIMI enables designers to add geometric constraints via drawn
gestures. In traditional drafting, a right angle is often indicated
with a brace symbol at the intersection of the two edges. SIMI
recognizes drawn input that looks like that brace and adds a new
constraint for the associated segments. Erasing either segment in a
right-angle constraint also removes the constraint.

\subsubsection{Length Constraints}

Another convention from drafting is to use hash marks to indicate that
two lengths are the same length. SIMI recognizes two or more hash
marks crossing line segments as a gesture to create a \textit{same
  length constraint}. If the user hashes a line segment that is
already involved in a same-length constraint, that segment is added to
the existing one. Removing a segment involved in a same-length
constraint only deletes the constraint if it is the last one.

A same-length constraint is satisfied when all the lengths are the
same. This tends to mean that the final length is the mean value of
the initial lengths.

SIMI also lets designers give specific length requirements. Currently
this is done by selecting a line (by over-tracing a portion of it) and
typing a number. Handwriting recognition would be preferred.

If one of the segments in a same-length constraint is given a
particular length, all segments take on that particular length.

\subsubsection{Flow Selection}

Flow selection~\cite{johnson-flow-selection} is a technique to
manipulate curves organically. The may `heat up' portions of curved
segments by holding the pen down nearby. Then, without picking the pen
up, the user can move the heated region by dragging the pen. Points
along the curve move more if they are `hotter'.

\subsubsection{Reference Points and Guides}

The user may create handles used to move segment endpoints around by
drawing `reference points'. These are dots made by swirling the pen
around in a small area (within 9 pixels) for a short time period (less
than 500ms). The user can then reposition them to adjust attached
geometry.

When reference points are present, the system shows guides that can
aid additional drawing. When there is one dot, the system uses the
pen's hover location, drawing a circle centered at the dot, and a line
passing through both. This can be useful if the designer wants to make
a hole centered at a particular location.

Two reference points give the user three circles, two lines, and shows
the midpoint. Three reference points give the user a circle that
passes through them, and indicates that circle's center.

\subsection{Constraints}

SIMI lets designers establish \textit{constraints} that enforce some
geometric relationship among items~\cite{borning-thinglab}. For
example, the user might draw a triangle and establish a right angle
constraint. No matter how the user manipulates the drawing (moving
vertices or changing segment lengths), the constraint engine will try
to ensure that particular corner remains a right angle.

SIMI uses an iterative, numeric solver that seeks to minimize the
total error of all constraints. A constraint's error is reported as
how far each related points must move until the constraint is
satisfied. However, points may be involved in many constraints, so it
is not generally possible to simply move points to where they satisfy
each constraint. To resolve contending constraints, the system
accumulates a change vector for each point by polling each
constraint. Each point is moved a small amount along its change
vector, and the process continues until the total error becomes very
small.

It is possible for the solver to get trapped in a loop as points
oscillate between several values. We use a `heat' metaphor as in
simulated annealing to avoid this case: the amount that points move
varies randomly, and is larger when there is more heat. Gradually the
system `cools off' and the points should settle in to a satisfactory
configuration.

Using a small number of primitive constraints related to distances and
angles, SIMI can create higher-level user constraints such as those
that keep segments at right angles or to be the same length.

\subsection{Stencils}

SIMI's final product is a ``cut file'': a vector drawing that can be
given to a laser cutter. This cut file typically contains a number of
stencils, which are closed 2D shapes that define the laser's
path. Stencils may have arbitrarily complex boundary geometry,
providing the edges do not intersect one another. Stencils can also
have any number of holes in them, which could be used for joints,
fasteners, or some other purpose.

To identify stencils, SIMI forms a graph with segment endpoints as
nodes and segments as edges. It then runs a depth-first search. The
longest path that forms a cycle from a given point is considered a
possible stencil. After completing the search, only the longest paths
are kept. Stencils are visually represented by shading the
interior. 

\section{EVALUATION}

* screenshots/photos from user study

* other results from user study...

\section{ACKNOWLEDGMENTS}

% TODO: fill this in later. Leave left blank for blind review.

\bibliography{simi}

\end{document}
