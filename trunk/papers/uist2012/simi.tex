\documentclass{article}

\usepackage{times}
\usepackage{uist}
%\usepackage[config, font=small, labelfont={sf,bf}, textfont=sf]{caption,subfig}
%\usepackage{setspace}
%\doublespace
\usepackage[config, font=small, labelfont={sf,bf}, textfont=sf]{caption}
\usepackage{subfig}
\usepackage{graphicx}

\begin{document}

% --- Copyright notice ---
%\conferenceinfo{UIST'11}{October 16-19, 2011, Santa Barbara, CA, USA}
%\CopyrightYear{2011}
%\crdata{978-1-4503-0716-1/11/10}

% Uncomment the following line to hide the copyright notice
\toappear{}
% ------------------------

\bibliographystyle{plain}

\title{Sketch It, Make It: Freehand Drawing\\
for Precise Rapid Fabrication}

\author{
\parbox[t]{9cm}{\centering
	     {\em Author One}\\
	     Institution Name\\
             City, ST, USA\\
	     user@institution.net}
\parbox[t]{9cm}{\centering
	     {\em Author Two}\\
	     Institution Name\\
             City, ST, USA\\
	     user@institution.net}
}

\maketitle

% TODO: change this
\abstract Abstract goes here. 

\classification{I.3.5 [Computational Geometry and Object Modeling]: Modeling packages}

% TODO: change this
\terms{Design, Human Factors}

\keywords{sketching, rapid fabrication, design tools, constraints}

\tolerance=400 % prevent words from sticking out in the margin

%% \begin{figure}[tb]
%% \vspace{1.9in}
%% \caption{A figure caption.  It is set in 9 point Helvetica type, with a
%% 0.5 cm wider margin on both left and right sides.} 
%% \label{fig-example}
%% \end{figure}

\section{INTRODUCTION}

A growing community of self-described \textit{Makers} design and build
many kinds of physical things~\cite{gershenfeld-fab}. Some are
electronic gizmos, while others are made entirely from traditional
materials. These ``new makers''~\cite{gross-new-makers} are empowered
by rapid fabrication machines such as 3D printers and laser cutters.

Laser cutters are among the more common and affordable fabrication
machines. They can be thought of as a very fast, strong, and precise
automated razor, cutting flat material (paper, wood, plastic, metal,
etc.). Many things can be made with only a laser cutter, sometimes
fastened with screws or glue.  Figure~\ref{fig:laser-example} shows
several examples of laser cut objects.

\begin{figure}[b]
\centering 
\subfloat[] {
  \label{fig:laser-example-a} 
  \includegraphics[width=0.4\linewidth]{img/flat-b.jpg}
}
\subfloat[] {
    \label{fig:laser-example-b}
    \includegraphics[width=0.4\linewidth]{img/flat-b.jpg}
}
\caption{Laser cut items.}
\label{fig:laser-example}
\end{figure}



Today, designers can choose among several modeling tools for laser
cutter projects. The most common is Adobe Illustrator, a
general-purpose, full-featured vector graphics editor. New users find
its interface familiar (at least superficially) because it resembles
other applications they have used previously. Despite this sense of
familiarity, participants in our formative study (presented later) had
a hard time using Illustrator quickly and effectively. Although
Illustrator is a powerful feature-rich tool it is intended primarily
for graphic design, and does not support the domain-specific activity
of designing for laser cutters. Other modeling tools like (Rhino or
SolidWorks) are perhaps more appropriate for this kind of modeling,
but also entail a substantial learning curve. To support design by
avocational users, appropriate modeling tools must be made
available~\cite{lipson-homefactory}.

We present ``Sketch It, Make It'' (SIMI), a modeling tool for laser
cutter design based on recognizing short sequences of input sketched
with a stylus. Using only freehand drawn input, SIMI enables a
designer to iteratively and incrementally create precise laser cut
models.

Research on sketch-based modeling tools~\cite{johnson-sketch-review}
typically describes sketching as an activity done mostly in the early
phases of design. Tools based on this assumption are justifiably
oriented towards capturing inprecise input while ignoring unimportant
detail. Only a few sketch-based systems support designers in later
stages when it is important to make
decisions~\cite{mori-plushie,saul-sketch-chair,naya-parsketch}.

% TODO: Mark asked for citations for the above statement about
% sketch-based systems that support late-phase design. I could offer
% ParSketch, Lineogrammer, Furniture Factory, SketchChair, and maybe
% dig up whatever is going on with Igarashi/Lipson/Stahovich labs. But
% as far as I know, this is not a very strong statement, as nobody
% that I know of really has a system that supports both fabrication
% *AND* human-in-the-loop precision.

We are inspired by the potential of freehand drawing as a basis for
precision modeling for several reasons. Sketching is quick and can be
easily learned. It is simple and modeless: unlike structured editing
software, a designer need not set a pencil's mode to line, circle, or
anything else. Yet (as we shall show), sketched input can provide
enough information to translate into a precise digital model.

\subsection{Research Contributions}

This paper offers four contributions: First we present the results of
a formative study on how people design objects for laser cutting. We
describe the process taken by avocational designers and their
difficulties in using structured modeling tools. We also analyze of
artifacts from popular Maker community web sites that reveal specific
design features common to many laser cutter projects.

Second, we introduce \textit{Sketch It, Make It} (SIMI), a tool
supporting incremental sketch-based modeling for designing precise 2D
shapes for laser cutting. Prior work on sketch-based modeling
techniques explored specific topics (e.g., corner finding, recognition
algorithms, mode-switching). We improve on existing sketch-based
interaction techniques and bring them together in an ensemble.

Third, we present an evaluation of SIMI. The system's success depends
on how well its individual features work together. We believe our
system gives users a fluid environment for designing, a belief
supported by this study.

\section{BACKGROUND}

Laser cut designs are typically composed of parts cut from solid, flat
pieces of material and assembled in various ways: laminated, notched,
bolted together, \textit{etc}. Most joints have small margins of
error. Lengths, angles, and relative position must be indicated
precisely so that the parts fit together properly.

The designer's primary concern is to specify part shapes to laser
cut. The software outputs a vector graphics file that defines these
shapes, with raster images indicating etching.

Many material types can be used, depending on the laser cutter. Common
media include wood, plastic, leather, and paper. Different materials
require different laser speed and intensity settings to achieve a
quality cut. Using medium-sized machines, hard material like metal can
be etched, but not cut. For example, a medium-sized machine has a cut
bed supporting material about 18 by 24 inches (approx. 46 by 61 cm). A
40 watt machine can cut through 3/8'' (1 cm) soft wood.

The laser leaves a gap in its wake, called a \textit{kerf}. The kerf
size depends on the speed and intensity of the laser. A typical laser
kerf is between 0.2mm and 0.5mm on a 40 watt cutter. This is an
important consideration when designing facets whose tolerances are
small with respect to kerf. A notch joint, for example, is ineffective
if it is 0.1 mm too large or small. When changing materials or laser
cutter settings designers often need to adjust their model to respect
the different kerf width. This is a time-consuming, trial-and-error
process.

The time to cut items depends on speed and intensity settings, and on
the size and complexity of the cut path. For example, the assembly
shown in~Figure\ref{fig:TODO} took about five minutes to fabricate and
used about \$3.50 (2012 USD) in material. 

% unlike some other past tools in sketching, precision is paramount
% here (explain why). This argument should go either here or when SIMI
% is first introduced in depth.

\section{RELATED WORK}

While ``the design process'' has important differences between
domains, two properties are common to many types of modern design
work, including design for laser cutters. First, designers sketch
throughout the process, especially at the outset. Second, a computer
tool is eventually used to render the design more precisely. These
properties are confirmed by observations of graphic
designers~\cite{wong-rr-prototypes}, automotive
engineers~\cite{kara-styling}, and software
developers~\cite{dekel-improvised-notation}. Sketching is a powerful
means for thinking and recording design intent, but as many have
observed, it is disconnected from the computer aided phase of the
design process~\cite{company-sketching-in-engineering}.

\subsection{Sketch-based interaction}

The rough appearance of freehand sketches encourages designers to see
beyond unimportant details, letting them make big-picture
decisions. Much prior work argues that beautification (e.g. redrawing
crudely drawn lines as straight segments) is antagonistic to
design~\cite{gross-cocktail}, at least during the conceptual
phases. SILK~\cite{landay-silk-chi} and DENIM~\cite{lin-denim} are
systems that take this perspective. They are sketch-based tools for
quickly drawing user interfaces. SILK lets users quickly draw GUIs,
recognizing input as UI elements like menus, scrollbars or buttons and
transforms the recognized sketch to a working implementation, while
retaining its rough appearance. DENIM is a drawing tool for creating
rough models for web sites. It recognizes a small set of
context-sensitive gestures.

Designing roughly helps the user avoid thinking about details that are
not yet important. For example letting designers place an interface
element, without needing to specify that it is 12 pixels to the left
of some other element. Eventually some of these values must be given
explicitly. Since SILK and DENIM are not designed for that purpose, a
second tool (e.g. a text editor) must be used.

% So: SILK throws it over the wall when conceptual design is done,
% expecting the designer to finish the job in code (e.g. how does one
% set the exact pixel alignment or padding if not in code?) SIMI lets
% users give precise details without the need for help from grown-up
% tools.

Some work takes an alternate view on the appearance of sketched
input. Systems such as Pegasus~\cite{igarashi-pegasus} and recent work
by Murugappan \textit{et. al}~\cite{murugappan-beautification}
`beautify' the drawing by replacing rough input with cleaned-up lines
or arcs. These 2D graphics systems infer the user's intention by
detecting geometric relationships. If more than one relationship is
possible, the system lets the user choose among a few
alternatives. These tools enable users to quickly and easily make
simple vector graphics. However, the aggressive inferencing can make
it difficult to make drawings that have subtle features the system can
not detect.

Sketch input is an appealing way to interact with computers because
users find it easy to provide. Unfortunately, sketch recognizers are
not yet sophisticated enough to reliably interpret arbitrary
drawings. Therefore researchers have created ways to close the gap
between input that people provide and the computer's ability to make
sense of it. For example a system may require users to draw in certain
ways (e.g. shapes must be drawn with single strokes, as in SILK) to
conform to the recognizer's capabilities.

While inferencing is powerful, it also prevents users from making
subtle distinctions. For example, an overly zealous inferencing engine
might snap objects together when ther designer would like to them to
simply be near one another without touching. 
% SIMI enables this level of control.

Well-known systems such as Teddy~\cite{igarashi-teddy} and
EverybodyLovesSketch~\cite{bae-everybody} are among those that develop
or refine sketch-based interaction techniques that that are both
natural for humans to use and easier for the computer to understand.
They provide a small grammar of easy-to-make gestures for people to
create and edit 3D drawings. The ease and power of these systems is
evident: even children can learn and use them to make complex
models. EverybodyLovesSketch in particular seeks to enable a broad
audience to create 3D perspective conceptual sketches by providing a
set of gestures and tools that work well together. However, it is hard
to engineer with these tools because they do not provide users the
ability to give dimensions to arbitrary lengths or angles.

% So: power and ease of use are not mutually exclusive, but it
% requires careful tuning of interaction techniques to make it
% work. SIMI builds on this.

\subsection{Sketch-based modeling for fabrication}

Computer support for fabrication design has been a topic of interest
for decades, under the rubric of computer aided design (CAD) and
computer aided manufacturing (CAM). While today interaction is mostly
performed with a keyboard and mouse, this was not always the case. For
example, SketchPad~\cite{sutherland-sketchpad} users controlled the
design by setting modes and parameters using one hand, while drawing
on the screen with a light pen in the other.

More recently, novel interfaces enable users to model items for
fabrication by sketching. For example, people can use
Plushie~\cite{mori-plushie} to design soft objects such as stuffed
animals. Users begin by creating 3D models of bulbous objects by
sketching shape outlines in a manner similar to
Teddy~\cite{igarashi-teddy}. The program outputs a set of 2D shapes
that users can cut from fabric, sew, and stuff.

Sketch Chair makes design for rapid fabrication more
accessible~\cite{saul-sketch-chair}. Users sketch the contours of a
chair's seat and back rest, and (in a different drawing mode) add
legs. The system includes a sophisticated physical simulator to let
the designer explore design consequences (for example to determine
whether will remain upright).

Domain-oriented tools such as Plushie and Sketch Chair enable people
to make things they otherwise would be unable to, but the designer
relinquishes a great deal of control to the system. In contrast, SIMI
users retain the ability to specify as much or as little as they like,
but without the help from domain-centric computation.

% TODO: Mark says: the above paragraphi is important, but it's not
% clear how. For example plushie doesn't let you make whatever you
% want? That is, your point here is that these systems constrain the
% user to a specific design space, but is that true?

SIMI builds on the work of a fairly small but interesting set of
sketch-based systems that support precision. With
ParSketch~\cite{naya-parsketch} designers create parametric 2D models
by incrementally recognizing sketched geometry and commands. It uses
pressure to distinguish between linework (high pressure) and
constraint commands (lower
pressure). Lineogrammer\cite{zeleznik-lineogrammer} is a sketch-based
2D drawing tool that lets users make rectified vector graphics. Like
Pegasus~\cite{igarashi-pegasus}, it works on the principle of
interactive beautification, supporting iterative sketch/rectify
sequences. The interactive nature of these precision-oriented systems
means the system has less work to do when its recognizer/rectifier is
invoked, leading to more accurate recognition and more satisfied
users.

% TODO: Jason says: Again, want to end with a stronger differentiator
% of what's new and interesting and different here from past work. Is
% it the integration? Is it the specification of precision? Is it the
% domain?  (Could argue it's more general than SketchChair)

\section{BACKGROUND STUDY}

To better understand the design practices involved with laser cutter
fabrication we conducted two related formative studies. We interviewed
people with experience designing these artifacts and watched them
work. We also surveyed and analyzed laser-cut items found on community
web sites to find common features. We conducted these studies to
better understand the kinds of tasks and problems designers face when
designing for laser cutters, which in turn informed SIMI's
development.

\subsection{Formative Study on Designer Work Practices}
\label{sec:formative}

We interviewed six designers to learn about their work practices and
to understand how they use their tools. All participants had
substantially different backgrounds, including mechanical engineering,
graphic design, and architecture. They were experienced with designing
for and using laser cutters. The participants were members of our
tool's target demographic: people who make (or want to make) things
with laser cutters as a hobby.

Each session lasted approximately an hour and was split evenly between
an interview and using software. We met participants in their own work
environments (offices or labs). We asked participants to describe
their design process and to show sketches or videos of their
work. Although there are differences (some subtle) in their process,
each followed the same overall pattern.

They all begin by thinking about a problem and sketching. Some
sketches are made to think about how to frame the project (what it is
for). Others help reason about how to make it (how it works, how it
fits together). Some designers explicitly note that sketching is a
necessary part of the process: it would be impossible to move forward
without making freehand drawings. Only after the idea is reasonably
well-formed do they translate their hand-made sketch into a computer
model (Figure~\ref{fig:translate}).

\begin{figure}[h]
  \centering
  \includegraphics[width=0.9\linewidth]{img/translate-sketch-to-computer.jpg}
  \caption{A common part of designing for laser cutters: translating a
    hand-made sketch to a computer modeling tool. The sketch includes
    a perspective drawing of the desired result, and 2D diagrams of
    individual parts with key dimensions indicated.}
  \label{fig:translate}
\end{figure}

After the work practices interview, we asked participants to recreate
the sketch shown in Figure~\ref{fig:interview-sketch} using a software
tool of their choice. We wanted to learn what problems people
encountered when executing the common task of translating a sketch to
a computer model.

\begin{figure}[h]
\centering 
\subfloat[The sketch users were given to replicate.] {
  \label{fig:interview-sketch-1} 
  \includegraphics[width=0.9\linewidth]{img/laser-me-1.jpg}
}

\subfloat[Drawing of how the part is used in context.] {
    \label{fig:interview-sketch-2}
    \includegraphics[width=0.9\linewidth]{img/laser-me-2.jpg}
}
\caption{Participants were asked to create the stencil at the top
  using modeling software.}
\label{fig:interview-sketch}
\end{figure}

Most participants chose to implement the sketch with Illustrator (5
users); one chose Rhino. In every case, the designer's strategy
involved common activities: creating or editing boundaries, aligning
or snapping items, using guide lines or reference points, measuring
distances, specifying or changing lengths and angles, and creating
finished ``cut files'' to send to the laser cutter.  They also engaged
in typical interaction management tasks such as selecting/deselecting
on-screen elements, or viewport management such as zooming and
panning.

Participants spent a good deal of time on operating overhead
(approximately 50\%). These tasks included searching for the
appropriate tool for the next task and recovering from errors. For
example, one designer, and experienced Illustrator user was aware the
``Path Finder'' tool and wanted to use it. The user searched the
program's menu structure and hovered over toolbar buttons to read tool
tips before finding it. Next, he invoked various functions of the Path
Finder, using the keyboard shortcut to undo after each failed attempt,
as he searched for the correct mode within the subcommand
palette. This process lasted approximately 80 seconds.

Occasionally participants used features in unorthodox ways to achieve
a desired outcome. For example, in order to remove an unwanted segment
of a polyline, one participant (a graphic designer) created an opaque
white rectangle to obscure it, rather than erase it. (``Don't tell
anyone I did this'', he said at the time).

Similar episodes are common: a person \textit{should} know the
`correct' action, but takes an alternate approach. Although the
alternative achieves the intended effect, it might be less efficient
(more operations, longer execution time) or it introduces unwanted
complexity (such as the invisible white rectangle).

To summarize, we found most common tasks and problems found in
interview study belong to three main groups:

\begin{itemize}
\item \textit{Defining geometry:} Creating/editing boundaries,
  aligning items, creating and using guide lines or reference points,
  measuring distances, and specifying lengths or angles.
\item \textit{Managing the editing tool:} Selecting/deselecting
  objects, view port management, finding and entering tool modes, and
  recovering from errors.
\item \textit{Cutfile:} Finalizing the cutfile by creating copies of
  items when more than one is needed, and positioning stencils.
\end{itemize}

\subsection{Artifact Analysis}

The formative study from the previous section helps us understand
\textit{how} people create laser cut items. To learn more about the
characteristics of those objects, we analyzed finished items from two
web-based communities.

Many users are motivated by the opportunity to share their designs
with others. Ponoko and Thingiverse are two currently popular web
sites for selling or sharing items that can be made with rapid
fabrication machines. Ponoko offers thousands of user-designed items
for sale, mostly produced by laser cutting. Thingiverse is a warehouse
of digital models that contains 3D-printable objects and designs for
laser cutters. From these two sites we selected a total of 55
laser-cut projects. On Ponoko we selected the most recent 50 items
that were made with laser cutters. Five were later disqualified
because they were too similar to other items in the set, bringing the
total to 45 items from Ponoko. On Thingiverse we searched for objects
with the ``laser cutter'' tag. We then examined the features of these
55 projects to understand what people are making with laser
cutters. The results of this feature analysis are summarized in
Figure~\ref{fig:ponoko}.

\begin{figure}[h]
  \centering
  \includegraphics[width=0.9\linewidth]{img/ponoko-graph.pdf}
  \caption{Frequency of certain features present in 55 laser-cut
    designs found on Ponoko and Thingiverse.}
  \label{fig:ponoko}
\end{figure}

We used ten properties to characterize each project, based on our own
experience designing objects for laser cutters, as well as
observations from the formative study. These properties are summarized
here and described in greater detail below:

\begin{itemize}
\item \textit{Symmetry}: Radial or linear symmetry is a dominant feature.
\item \textit{Repeating geometry}: Linework is repeated several times,
  often in a way that suggests procedural generation.
\item \textit{Right Angle}: Edges meet at 90-degree angles.
\item \textit{Notch and Finger Joints}: Two parts come together using one of
  the joints illustrated in Figure~\ref{fig:joint}.
\item \textit{Rounded Corners}: Right-angle corners are slightly blunt.
\item \textit{Splines}: Curved linework (not counting rounded corners).
\item \textit{Single Flat}: The project is composed of a single, flat
  piece of material (e.g. a coaster).
\item \textit{Fasteners}: Clear use of glue, screws, or bolts.
\item \textit{Raster etch}: Laser cutter etched patterns (e.g. words,
  images) rather than cutting all the way through material.
\end{itemize}

\begin{figure}[h]
\centering 
\subfloat[Notch joints.] {
  \label{fig:joint-notch} 
  \includegraphics[width=0.4\linewidth]{img/joint-notch.png}
}
\subfloat[Finger (box) joints.] {
    \label{fig:joint-finger}
    \includegraphics[width=0.4\linewidth]{img/joint-finger.png}
}
\caption{Two common methods to join parts. Notch joints are used when
  parts intersect along part midsections; finger joints (alternately
  called box joints) join parts along edges.}
\label{fig:joint}
\end{figure}

%% Single part projects were typically artistic, using expressive, curvy
%% linework. Among multi-part objects, approximately 75\% used finger or
%% notch joints (Fig.~\ref{fig:joint}). The rest used fasteners.

%% The more professional-looking models were those that used rounded
%% corners. Several designs used raster etching for artistic
%% effect. Repeating geometry was found in most models. These patterns
%% involve sequences of lines or curves with consistent length and
%% angles. Some patterns were quite ornate.

\section{SKETCH IT, MAKE IT}

Based on observing designer's work practices (formative study) and the
artifacts they make (artifact analysis), we developed \textit{Sketch
  It, Make It} (SIMI), a sketch-based tool for modeling laser cut
items. We aim to address problems with current modeling systems
enumerated above in a tool that specifically supports designing
laser-cut items.

Design for laser cutting requires precision. The greatest distinction
between SIMI and prior sketch-based design tools is that users can
accurately specify geometry. That is, the user may change lengths,
distances, to specific values. But in keeping with the spirit of
sketch-based modelling, the user is not obliged to give details if
they are not important.

%% The focus of this work is to produce a useful and usable sketch-based
%% design tool for the domain of laser cutting. The main challenge in
%% achieving this was the lack of a set of interaction sketch-based
%% techniques that work in harmony. Prior work has focused largely on
%% individual techniques that clearly map problems to solutions. In
%% contrast, our system address the challenge of developing a system that
%% integrates many experimental interaction techniques into a coherent
%% whole.

SIMI users draw with a stylus, using a button with thier other hand
for a few actions. The system recognizes input as either geometric
linework or gestural commands. Linework includes straight lines,
elliptical arcs, splines, circles, and ellipses.

Users invoke commands to operate on linework by drawing gestures. Some
gestures are recognized and execute immediately, such as the erase
(scribble) gesture. Others, such as a command to constrain segments to
be the same length, are recognized after the user presses the button,
or after a timeout.

The system recognizes a closed 2D path as a `stencil'. Stencils are
shapes that can be placed on the virtual laser cutter bed. Several
copies of a stencil can be added. The system generates a vector file
for laser cutting.

After cutting stencils, the user assembles the cut parts into their
final configuration. Figure~\ref{fig:laser-example} shows examples of
projects made with SIMI.

\subsection{Sketch Interaction Techniques}

\begin{figure}[h]
\centering \subfloat[Automatic: merge when endcaps intersect
  (drawn in blue).] {
  \label{fig:latch-auto} 
  \includegraphics[width=0.43\linewidth]{img/latch-auto-endcaps.pdf}
}\hspace{5mm}
\subfloat[Endpoint latching.] {
  \label{fig:latch-endpoint} 
  \includegraphics[width=0.43\linewidth]{img/latch-manual-endpoint.pdf}
}
\\
\subfloat[Continuation latching.] {
    \label{fig:latch-continuation}
    \includegraphics[width=0.43\linewidth]{img/latch-manual-continuation.pdf}
}\hspace{5mm}
\subfloat[T-Junction latching.] {
    \label{fig:latch-tjunct}
    \includegraphics[width=0.43\linewidth]{img/latch-manual-tjunct.pdf}
}
\caption{Automatic and manual latching used to bring segments together.}
\label{fig:latch}
\end{figure}


%* Stylus with offhand button



% TODO: Jason says: Maybe even make this a philosophical argument,
% wanted to see how far you could push pen-only interactions

Guiding the development of SIMI is the principle that the designer
should never need to set down the pen. Input is provided entirely with
a stylus except for a single button used by the non-dominant hand that
gives access to additional commands. The gestures used to invoke
commands or add constraints are summarized below.

\subsubsection{Latching}

Users often want adjacent lines to meet at a common point. In the
formative study from the pervious section, we noticed people
struggling to make lines coterminate. Sometimes the designer would
simply extend the lines past each other to ensure the laser path will
make the corner correctly.

Latching is the process of combining adjacent segments (lines,
splines, arcs, \textit{etc.}) to meet at a common point. For example,
users may draw a square with with four strokes, and eight unique end
points, but a square should have only four corners.

SIMI provides two methods for latching segments, illustrated in
Figure~\ref{fig:latch}. One is automatic: the system analyzes new
linework for cases where the user probably meant their segments to
join together, and adjusts one or more segments to join. However,
automatic latching can be problematic if it is too zealous. Early
approaches used only distance to decide which segments should be
latched~\cite{herot-latch-corners}: if two endpoints were within $x$
units of one another, latch them. This works well when segment lengths
are large relative to $x$. But when the segments are short, a naive
latcher merges points that the designer intended to remain distinct.

SIMI's automatic latcher uses length and the segment direction at the
endpoint. It creates an \textit{endcap}: a short line segment centered
at an end point that extends the line by a fraction of its length
(currently 0.1). For nearby segments to latch automatically, their
endcaps must intersect. SIMI can latch two or more segments this way.

The automatic latching process is intentionally conservative to avoid
frustrating users. Therefore it often misses cases where the user
wanted lines to meet. SIMI gives users a simple method to latch
segments missed by the automatic latcher: draw a small circle around
the endpoints to be latched.

All linework in SIMI is meant to compose stencils, which are closed
sequences of latched segments. Therefore the designer must be able to
find and fix un-latched segments. To reveal un-latched segments the
system draws a red marker at lonely endpoints.

Three different spatial arrangements can be latched: endpoint
latching, continuation, and T-junctions (see
Figure~\ref{fig:latch}). Endpoint
latching~(Figure~\ref{fig:latch-endpoint}) is what the automatic
latcher does. Continuation
latching~(Figure~\ref{fig:latch-continuation}) is when the user brings
together two segments that are close to the same direction at the
joined point. Continuation latching replaces two segments with a
single larger segment. A T-junction~(Figure~\ref{fig:latch-tjunct}) is
when a segment endpoint latches to the middle of another segment,
splitting the second segment in two.

\subsubsection{Erase}

\begin{figure}[h]
  \centering
  \includegraphics[width=0.9\linewidth]{img/erase-all.pdf}
  \caption{Erase gesture: before, during, and after.}
  \label{fig:erase}
\end{figure}

Users may want to remove linework for various reasons: deleting
unwanted or accidentally drawn lines, or as part of a deliberate
strategy to cut away geometry to allow new shapes to
emerge~\cite{zeleznik-lineogrammer}. Like latching, erasing is a
common task so it is invoked with a simple scribble gesture. 

%% During development we tried various algorithms for detecting the
%% scribble. The first attempts were computationally intensive---they
%% could only be performed once after the pen was released. However, test
%% users found it difficult to perform the gesture correctly. Worse, when
%% done incorrectly the input would be recognized as linework and remain
%% on the canvas. This raises another need to erase using the same
%% problematic algorithm. This caused considerable frustration.

Our algorithm for detecting erasure executes efficiently during the
pen stroke. When an erasure gesture is detected mid-stroke, it
provides visual feedback that gives users confidence and avoids
frustration. Figure~\ref{fig:erase} shows an erase gesture with the
visual feedback.

%% Erase (scribble) gestures are detected as follows. First we assign
%% each point $P_i$ with a time stamp $T_i$, a curvilinear distance
%% $D_i$, and a heading vector $H_i$. Curvilinear distance is the path
%% length along the stroke from the first point: $D_0=0$, and the
%% rest are $D_i = D_{i-1} + distance(P_{i-1}, P_i)$.

%% A pen stroke is not considered an erasure if the pen has moved less
%% than a minimum distance from the start point (we use 10 pixels).

%% The heading $H_i$ is a normalized vector from $P_{i-k}$ to $P_{i+k}$,
%% for a window size of some $k$ (we use $k=1$ but for higher resolution
%% input surfaces $k$ should be larger). The first $k$ points use $H_k$
%% for their heading.

%% Next we add points to a list of samples $S$. If $D_i$ is more than
%% some threshold beyond the most recently added sample point, $P_i$ is
%% added to $S$. When a new sample point is added, it assembles a
%% sub-list $R$ of recent sample points that occurred within $t$
%% milliseconds (our implementation uses 100ms). If the angle between any
%% point in $R$ and the new point is greater than some value (we use
%% $\pi$ radians), it increments a `corner' count value for the current
%% pen stroke. When enough corners are found (we use 5) for a stroke, the
%% system draws feedback to alert the user that the gesture has been
%% recognized and halts recognition until the pen is lifted.

%% Because the sample list depends on a relatively short duration, the
%% user must scribble vigorously to activate the erase gesture. The user
%% who intends to erase but draws slowly, quickly learns to scribble a
%% little faster and wait for the visual feedback.

\subsubsection{Undo and Redo}

SIMI provides a novel means of performing Undo and Redo that lets
designers browse their entire design history. Participants in the
formative evaluation used the Undo and Redo features in two distinct
ways: to revert state following epistemic actions, or to recover from
usability problems~\cite{akers-undo}. First, Undo gives designers
confidence to modify their work to see what changes might look
like. Epistemic actions~\cite{kirsch-epistemic-action} are taken to
ask ``what if'' questions, like rotating an object 90 degrees to see
if it that orientation is better. Such actions support creative
exploration. If the designer does not like their modifications they
simply Undo to a prior state. The second class of Undo events stems
from errors: either usability problems or user error. 

Users undo by holding down the offhand button and dragging the pen to
the left. Every 40 pixels left triggers one undo action. This lets the
designer undo several actions by simply dragging farther to the
left. Redo is done by dragging to the right. Both undo and redo
actions can be triggered by the same stroke by changing direction,
letting designers scan for a desired state.

\subsubsection{Right Angle and Length Constraints}

Most laser cut stencils employ right angles, symmetry, and repeated
geometry (see Figure~\ref{fig:ponoko}). Designers can create stencils
that have these properties by imposing geometric constraints. A
constraint is a rule that enforces some mathematical property, usually
in the form of a geometric relationship among two or more
elements.

\begin{figure}[h]
  \centering
  \includegraphics[width=0.9\linewidth]{img/constraints-all.pdf}
  \caption{Gestures for adding a right angle (left) and same-length.}
  \label{fig:constraints}
\end{figure}

In SIMI, designers add constraints by drawing gestures. In traditional
drafting and standard geometry diagrams, a brace symbol at the
intersection of the two edges indicates a right angle. SIMI recognizes
drawn input that looks like that brace and adds a constraint on the
associated segments. These are shown in
Figure~\ref{fig:constraints}. Erasing either segment in a right-angle
constraint also removes the constraint.

Another drafting convention uses tick marks (hash marks) to indicate
that two lines are the same length. SIMI recognizes two or more ticks
crossing line segments as a gesture to create a \textit{same-length
  constraint}. A same-length constraint is satisfied when all segment
lengths are equal. The target length is the mean value of the
constituent lengths.

SIMI also lets designers set specific lengths, invoked by selecting a
line (by over-tracing a portion of it) and typing a number. If one of
the segments in a same-length constraint is assigned a particular
length, all segments take on that particular length.

\subsubsection{Flow Selection}

% Justify wrt user study and Ponoko analysis

About one-third of the models examined in our laser cut artifact
analysis involved curves that are typically modeled using splines
(e.g. Bezier curves). SIMI provides Flow
Selection~\cite{johnson-flow-selection} to enable users to manipulate
create and modify splines (Figure~\ref{fig:fs}). The user `heats up'
portions of curved segments by holding the pen down near the
curve. Then, without picking up the pen, the user deforms the heated
region by dragging the pen. ``Hotter'' points along the curve move
more.

\begin{figure}[h]
\centering \subfloat[Selecting (``heating'') points along a curve near
  the stylus. The selection grows as long as the stylus is held down. ] {
  \label{fig:fs-1} 
  \includegraphics[width=0.4\linewidth]{img/fs-1.pdf} }
\hspace{3mm} \subfloat[Deforming the region by moving the
  stylus. ``Hotter'' points (close to the pen) are moved more than
  those farther away.] {
  \label{fig:fs-2} 
  \includegraphics[width=0.4\linewidth]{img/fs-2.pdf}
}
\caption{Flow selection.}
\label{fig:fs}
\end{figure}

% TODO: add to flow selection section.

% \subsubsection{Reference Points and Guides}

% Justify wrt user study and Ponoko analysis

%% In our study on designers, we noticed people measuring distances and
%% drawing temporary lines used to anchor subsequent linework. For
%% example, the stencil illustrated in
%% Figure~\ref{fig:interview-sketch-1} calls for a semi-circular region
%% in the middle of the top edge. Participants noted that their preferred
%% strategy for making this stencil was to draw a long line along the
%% top, and add the circular region afterwards. However, none of the
%% Illustrator users were able to successfully take this strategy because
%% they did not know how to find the midpoint of the line. Instead of
%% carrying out their desired plan they measured the distance of the top,
%% and added co-linear lines with half the length that met in the
%% middle. This was an error-prone and time-consuming process. To support
%% strategies like the one mentioned above, SIMI gives users the ability
%% to use reference points and guides.

%% \begin{figure}[h]
%%   \centering
%%   \includegraphics[width=0.9\linewidth]{img/guides-all.pdf}
%%   \caption{Reference points and guides. The first three panels show
%%     one, two, and three reference points and the guides that are
%%     displayed as a result. The designer uses the circular guide to
%%     create the circular arc shown in the final panel.}
%%   \label{fig:guides}
%% \end{figure}

%% The user may create handles to move segment endpoints around by
%% drawing `reference points'. These are dots made by swirling the pen
%% around in a tiny area (within 9 pixels) quickly (less than 500ms). The
%% user can then reposition the reference points to adjust attached
%% geometry by dragging them.

%% When reference points are present, the system offers guides to aid
%% additional drawing. The possible combinations are illustrated in
%% Figure~\ref{fig:guides}. When there is one reference point, the system
%% uses the pen's hover location, drawing a circle centered at the
%% reference point, and a line passing through both. This can be used to
%% make a hole centered at a particular location.

%% Two reference points give the user three circles, two lines, and shows
%% the midpoint. Three reference points give the user a circle that
%% passes through them, and indicates that circle's center.

\subsection{Constraint Engine}

SIMI lets designers establish \textit{constraints} that enforce
geometric relationships among items~\cite{borning-thinglab}. For
example, the user might draw a triangle and establish a right angle
constraint. No matter how the user manipulates the drawing (moving
vertices or changing segment lengths), the constraint engine maintains
that particular corner as a right angle.

SIMI's iterative, numeric solver minimizes the total error of all
constraints. A constraint's error is computed as how far each related
point must move to satisfy the constraint. However, a point may be
involved in several constraints, so it is not generally possible to
simply move points to where they satisfy one constraint because it
might break one or more other constraints. To manage contending
constraints, the system computes a change vector for each point by
computing the change required by all related constraints. Each point
moves a small amount along its change vector, and the process
continues until the total error becomes minuscule.

The solver can get trapped in a loop as points oscillate between
several values. We use simulated annealing~\cite{metropolis-annealing}
to avoid this case: the amount that points move varies randomly, and
is larger when there is more entropy. Gradually the system reduces the
level of randomness and the points settle in to a satisfactory
configuration.

\subsection{Stencils}

SIMI's final product is a ``cut file'': a vector drawing for a laser
cutter. This cut file typically contains a number of stencils, which
are closed 2D shapes that define the laser's path. Stencils may have
complex boundary geometry with non-intersecting edges. Stencils can
also have any number of holes in them, for joints, fasteners, or other
purposes.

To identify stencils, SIMI forms a graph with segments as edges and
their endpoints as nodes. It then runs a depth-first search. Cycles
from a given point is a candidate stencil. After completing the
search, only the longest paths are kept. Stencils are visually
represented by shading the interior.

\section{SYSTEM EVALUATION}

We evaluated Sketch It, Make It in two ways. First, we conducted a
workshop with undergraduate architecture students. Second, we compare
experienced SIMI user's strategy for making an object with that of
experienced Illustrator users.

\subsection{Student Workshop}

We held a workshop with 60 undergraduate architecture students in
order to gather qualitative feedback about how easy or difficult SIMI
was to learn and use. The primary author ran the workshop, held over
two days in a room equipped with iMacs and Wacom Intuos tablets. To
accommodate everybody, each day was divided into three sessions of
twenty students. Each session lasted half an hour, meaning each
student used SIMI for an hour.

% thing 1: Using tablet was difficult.

Initially, students complained that the tablet hardware was difficult
to use, but they acclimated to it after about ten minutes. 

% thing 2: Very little confusion on how to make geometry and
%    constraints. Erasing was difficult until they were shown how to
%    do it.

After they had adjusted to using the tablets, they quickly learned to
make linework and create constraints. At first they had trouble with
erasing and using control dots, but learned to use these features in
time. The students had access to a video demonstrating how various
techniques were performed, as well as a paper ``cheet sheet'', but did
not use them. Rather than use these resources, the students would ask
the primary author. Receiving personal help seemed to be far superior
because assistance could be tailored to their particular needs.

% thing 3: Most comments were about missing features, not about how to
%    use the system.

We were expecting students to have a harder time using SIMI because
the hardware (tablets) and interaction paradigm (sketch-based
modeling) were both new to them. However, on the second day, most
questions and comments regarded missing features, not about how to use
the system. 

The students were given an extra credit assignment to complete a short
survey. \textit{Results of survey here... probably a couple paragraphs
  and a table/figure for the data. Hopefully the above statement about
  questions/comments being about missing features is corroborated by
  the survey results.}

\subsection{Expert Designer Strategy Analysis}



\section{IMPLEMENTATION}

SIMI is programmed in approximately 20,000 lines of Java code. It uses
the Java binding for OpenGL (JOGL) for graphics and NIST's JAMA
package for linear algebra. When packaged as an executable, the Mac OS
X application is 8.2 megabytes.

\section{FUTURE WORK}

Future work...

\section{CONCLUSION}

Conclusion...

\section{ACKNOWLEDGMENTS}

% TODO: fill this in later. Leave left blank for blind review.

Acknowledgements...

\bibliography{simi}

\end{document}
