% Keep these two lines in order for typesetting to work w/ the funny fonts.
%!TEX TS-program = xelatex
%!TEX encoding = UTF-8 Unicode

\documentclass[12pt]{article}
\usepackage[margin=1.25in]{geometry}
\usepackage[config, font=small, labelfont={sf,bf}, textfont=sf]{caption,subfig}
\usepackage{graphicx}
\usepackage{hyperref}
\usepackage{amssymb}
\usepackage{hyperref}
\hypersetup{
    colorlinks,
    citecolor=black,
    filecolor=black,
    linkcolor=black,
    urlcolor=black
}
\usepackage{fontspec,xltxtra,xunicode}
\usepackage{setspace}
\defaultfontfeatures{Mapping=tex-text}
%\setromanfont[Mapping=tex-text]{Baskerville}
\setromanfont{Fanwood Text}
%\setsansfont[Scale=MatchLowercase,Mapping=tex-text]{Gill Sans}
%\setsansfont[Scale=MatchLowercase]{Cabin}
%\setmonofont[Scale=MatchLowercase]{Andale Mono}
%\usepackage[margin=10pt,font=sf,labelfont=bf, labelsep=endash]{caption}

%% \usepackage{titlesec}
%% \newfontfamily\sectionfont{League Gothic}
%% \titleformat*{\title}{\Huge\bfseries\sffamily}
%% \titleformat*{\section}{\Huge\bfseries\sffamily}
%% \titleformat*{\subsection}{\Large\bfseries\sffamily}
%% \titleformat*{\subsubsection}{\large\bfseries\sffamily}

\newcommand{\sectionline}{%
  \vspace{\baselineskip}%
  \hspace{\fill}\rule{0.5\linewidth}{.7pt}\hspace{\fill}%
  \par \vspace{\baselineskip}
}

\newenvironment{packed_enum}{
\begin{enumerate}
  \setlength{\itemsep}{1pt}
  \setlength{\parskip}{0pt}
  \setlength{\parsep}{0pt}
}{\end{enumerate}}

\newenvironment{indentwholepara}[1]
{\begin{list}{}
         {\setlength{\leftmargin}{#1}}
         \item[]
}
{\end{list}}

\title{Thesis Proposal:\\ Interaction Techniques for Sketch-based
  Design} \author{Gabe Johnson\\School of Architecture\\Carnegie
  Mellon University\\ \\ (This is a draft)}

\begin{document}
\thispagestyle{empty}
\maketitle
\vspace{70pt}
\begin{center}Committee\\
\vspace{24pt}
\rule{0.8\linewidth}{0.7pt}
\\
\vspace{24pt}
\textit{Mark D. Gross} (Chair) --- School of Architecture --- CMU\\
\vspace{8pt}
\textit{Ellen Yi-Luen Do} --- College of Architecture \& College of Computing --- Georgia Tech\\
\vspace{8pt}
\textit{Jason I. Hong} --- Human Computer Interaction Institute --- CMU\\
\end{center}

\newpage

\singlespacing
\tableofcontents
\doublespacing
\newpage

\doublespacing % Note: there is also a doublespacing command after the
               % page of figures, since that page changes spacing
               % temporarily.
\addcontentsline{toc}{section}{Abstract}
\begin{abstract}
Rapid prototyping machines like laser cutters and 3D printers are
becoming more common. However, the associated modeling software has a
significant learning challenge for novice users. I propose to create a
modeling system that allows for design of objects via freehand
sketching, a skill that most people already have. Sketch-based
interaction is viewed as a promising design method, but unlike
traditional WIMP applications there are no standard interaction
techniques for sketch-based applications. The proposed system will
explore sketch-based interaction techniques for moderately complex
tasks such as the design of objects for production via laser cutter. I
will develop a series of sketch-based interaction techniques that are
appropriate for their tasks and work well as an ensemble. 
Additionally, I will test the utility of this system against existing
design methods and tools.
\end{abstract}

\newpage

\section{Problem Statement}

In recent years, computational support for sketching has focused
mostly on the early phases of design. This is for good reason: paper
and pencil sketching remain central to design practice despite the
ubiquity of powerful computer-based design tools. Sketching allows
people to quickly jot down ideas or exchange thoughts with others;
they provide a medium through which designers think about problems and
potential solutions. Freehand drawing is an activity that nearly all
designers---professional and avocational alike---readily employ.

Ideas that begin as rough sketches may eventually be prototyped with
rapid prototyping machinery such as 3D printers, CNC routers, or laser
cutters. Over time, this sort of machinery will become more
affordable, utilize a wider range of materials, and produce
higher-quality output. Rapid fabrication machines offer the potential
to enable people to take active control of the design of their world,
rather than being passive consumers of products somebody else has
made.

However, a gap remains between the technology used to design and the
technology used to prototype physical items. Today, a typical design
process might involve several paper sketches, followed by a session
with a computer modeling system such as SketchUp, SolidWorks, or
Rhino. When the designer is satisfied with the CAD model, the rapid
fabrication machinery is put to work. The designer can then evaluate
the output and decide what to do next: keep the design as it is,
change the CAD model, or ``go back to the drawing board'' and sketch
new ideas. It is common for people to physically sketch while making
revisions or redesigning, because it is often easier and faster than
working with existing CAD models. This is especially the case with the
users targeted by this system: avocational designers who are not
necessarily proficient with professional tools.

The software proposed here is in a category that does not yet exist
outside of research labs. Consequently, it is difficult to gauge what
problems users have with such software without building it. I argue
that it is beneficial to look beyond current user needs and invent
tools for the future.

Kirsh and Maglio~\cite{kirch-epistemic-action} distinguish between
\textit{pragmatic} and \textit{epistemic} actions. Structured modeling
tools enable users to manipulate domain elements: boundaries,
vertices, textures, layers, light sources, and so on. These pragmatic
actions transform the model. When people sketch, they frequently make
marks that aid thinking but are not intended to be part of the
model. These epistemic actions transform the designer's state of mind
that potentially make it easier to think about how to
proceed. Computer modeling tools tend to be very good at supporting
structured, pragmatic actions. Unfortunately, computer tools are poor
at supporting sketchy, epistemic actions.

Goel's studies found that designers using structured computer tools
came up with fewer and less creative ideas than others who sketched
\cite{goel-sketches-of-thought}. While this study is by now close to
twenty years old, the nature of structured design tools has not
changed (and sketching certainly has not). It is faster and easier to
sketch an idea than it is to produce it in a modeling tool. This rapid
idea generation leads to more ideas, giving a larger pool from which
to draw new ideas. Ultimately sketching promotes better design.

Despite the power of sketching for exploring design ideas, structured
tools are seen as more appropriate for making incremental revisions to
a single idea. Consider the following quote
from~\cite{newman-web-designers}:

\begin{quotation}
\textit{``The beginning of each step I'll do on paper. As soon as I feel
  like I'm going to be starting any design revisions, then I'll move
  to [an electronic tool]... because it's easier to make changes to
  these things.''}
\end{quotation}

This proposal describes an approach that brings sketching and computer
modeling activities together in the same tool. It lets users sketch as
roughly or precisely as they like, iteratively and interactively
making models that can be manufactured using rapid fabrication
machines. Specifically, the system will support users to model objects
for production on laser cutters. Such objects are composed of flat,
rigid parts. Figure \ref{fig:flat} shows example household objects
that could be designed using the proposed system and produced with a
laser cutter.

This project's first motivation is to empower those who are not
professional designers to create meaningful objects with rapid
fabrication machines. It can not be assumed that these users have been
taught to draw as an architect would have been, nor can it be assumed
they have invested a great deal of time to learn the often complicated
modeling software that was developed for professional use. 

The second motivation is to explore the space of sketch-based
interaction as it applies to modeling physical objects. A success
criterion is that the user should never need (or want) to set down
their stylus in favor of keyboard or mouse input.

\begin{figure}
\centering 
\subfloat[] {
  \label{fig:flat-a} 
  \includegraphics[height=2.2in]{img/flat-a.jpg}
}
\hspace{1cm} \subfloat[] {
    \label{fig:flat-b}
    \includegraphics[height=2.2in]{img/flat-b.jpg}
}
\caption{Household objects made with a laser cutter.}
\label{fig:flat}
\end{figure}

\subsection{Conversational Sketch-based Interaction}

The premise of interactive systems is that there is an ongoing
``conversation'' between the user and computer. In sketch-based
systems, users ``speak'' with a stylus; computers ``speak'' with
graphics. Just as humans have social norms for engaging in
conversation that allow us to talk to people we have not met before,
there should be norms for human-computer interaction in calligraphic
systems so our experience from one application can be used in the
next.

Olsen~\cite{olsen-ui-research} argues the dominant paradigm of ``one
display, one keyboard, one mouse'' hinders progress in interactive
systems that use alternate hardware configurations. There is no
standard for sketch-based interaction, where typical hardware is a
pen-sensitive display and stylus. In recent years, researchers have
begun to give more attention to interaction aspects of sketch
recognition-based user interfaces, but the extent of this work remains
limited.

For each of the several technical challenges in sketch-based systems
research there is a human component. Segmentation and recognition are
concerned with perceptual and semantic interpretation, which is
inevitably wrong on occasion. When does the system attempt to
recognize input? How detailed must the recognition be in order to
match the user's intentions? What salient aspects are to be
recognized? How does the system tell the user what has been
recognized? When should this occur? How can the user work with the
system to recover from mistakes?

The answer to these questions presumably begins with ``it depends on
what the user is trying to achieve.'' In the proposed system, ideas
are gradually refined from sketches into manufacturable models. The
user's goals are different early in this process compared with latter
stages. Early in this process the user's goal is to simply record
ideas, possibly to share with other people. The early phase is usually
about \textit{big ideas}. Later, the user's goals might be to add
specific details such as lengths, angles, and how different parts
interact---this phase is about \textit{details}.

\subsection{Proposal Organization}

The remainder of the proposal is organized as follows. First I give a
brief overview of the thesis goals, scope, and methods. Next I present
a motivating scenario of how the system might be used, and relates
many of the sketch-based interaction techniques. Previous related work
on sketch recognition, sketch interaction, and rapid fabrication is
then discussed. Next I detail my own efforts in sketching and design
tools for fast fabrication. The proposal ends with the thesis
contributions, a discussion of how the system will be evaluated, and a
time line for project completion.

\section{Thesis Summary}

This section summarizes the purpose, scope, and methods of my
dissertation.

\subsection{Thesis Statement}

This thesis will present a coherent set of calligraphic interaction
techniques for iteratively making structured drawings that have enough
detail that objects can be fabricated on a laser cutter. Current
computer design tools impose needless, unnatural complexity on
designers. Sketch-based interaction is chosen because of its
widespread use in thinking about and representing design ideas
quickly. If successful, the sketching techniques will enable
relatively inexperienced designers to design and fabricate meaningful
artifacts.

\subsection{Research Scope}

The tool under consideration here supports novice designers in the
fairly narrow domain of design for 2D laser-cutter fabrication. This
domain is chosen because laser-cut parts typically have some aspects
that require precise measurements. While it is necessary to attend to
several domain-specific aspects of the project, my aim is not to
create a commercial-grade design environment.

The research scope of this project is to develop calligraphic
techniques to facilitate the design of these 2D parts. I will evaluate
these techniques in context of the laser-cutting fabrication design
domain.

\subsection{Methods}

This dissertation will center on iteratively creating a calligraphic
design tool. Development will be based on two sources:

\begin{packed_enum}
\item Prior work: This includes the work of others (largely summarized
  in \cite{johnson-sketch-review}) and my own past work.
\item Iterative evaluation: I will regularly test the system with CMU
  students and other volunteers. This means that the plans outlined
  here could change substantially based on evaluation results.
\end{packed_enum}

\section{Motivating Scenario}

\begin{figure}[] 
\centering
\subfloat[Initial sketches of possible designs (preferred idea is circled).] { 
   \label{fig:initial-sketches}
   \includegraphics[width=1.8in]{img/physical-sketches-circled-part.pdf} 
}
\hspace{5mm} \subfloat[Sketch of the user's preferred design in
  greater detail.] {
   \label{fig:final-sketch}
   \includegraphics[width=1.8in]{img/final-sketch.jpg} 
}
\hspace{5mm} \subfloat[Final physical output made with a laser
  cutter.] {
   \label{fig:final-physical-output}
   \includegraphics[width=3.2in]{img/final-physical-output.jpg} 
}
\caption{Initial sketches and photograph of one part of the physical
  output.}
\label{fig:physical-sketches}
\end{figure}

This section illustrates how a user might design the toothbrush holder
shown in Figure~\ref{fig:flat-a} using the proposed system. There is
also a video of this process at the URL
\href{http://vimeo.com/17997357}{http://vimeo.com/17997357}.

This narrative is admittedly contrived, because the user is
`designing' a part that has already been completed. It is included in
order to present several proposed interaction techniques as they might
be used together. The scenario in this section does not address the
important interaction techniques that might lead a user to create
several quick drawings used to think about the range of possibilities
(as shown in Figure~\ref{fig:physical-sketches}).

Some toothbrush holders rest on the sink while others are attached to
the wall in some way; some hold a single toothbrush while others
support up to four; some are similar to cups while others resemble
racks. The user makes a variety of drawings (see
Figure~\ref{fig:initial-sketches}) to help think about particular
needs and preferences, and decides to focus on a rack-style holder,
shown in the lower right corner of the initial sketch. This design
features two slats supported at each end by side pieces. Toothbrushes
fit through holes in the top slat and rest on the solid bottom slat.

So far the user had simply been sketching without the benefit of
computation. But now the user is ready to begin thinking about details
of how the object will be made, and the system can let the user supply
information about dimensions.

There are three unique parts used in this design, shown in
Figure~\ref{fig:final-sketch}. To illustrate the interaction
techniques featured in this system, the remainder of this section
focuses on the design of the part with circular holes in it (shown in
Figure~\ref{fig:final-physical-output}). Figures
\ref{fig:ix-draw-bounds}--\ref{fig:ix-draw-guides} depict the design
process.

The user realizes that toothbrush holders come in many sizes, and it
is not completely apparent what the ``right'' dimensions are. Some
properties like width and depth should be parametric, which allow the
model to produce objects in a variety of sizes. Much of the following
activity is guided by the desire to retain flexibility in the model.

The user begins drawing the part as a rectangle. Even though the user
knows the finished slat will have notches cut from several locations,
it is easy to begin by drawing it as a rectangle and removing the
material for the notches later. The user asks the system to interpret
the drawing, and the system replaces the input with a beautified
rectangle~\cite{pavlidis-beautifier}.

The slat features four notches, each of the same size. To make these
the user begins by drawing just one. A pen stroke that begins and ends
outside a part, but traverses a part boundary will remove material
(see Figure~\ref{fig:ix-remove-from-edge}). This cutting gesture has
been used in earlier systems like Teddy~\cite{igarashi-teddy}.

Next the user would like to add dimension details to the
notch. Because the notch is relatively small on the screen, it is
helpful to zoom in (Figure~\ref{fig:ix-zoom-pan}). Zooming is
performed by a double-circle gesture, borrowing the zooming
technique from Lineogrammer~\cite{zeleznik-lineogrammer}. Clockwise
and counter-clockwise gestures zoom in and out respectively. For a few
moments after zooming, a widget appears that lets the user pan to the
desired location.

The user would like to parameterize the notch dimensions so they may be
changed later by name. To parameterize a length, the user draws a line
with arrows on both ends near the desired line segment, and labels it
with writing (Figure~\ref{fig:ix-notch-param}).

The first notch's geometry should be replicated for the next
three. The user can give the system a
\textit{hint}~\cite{mcdaniel-gamut} by selecting existing
elements. The selection is used as a strong suggestion for the next
editing operations. The user selects the notch by tracing over it. The
selection is graphically acknowledged. Next the three more notches are
drawn at the appropriate places. The \textit{nd} and \textit{nw}
parameters are implicitly copied from the original to the other
notches because they were part of the hint (see
Figure~\ref{fig:ix-hints}).

The user proceeds to give names to the height and width lengths. In
addition the width is given a value, expressed in terms of height (see
Figure~\ref{fig:ix-set-params}). This updates the drawing to reflect
the new width as a function of current height.

Designers often use external tools such as straight edges, French
curves, or stencils when precision is desired. Alternately, people
might lightly draw \textit{construction lines} that guide subsequent
drawing activity~\cite{company-sketching-in-engineering}. The user
would like to draw two circles equidistant from the center of the
slat. To precisely indicate the center of the part, the user draws two
construction lines through the midpoints of two sides
(Figure~\ref{fig:ix-guide-lines}). The designer uses these guides
to place ``fat dots'' in several locations.

The user now wonders if three holes might fit, so two additional
circles are made without using guides. The user decides to stick with
the two-hole design, so the two new holes are erased using a
scribble-out gesture (Figure~\ref{fig:ix-erase}). It is inevitable
that the system's interpretations will occasionally conflict with the
user's intentions. Rather than aiming for perfect recognition (which
is impossible due to ambiguity) this system aims to enable users to
easily recover from such recognition conflicts by providing
appropriate interaction techniques.

The user continues to work, using interaction techniques already
described to parameterize the distance between the center of the part,
and the center of the holes. The user also draws one point where one
of the hole boundaries will be (Figure~\ref{fig:ix-combined}).

The user can hide non-boundary elements like parameters and
construction lines and view the drawing (Figure~\ref{fig:ix-final})
before making the part on a laser cutter. Compare this with the final
output shown earlier in Figure~\ref{fig:final-physical-output}.

\newpage
\newgeometry{left=0.75in,right=0.75in,top=1in,bottom=1in}
\twocolumn

\begin{figure}[] 
\centering
\subfloat[] { 
   \label{fig:ix-draw-bounds-1}
   \includegraphics[width=1.35in]{img/ix-draw-bounds-1.png} 
} \subfloat[] {
   \label{fig:ix-draw-bounds-2}
   \includegraphics[width=1.35in]{img/ix-draw-bounds-2.png} 
}
\caption{System identifies and rectifies closed shapes.}
\label{fig:ix-draw-bounds}
\end{figure}

\begin{figure}[] 
\centering
\subfloat[] { 
   \label{fig:ix-remove-from-edge-1}
   \includegraphics[width=1.35in]{img/ix-remove-from-edge-1.png}
} \subfloat[] {
   \label{fig:ix-remove-from-edge-2}
   \includegraphics[width=1.35in]{img/ix-remove-from-edge-2.png} 
}
\caption{Remove (or add) material with gestures that cross the part
  boundary.}
\label{fig:ix-remove-from-edge}
\end{figure}

\begin{figure}[] 
\centering
\subfloat[] { 
   \label{fig:ix-zoom-pan-1}
   \includegraphics[width=1.35in]{img/ix-zoom-pan-1.png} 
} \subfloat[] {
   \label{fig:ix-zoom-pan-2}
   \includegraphics[width=1.35in]{img/ix-zoom-pan-2.png} 
}\\
 \subfloat[] {
   \label{fig:ix-zoom-pan-3}
   \includegraphics[width=1.35in]{img/ix-zoom-pan-3.png} 
} \subfloat[] {
   \label{fig:ix-zoom-pan-4}
   \includegraphics[width=1.35in]{img/ix-zoom-pan-4.png} 
}
\caption{Double-spiral gesture zooms in (clockwise) or out
  (counter-clockwise)~\cite{zeleznik-lineogrammer}. Pan widget appears
  after zoom gesture.}
\label{fig:ix-zoom-pan}
\end{figure}

\begin{figure}[] 
\centering
\subfloat[] { 
   \label{fig:ix-notch-param-1}
   \includegraphics[width=1.35in]{img/ix-notch-param-1.png} 
} \subfloat[] {
   \label{fig:ix-notch-param-2}
   \includegraphics[width=1.35in]{img/ix-notch-param-2.png} 
}
\caption{Make length parameters with double-headed arrows and text.}
\label{fig:ix-notch-param}
\end{figure}

\begin{figure}[] 
\centering
\subfloat[] { 
   \label{fig:ix-hints-1}
   \includegraphics[width=1.35in]{img/ix-hints-1.png} 
} \subfloat[] {
   \label{fig:ix-hints-2}
   \includegraphics[width=1.35in]{img/ix-hints-2.png} 
}
\caption{Make hints by selecting elements. The hint guides
  interpretation.}
\label{fig:ix-hints}
\end{figure}

\begin{figure}[] 
\centering
\subfloat[] { 
   \label{fig:ix-set-params-1}
   \includegraphics[width=1.35in]{img/ix-set-params-1.png} 
} \subfloat[] {
   \label{fig:ix-set-params-2}
   \includegraphics[width=1.35in]{img/ix-set-params-2.png} 
}
\caption{Set parameter values. Model updates to reflect new
  constraints.}
\label{fig:ix-set-params}
\end{figure}

\begin{figure}[] 
\centering
\subfloat[] { 
   \label{fig:ix-guide-lines}
   \includegraphics[width=1.35in]{img/ix-guide-lines.png} 
} \subfloat[] {
   \label{fig:ix-guide-dots}
   \includegraphics[width=1.35in]{img/ix-guide-dots.png} 
}\\
 \subfloat[] {
   \label{fig:ix-guide-circles}
   \includegraphics[width=1.35in]{img/ix-guide-circles.png} 
}
\caption{Guides facilitate accurate
  drawing. Guidelines~\subref{fig:ix-guide-lines} and reference
  points~\subref{fig:ix-guide-dots} drawn as indicated; circular
  guides~\subref{fig:ix-guide-circles} made by selecting reference
  points.}
\label{fig:ix-draw-guides}
\end{figure}

\begin{figure}[] 
\centering
\subfloat[] { 
   \label{fig:ix-erase-1}
   \includegraphics[width=1.35in]{img/ix-erase-1} 
} \subfloat[] {
   \label{fig:ix-erase-2}
   \includegraphics[width=1.35in]{img/ix-erase-2} 
}
\caption{The user draws two holes but changes their mind and erases
  them with a scribble gesture.}
\label{fig:ix-erase}
\end{figure}

\begin{figure}[] 
\centering
\subfloat[] { 
   \label{fig:ix-combined}
   \includegraphics[width=1.35in]{img/ix-combined.png} 
} \subfloat[] {
   \label{fig:ix-final}
   \includegraphics[width=1.35in]{img/ix-final.png} 
}
\caption{Combine techniques to finish part.}
\label{fig:ix-final-steps}
\end{figure}
\restoregeometry
\newpage
\onecolumn
\doublespacing
\section{Related Work}

While ``the design process'' has important differences from domain to
domain, nearly all designers sketch. The ease of freehand drawing
allows people to quickly externalize ideas to analyze their
consequences, as well as to help see new
solutions~\cite{lawson-designers-think,goldschmidt-dialectics}. Many
design professions formalize the importance of sketching by making it
central to the schooling: architects, industrial and interaction
designers, and mechanical engineers are all explicitly taught to
sketch. And even though sketching is not taught in other design
domains (like software engineering), freehand drawing is still
commonly done. Even people that consider themselves non-designers make
quick drawings to help think through everyday problems like how the
furniture in their house might be arranged. Freehand drawing is done
by people of all ages~\cite{goldschmidt-backtalk} and experience
levels~\cite{suwa-analysis-students}.

The prior literature on technical aspects related to this project can
be described in three categories: sketch recognition, calligraphic
interaction, and rapid fabrication.

\subsection{Sketch Recognition}

Ideally, computers would understand sketches as readily as
people. This is of course a very difficult problem of artificial
intelligence. Therefore, researchers often restrict the problem by
obliging users to draw shapes in prescribed
ways~\cite{rubine-recognizer,wobbrock-dollar}, or by limiting the user
to work in specific domains with fairly small visual vocabularies (on
the order of tens of meaningful primitives). Further improvements are
made if the system can correctly identify contextual clues (often
driven by domain semantics) to prune unlikely
interpretations~\cite{gross-ecn-uist,do-phd-thesis}.

Many sketch recognition approaches include at least two distinct
tasks. The first is primitive ink parsing (also known as
segmentation). This process interprets raw user input (time-ordered
point sequences) into atoms such as straight lines, curves,
intersections, and corners \cite{paulson-paleosketch}. In systems that
recognize multi-stroke elements, this step is also usually responsible
for determining which atoms should be considered together. Segmenters
often leverage temporal
data~\cite{sezgin-masters-early-ink,wolin-smr}, spatial
data~\cite{kara-recognizer-cg}, or both~\cite{cates-phd-thesis}.

The second task is to analyze these atoms to perform
recognition. Recognition accuracy is dependent on the quality of the
primitive ink parsing. To increase robustness and accuracy, some
techniques supply multiple possible segmentations to the
recognizer~\cite{alvarado-dynamic-bayes}.

Design sketches from most domains combine diagrammatic ink and written
language. It is helpful for the system to identify what is writing and
what is not~\cite{shilman-discerning-structure}. If done correctly,
such meta-recognition can ease the recognition task by reducing the
amount of input to interpret.

\subsection{Calligraphic Interaction}

Applications often interpret user input differently depending on which
\textit{mode} the program is in (e.g. line mode, circle mode, text
mode, and so forth). This simplifies the computer's task because it is
unambiguous how to interpret user input. But this complicates the
user's task because they must manage modes while designing. 

One way to make mode changing transparent is to infer the user's
intention. The inferred mode protocol~\cite{saund-inferred-mode}
analyzes user input to determine if user input is unambiguous enough
for action to be taken. When ambiguity exists, the system might
mediate by asking what was intended, or take no action and wait for
additional input~\cite{mankoff-burlap}.

Another way of easing the problem of mode is to make it easier for
users to change between them. This might involve buttons pressed with
the non-dominant hand, or with pen gestures such as dwell or pigtail,
or by using pressure\cite{li-mode-switching}.

Drawn gestures can be powerful, but users must first know of their
existence. Some gestures seem easy enough that people do not
necessarily need to ``learn'' them (e.g. scribble over something to
erase it), but many gestures are hard to remember and might be nearly
impossible to discover without assistance. GestureBar is a novel
approach to let users discover and practice
gestures~\cite{bragdon-gesturebar}.

Last, mode can be managed as it is in traditional applications, where
people use on-screen widgets. These widgets may be present at all
times~\cite{forbus-nusketch-battlespace}, invoked via
gesture~\cite{grossman-hover-widgets,kurtenbach-marking-menus} or by
placing the pen near onscreen
ink~\cite{marinkas-shadowbutton,grossman-handle-flag}.

Design sketches often involve a combination of text and pictures. In
situations where the computer should recognize text (or at least
recognize which ink specifies writing) it is necessary to distinguish
between what is text and what is not. One approach is to automatically
classify input as text or non-text based on a statistical analysis of
its visual properties \cite{patel-detect-text}. Another approach is to
ask the user to specify what is text by performing a gesture. This was
the approach taken by the developers of
Lineogrammer~\cite{zeleznik-lineogrammer}.

The proposed work is similar to Lineogrammer and the ParSketch
system~\cite{company-sketching-in-engineering,naya-parsketch} in
several key respects: they both offer the ability to create precise
drawings by using sketch and gesture recognition, and they both eschew
modal input when possible. The proposed work integrates many more
sketch- and pen-based interaction techniques. Also, because it is in
support of machined output, it is necessary that the drawing give
enough detail that objects can be manufactured.

\subsection{Rapid Fabrication}

In the 1980s, inexpensive but high-quality printers enabled desktop
publishing to become common. Prior to this, people relied on printing
companies and graphic designers if they needed to create
professional-looking printed output. 

% Price information: 
% 2001: 12,900 (ULS 25 watt)
% 2006: 9,995 (ULS 25 watt)
% 2010: 8,500 (ULS 25 watt)
% 2011: 6,850 (ULS 25 watt)
%
%   http://www.rcgroups.com/forums/showthread.php?t=16912 claims that
%   a 25-watt model from Universal Laser systems cost $12,900. Several
%   comments in that thread are in line with that price estimate for
%   home-garage-lab use.
%   http://www.microgeo-usa.com/ProductDetails.asp?ProductCode=universal-laser-VLS2.30
%   currently prices the ULS 25 watt 16x12 unit as costing $6,850.

% http://55-website.com/xo1/ulsinc/english/PDFs/EJ_VL_Article_Reprint.pdf is a press article from 2006 that prices the 25 watt laser at 10,000

Physical prototyping machines might very well become as common as
desktop printers, which will change our perception of the role of
computers~\cite{eisenberg-homespun}. Figure~\ref{fig:prices} shows
prices for a comparable~25-Watt,~16''x12'' laser cutter model from
Universal Laser Systems. These values were found on hobbyist forums on
the Internet. While these data may not be exact, they do show the
price of desktop laser cutting machines has been cut by almost half in
the past ten years. While still out of reach for average hobbyists to
afford, they are inexpensive enough for schools and ``hackerspaces''
to buy for their members.

For designers that do not have access to rapid fabrication machines,
there are quite a few recently founded Internet-based companies that
offer fabrication services for designers. For example,
Pokono\footnote{http://www.pokono.com} lets designers upload cut files
that are analyzed for correctness, and gives people a wide array of
material stock (felt, cardboard, acrylic, leather, wood) and colors.

\begin{figure}[h] %  figure placement: here, top, bottom, or page
   \centering
   \includegraphics[width=4in]{img/prices.png} 
   \caption{Prices of Universal Laser Systems 25-Watt 16x12 inch
     machine in various years.}
   \label{fig:prices}
\end{figure}

This is a democratizing turn of events, but the current crop of
software for designing these objects is typically too hard for most
people to use~\cite{landay-design-tools} and is a limiting factor.

Sketching has long been a subject of study by makers of physical
things. Leonardo da Vinci's legendary sketchbooks are an early example
of the utility of sketching for both thinking and for
specifying. Demonstrations of the earliest computer-based sketching
system, Sketchpad~\cite{sutherland-sketchpad}, were often focused on
drawing machinable parts.

Until fairly recently, many systems for 3D modeling or fabrication
have used the term ``sketching'' as shorthand for ``quick'' in
comparison to traditional CAD modeling
interfaces~\cite{bloomenthal-sketch-n-make,pugh-thesis-viking,zeleznik-sketch}. Google
SketchUp is a commercial example of this.

Many researchers interested in sketch input for designing physical
objects have been concerned with recognizing 3D drawings,
e.g.~\cite{lipson-correlation,masry-3d-sketch}. Such work focuses on
interpreting the 3D geometric meaning of a finished 2D sketch. In a
sense, that approach assumes that the designer's creative work has
concluded and the job of the computer is to translate the drawing into
a 3D model. This contrasts with interactive approaches that aim to
support an ongoing conversation between the human and computer. A
conversational system tacitly acknowledges that the user's ideas are
not fully formed from the onset, but become progressively clearer as
they work.

With the availability of prototyping machines, sketch-based systems
for fabrication are beginning to receive attention. The Furniture
Factory accepts 2D sketch input that is recognized as a 3D
configuration of planes~\cite{oh-fab}. The user adds pieces
incrementally, so there is no need to make a complete sketch all at
once. The system automatically computes how these planes can be joined
together and generates a series of 2D pieces that to be produced with
a laser cutter.

Saul's Sketch Chair \cite{saul-sketch-chair} is another sketch-based
system for generating furniture. It lets users sketch the contours of
a chair's seat and back rest, and using a different drawing mode, add
legs. The system includes a sophisticated physical simulator to let
the designer explore its physicality (for example to determine if it
will remain upright), and if it will be comfortable. It also allows
designers to change subtle properties of curves using onscreen control
handles. My proposed system differs from Sketch Chair in two important
respects. First, Saul's system presumes the user is making a chair or
chair-like object, and provides highly specialized software analysis
tools. To borrow from Henry Ford, you can make anything you want, as
long as it is a chair. My system is more general, so the range of
possibilities with my tool is wider. Second, Sketch Chair (and many
systems like it) constrain designers to using a small set of part
junction types. My system does not propose to include pre-specified
junction types. This is both liberating (because designers are free to
chose whichever type they want) but also encumbering (because they
must design the junctions themselves).

Plushie lets people make bulbous textile objects such as plush toys or
balloons~\cite{mori-plushie}. Users iteratively design objects by
drawing object boundaries or giving editing commands by sketched
gesture. The interaction is based on prior systems in the Teddy family
tree~\cite{igarashi-teddy}.

\section{My Prior Work}

My work is technology-focused but it is driven by observing
people. This section first discusses my experience studying designers'
work practices and the problems they face. I then describe the
software tools I have made.

\subsection{Designer Observations}

I spent a few weeks at MAYA Design in Pittsburgh during the winter
of~2005--2006~\cite{johnson-tiny-ethnography}. In this time I
interviewed designers and their managers and observed the company's
design culture. I am currently in the midst of conducting a series of
interviews with designers who design artifacts for production with
laser cutters. This section relates my findings from both studies.

Informal media accounted for the most visible artifacts of designer
work practice. Sketches, sticky-notes, and whiteboard drawings were in
plain view in nearly all employee areas. Even the kitchen had
whiteboards; when the whiteboards were full I observed people drawing
directly on the windows. Freehand drawing was a highly valued skill
and was central to individual designers' work and group communication.

Four of the five designers interviewed showed me their
sketches. Interestingly, those same designers were less inclined to
show me computational models such as Illustrator files. While
designers did not typically voluntarily show me computational
artifacts, they did show me printed versions of their computational
models. Nearly all of these printed pages had handwriting and sketches
on them, sometimes by more than one person. 

People reported they iterate between virtual and paper versions of
their designs. The designers would edit a computational model with an
application like Photoshop, and then print a paper copy for personal
or group use. They would then draw or write directly on that page as
they explored variations or made refinements. If a paper-based editing
session was useful, the designers would then manually transfer these
changes back into their application. The process of printing, editing,
and manually merging changes was seen a number of times.

More recently I have begun to study designers using laser
cutters. Like at MAYA, iterating between physical and computer-based
artifacts appears to be a dominant process for the laser cutter
users. The current interview study is ongoing. These findings are
preliminary, but they do agree with my past experience at MAYA.

Three interviews have been conducted to date, and several more are
planned. The interviews were conducted in whichever location they were
likely to do this design work. I began by asking each designer to
describe their design process. Each designer was able to show me
sketches or videos of their work, and in two cases could take a recent
project off a nearby shelf. While there are subtle (and some
substantial) differences in their processes, each designer followed
the following pattern.

They begin by thinking about a problem and making drawings. Some
drawings are made to think about how to frame the project (what is it
for), while others help reason about how to make it (how it works, how
it fits together). When the idea is reasonably well-formed they will
make the model in software. 

Following the interview about past design projects, I gave
participants a sketch (Figure~\ref{fig:interview-sketch}) and asked
them the implement it using their software tool of choice. The purpose
of this task was to learn what problems people faced when performing
the common task of using modeling software to actualize an idea based
on a sketch.

\begin{figure}
\centering 
\subfloat[The part users set out to replicate.] {
  \label{fig:interview-sketch-1} 
  \includegraphics[width=0.5\linewidth]{img/laser-me-1.jpg}
}
\hspace{1cm} \subfloat[Drawing of how the part is used in context.] {
    \label{fig:interview-sketch-2}
    \includegraphics[width=0.5\linewidth]{img/laser-me-2.jpg}
}
\caption{Sketches given to participants to implement in modeling software.}
\label{fig:interview-sketch}
\end{figure}

One participant used Rhino, while two used Illustrator. The Rhino user
finished the task in approximately ten minutes while the Illustrator
users finished in about twenty. All users felt there was enough
information in the sketch to make the part, even though there were
some aspects that were not clear, such as the overall width of the
part. 

Rhino directly supports people to use existing geometric elements when
placing new ones. For example, the Rhino user created the top edges
first, and used the top corners as guides to place the notches
directly below. 

Compared with the Rhino user's approach, the Illustrator users
employed a fairly haphazard strategy to implement the sketch. The
first Illustrator user was comfortable with the \textit{Path Finder}
tool, used to combine shapes by intersecting or subtracting existing
paths. The other Illustrator user was aware of this tool but was not
accustomed to using it. Instead, this person would `remove' unwanted
lines by covering them up with opaque white rectangles. The obscured
lines would not be visible, and would therefore not be cut. This did
introduce a good deal of complexity to the task, however. In both
Illustrator cases, the designers would zoom in and use the mouse or
keyboard to move vertices and lines until they appeared to align. In
these cases the program did not assist users to `snap' elements
together.

Both Illustrator participants used the Undo function routinely. The
dominant use of this was to revert after making a single failed
attempt to edit some element. There were a few instances where the
designer believed their current approach was flawed in such a way that
it would be better to back up several dozen operations because a
single decision early on was now observed to be problematic. This is
consistent with the study by Akers \textit{et. al} on the use of
undo~\cite{akers-undo}.

From the single observation it appears that Rhino is better suited to
the task than Illustrator in terms of speed of execution and
efficiency of operations.

After completing the model for the single part, I asked participants
to take whatever steps remain before fabricating the final parts
(roughly sketched in Figure~\ref{fig:interview-sketch-2}). All
participants then proceeded to generate a \textit{cut file}, which
involves arranging copies of the finished model on a sheet to make
efficient use of material. For this step, the Rhino user saved the
model as a PDF and opened it in Illustrator. This user created a red
rectangle equal in size to the stock material that would be placed on
the laser cutter bed. This was done to ensure the laser would not cut
beyond the available material. The other two participants did not
perform this step. All three users arranged the parts by copying the
original, and pasting copies on a new sheet. The copies were then
rotated and moved to make efficient use of material. The Rhino user's
cut file is shown in Figure~\ref{fig:cutfile}.

\begin{figure}[h] 
   \centering
   \includegraphics[width=4in]{img/participant-1-cutfile.pdf} 
   \caption{Final cut file. Black indicates cut lines, while red is
     ignored by the laser cutter and indicates the stock material
     boundaries.}
   \label{fig:cutfile}
\end{figure}

\subsection{Computational Design Tools}

The Designosaur~\cite{oh-fab} is a prototype system for kids to design
their own wooden dinosaur ``skeletons''. They begin by drawing
individual bone outlines and specify notch locations where the bone
joins another. To support this, the user can name bones and use those
names indicate how the bones fit together.

We noticed that editing bone boundaries presented some challenges.
First, dinosaur model bones ares frequently symmetric (e.g. ribs or
vertebrae). The system featured a simple ``mirror mode'' that helped
users make the parts symmetric about the horizontal or vertical axes.
Second, it is useful to be able to change the shape of bones
easily. To tweak the bone curvature without redrawing, we invented a
local editing technique called Flow selection.

Flow selection~\cite{johnson-flow-selection} is a modeless, time-based
selection and editing technique for picking a region of a 2D boundary
and changing its shape. It is especially useful in cases such as the
Designosaur when organic curves are desired. The user sets their
stylus down near a boundary and holds it still as a selection begins
to form. An apt metaphor is that the pen is ``heating'' the region,
which is strongly selected near the pen, and progressively less
selected as distance increases along the boundary path. Then the user
moves the pen to begin reshaping the selected region---the more
strongly selected the ink, the more it moves. The user may lift their
pen up to end the operation, or hold it still once again to smooth the
region. This is helpful in cases where the user input includes
unwanted bumps or jagged lines.

While the Designosaur is based on sketching, the follow-on system is
based on programming. FlatCAD is a system for designing and
manufacturing things on a laser cutter by writing programs in a
language called FlatLang~\cite{johnson-flatcad}.  FlatLang features a
``flying turtle''---a 3D version of the LOGO turtle---which lets the
user compose 3D models of objects built from flat pieces. This helps
users to specify and see how parts will fit together. FlatCAD made it
easy to design everyday items using parameters, so the same model
(FlatLang program) could be used to generate a variety of physical
objects. The objects shown earlier (Figure~\ref{fig:flat}) were made
using FlatCAD.

My experience with the Designosaur sparked an interest in sketch-based
interaction techniques, while FlatCAD kindled an interest in tools for
making precise, machinable models. The Designosaur lacks the ability
to make precise edits: make a notch \textit{here} facing \textit{this
  direction}; make \textit{this} collinear with \textit{that}; and so
on. FlatCAD addressed this issue but introduced roughly the opposite
problem: it is impossible to be non-specific or tentative when writing
a FlatLang program. The current proposal is an effort to bridge the
gap between the positive aspects of sketching (rough, tentative,
fluid) with the positive aspects of FlatCAD-style modeling (precise,
specific, parametric).

To better understand the space of sketch-based design tools, I
co-authored an extensive survey with my thesis
committee~\cite{johnson-sketch-review}. This survey was motivated by
the question: \textit{After forty years of research on computational
  support for sketching, why there are so few real world applications
  of this technology?} In my personal view, a significant reason is
that we lack appropriate sketch-based interaction techniques. The
proposed work seeks to establish a better understanding of what
constitutes good sketch-based interaction. The proposed system is a
vehicle for this research.

\section{Evaluation}

Interaction techniques are challenging to evaluate. There is a
tradition in human-computer interaction to evaluate novel interaction
techniques in a laboratory setting. People perform simple tasks
designed to evaluate the technique in isolation. While this evaluation
approach has its benefits, it avoids the reality that people use
interaction techniques \textit{with other techniques} and \textit{in
  service of a larger purpose}. 

In acknowledgment of this, the proposed system will be evaluated as a
whole, rather than testing isolated pieces. The proposed system offers
a set of techniques that are intended to work harmoniously together to
support users in the realistic goal of designing household objects
using flat material. To test if the system works as intended it will
be compared with existing tools such as Illustrator, SolidWorks,
SketchUp, Rhino, as well as plain pencil and paper. 

The evaluation criteria in this ``whole system'' analysis focus on
what users are able to do (or not do) that would be possible with
traditional structured design software. What is possible to create
using this system? What is \textit{not} possible? What are the common
interaction problems or errors? Do users tend to develop particular
usage patterns that might be exploited? Or do usage patterns indicate
underlying flaws with the system?

Iterative evaluation will be used throughout the development
process. Therefore, the exact list of features (and perhaps evaluation
criteria) must be flexible. Test participants will be asked to design
and fabricate a functional object such as the toothbrush holder
in~Figure\ref{fig:flat-a}. Object choice will depend on the system's
capabilities at the time of testing.

This whole-system approach will allow me to gauge several factors that
compare the proposed system with existing WIMP-based tools:

\begin{packed_enum}
\item Time taken for design and fabrication
\item Number of unique design ideas (as measured
  in~\cite{goel-sketches-of-thought})
\item Overall user satisfaction
\item User satisfaction with individual techniques
\item User satisfaction with certain combinations of techniques
\item How ``learnable'' the interaction style is
\item How system supports (or does not support) alternate working
  styles
\item How well tool helps people \textit{think}, \textit{create},
  \textit{explore}, \textit{visualize}, \textit{specify}, and
  \textit{fabricate}
\item User feedback: criticism and suggestions
\end{packed_enum}

This evaluation rubric does not explicitly address technical factors
traditionally found in work done on sketch based systems. These
factors include recognition accuracy or computational complexity of
various algorithms. Measures such as user satisfaction should account
for such aspects: if the system misinterprets input but provides
adequate methods for recovery, recognition accuracy is not a problem.

\section{Contribution}

Currently there are few interaction techniques for sketch-based
interfaces, and even fewer examples of coherent sets of techniques. As
stated earlier, it is desirable to provide a design system in which
the user can do everything without setting down the stylus.

This thesis will explore the space of calligraphic interaction
techniques to support researchers seeking to make sketch-based design
systems in contexts beyond the laser cutter scenario described above.

This thesis will contribute the following:
\begin{packed_enum}
\item A set of calligraphic interaction techniques. Some will be based
  on existing techniques (e.g. Lineogrammer's zoom), others will be
  novel. Here, \textit{technique} is broadly defined to include
  interactive onscreen widgets, gestural phrases, and user interface
  conventions.
\item An analysis of \textit{the contexts and use cases} when each
  technique is appropriate. For example, rectification might be
  appropriate for latter stage design but distracting during early
  exploratory sketching.
\item An analysis of \textit{which techniques} are complementary.
\item Recommendations for developing further techniques.
\item Software engineering implementation details.
\end{packed_enum}

The best interactive systems have discoverable, consistent interaction
techniques working in concert. Companies such as Microsoft and Apple
have user interaction guidelines for various platforms (Windows,
Windows Mobile, OS X, iOS) that give developers standard advice for
developing such systems. No guide yet exists for sketch-based
interactive systems. This thesis will provide a concrete step to
providing one.

\section{Time line}

In this section I first describe a short schedule for developing the
sketch-based system, followed by a longer-term schedule for completing
the dissertation. 

\subsection{System Development}
\label{sec:system-development-schedule}

I have already written a good deal of code this project, so it should
not take long to develop a rudimentary prototype. Existing code
includes ink analysis and supported data structures (such as Masry's
Angular Distribution Graph~\cite{masry-3d-sketch} among others) and
Ouyang's PCA-based printed character
recognizer~\cite{ouyang-visual-recog}. External tools will be used
whenever possible.

The following table presents a tentative schedule for system development.

\vspace{12pt}
\begin{tabular}{ | l | c | l | }
  \hline

  \textbf{Milestone} & \textbf{Num. Weeks} & \textbf{Topic} \\

  \hline \hline

  Milestone 1 & 1 week & Primitive stroke analysis to generate constraints \\
  
  Milestone 2 & 1 week & Establish boundary model \\

  Milestone 3 & 1 week & Recognize gestures to add/subtract material \\

  Milestone 4 & 1/2 week & Erase unwanted ink with scribble gesture. \\

  Milestone 5 & 1/2 week & Implement zoom and pan widgets. \\ 

  Milestone 6 & 1 week & Indicate and name linear dimensions. \\

  Milestone 7 & 1/2 week & Indicate and name angular and location dimensions. \\

  Milestone 8 & 1/2 week & Indicate guides (points, lines, circles). \\ 

  Milestone 9 & 1 week & Indicate hints and use them for interpretation. \\ 

  Milestone 10 & 1 week & Indicate parameter values (scalar) \\ 

  Milestone 11 & 1 week & Indicate parameter values (expression) \\

  \hline
  
  & 9 weeks total & \\

  \hline

\end{tabular}
\vspace{12pt}

Throughout this process I will conduct informal user studies to
calibrate my work. Additionally, at each step in the process I will
ensure that the system does what it should by using it to make items
with a laser cutter. 

A common heuristic project managers use to predict completion time is
to multiply the predicted time by three---this brings the expected
duration to 27 weeks, or about six months. It is of course my
intention to work more quickly than this.

\subsection{Dissertation Schedule}

The schedule for completing and defending the dissertation
follows. Because the completion time for system building has such a
wide spread, best- and worst-case schedules are given.

\vspace{12pt}
\begin{tabular}{ | l | c | p{80mm} | }
  \hline

  \textbf{Component} & \textbf{Num. Weeks} & \textbf{Remark} \\

  \hline \hline

  System Development & 9--27 weeks & Schedule given above in
  section~\ref{sec:system-development-schedule}. \\

  User Interviews & 2--4 weeks & Understand current problems and
  processes. \\

  System Evaluation & 2--4 weeks & Includes modifications based on
  feedback. Concurrent with system development. \\ 

  Dissertation draft & 4--8 weeks & Borrow from literature
  survey~\cite{johnson-sketch-review} and this proposal. \\

  Dissertation revisions & 3--8 weeks & ---\\

  Defense preparation & 2 weeks & ---\\

  \hline

  \multicolumn{3}{|c|}{
    Summary: 9--27 weeks development, 13--26 weeks other = 
    \textit{22--53 weeks total}
  } \\

  \hline

\end{tabular}

\singlespacing\newpage
\bibliographystyle{plain}

\addcontentsline{toc}{section}{References}
\bibliography{sketch-bibliography}

\end{document}

% how do you accellerate the process by which they become good

% where do they spend the most time?

% the multiple undo problem: backing up 3 minutes, fixing for another
% 3 = 6 minute mistake.

% q1: what's the effective role of sketching?

% q2: how to reconcile the rough nature of sketching and the precise
% demands of laser cutters?

% maybe: start with a physical sketch, scanned, as the basis for
% gradually making it structured.

% paul debevec, image use in modeling

% make a taxoomy of the things people make, trying to create on the
% ponoko blog. make a list of 

% Fred Brooks: (no silver bullet)

% accidental: problems that would be fixed over time

% essential: no matter how much better our software gets, we still
% have to deal with them.


