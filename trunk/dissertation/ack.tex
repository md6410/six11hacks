When I was ten years old, I liked to play {\tt trek} on my dad's CP/M
computer. The goal was to control the Enterprise to blow up
Klingons. To shoot, type an angle: $90$ for up, $270$ for down. But it
was tricky when the bad guy was at a weird angle. One morning my dad
stepped in to show me a thing or two.

He showed me how to think about spatial relationships with circles and
triangles. He explained the calculator's weirder buttons (e.g. {\tt
  arc tan}). If I knew he was subversively teaching math, I wouldn't
have been interested. But his trick worked. Math was no longer just a
chore foisted on me at school. It had a use---\textit{math can be used
  to blow up Klingons}.

When I was twelve, I decided to write a book. My mom patiently read my
(terrible) drafts, but she encouraged me to stick with it to
completion. I kept on with it, and finished after two years. She keeps
a copy of the final version for potential blackmail purposes.

I first want to thank my parents. Dennis Johnson taught me how to be a
nerd. Pat Goehring showed me that encouragement and tenacity were
vital ingredients to finishing epic projects (even if that project is
horrible). Throughout graduate school my parents have been absolutely
amazing. My sister Kelly and I are lucky to have them. 

My high school teachers Tom Lippert, Al Naylor, and Loren Flater were
the first to challenge me as an adult. Steve Shanley and Justin Beahm
were my bandmates: we taught one another collaboration while producing
an amazing noise. Jeff Rusnak suavely convinced me to move to Colorado
after one year at the University of Northern Iowa.

At the University of Colorado, Clayton Lewis was the first to direct
me into research. He introduced me to Gerhard Fischer, who brought me
into his group, the Center for LifeLong Learning and Design (L3D). My
first mentor there was Jonathan Ostwald, who taught me the value of
characterizing and recording failure. Tammy Sumner and Leysia Palen
were always available to talk about my projects and long-term
goals. Without Tammy and Leysia's encouragement I wouldn't have
considered grad school. Leysia introduced me to Mark and Ellen. Eric
Scharff and Rogerio dePaula were L3D graduate students who gave me
their perspectives on research and life in grad school (they could not
scare me off).

\newpage

Casey Jones was an unending supply of encouragement as I applied for
grad school. Her ``Mr. Smarty Pants'' routine convinced me that maybe
I really did have business going back to school.

At Carnegie Mellon, I was fortunate to be in the same cohort as Eric
``Tiller'' Schweikardt, Tony Tang, Sora Key, Yeonjoo Oh, and Shaun
Moon. Susan Finger, Jason Hong, Sara Kiesler and Carolyn Ros\'e gave
much-needed guidance during these early years. Martin Brynskov and
Martin Ludvigsen injected some much-needed Danishness and intellectual
vibes to our lab.

My mentor at Google was Brian Brewington. Beyond nerdy math and
programming tricks, he introduced me to photography.  Jeff Nickerson
was an amazing mentor during my year at Stevens Tech. He can take a
lot of credit for keeping me in the research game. Kumiyo Nakakoji has
an uncanny ability to see things differently and clearly, and to
inspire me to work even harder. Karolina Glowacka was an amazing
friend who was able to explain advanced AI and statistics topics while
helping me to decimate the beer supply. She was an excellent sounding
board and bogon-filter for my many off-the-wall ideas.

The most recent set of Codelabbers are a vibrant community of weirdos
and brainiacs. My thesis would not have happened without these people
around: Zack Jacobson-Weaver, Deren Guler, Madeline Gannon, Hyunjoo
Oh, Andrew Viny, Cheng Xu, Huaishu Peng, Kuan-Ju Wu, Rita Shewbridge,
and Hiro Yoshida. In particular, Tobias Sonne (another Dane!),
deserves credit for sparking the Codelab community, even though he was
only around for one semester. Just as I was leaving, Nick Durrant and
Gill Wildman entered the scene and gave me a final boost.

Very special thanks goes out to my committee. Mark and Ellen brought
me to CMU and provided a great environment. Even after she went to
Georgia Tech, Ellen was always available to collaborate or provide a
much-needed jolt of reality. For the past two years, Jason has stepped
up to play a central role in my PhD, by always being available to
talk, and by bringing me into his weekly student meetings. Mark has
been more than a thesis advisor: he changed the way I view the
world. I'm pretty sure that is a good thing, but the jury is still
out.

Of course I could not finish this without thanking Sputnik, the
coolest Puerto Rican Beach Mutt in the history of the world. He's been
with me throughout the entire development of my thesis system and has
kept me sane (mostly) throughout. In fact, he's curled up next to me
as I type this. Good boy.
