\chapter{Related Work}

This chapter discusses two high level topics. First I discuss design,
paying close attention to rapid fabrication and laser cutting. Second
I provide background on research pertaining to sketch based interaction.

\section{Design for Rapid Fabrication}

A growing community of self-described \textit{makers} design and build
many kinds of physical things~\cite{gershenfeld-fab}. Some are
electronic or robotic gizmos, while others are made from traditional
material. These ``new makers''~\cite{gross-new-makers} are empowered
by rapid fabrication machines like 3D printers and laser cutters.

It is possible that we are beginning to see a shift from an economy
based on mass-production (in factories) to one that includes
mass-customization (in homes, schools, and community
centers)~\cite{economist-fab}. Rapid fabrication machines continue to
decline in price while improving in quality. A new sector of
businesses use rapid fabrication to cater to the needs of hobbyist
designers as well as people that need highly customized
goods~\cite{paulos-citizenscience}. For example, companies such as
Ponoko fabricate and send users physical output based on digital
models uploaded over the web.

Rapid prototying machines are used in many domains: mechanical
engineering, architecture, craftwork, industrial design, to name a
few. While many users of these machines are professional, a growing
population of Makers are not necessarily educated to design and build
things. Instead, members of this Maker group might be better described
as hobbyists or semi-professionals. They are smart and motivated, but
do not necessarily have a great abundance of time or money.

\subsection{Rapid Fabrication Machines}

The software modelling tool discussed later in this thesis targets
laser cutters, which are just one particular type of rapid fabrication
machine. I will first describe the broader area of rapid fabrication
in order to situate the specific machine in question.

There are several related (but not necessarily interchangeable) names
for machinery to in this space. \textit{Numerical Control (NC)}
machines were introduced in the 1950s that ran programs encoded on
punched tape. \textit{Computer Numerical Control (CNC)} introduces a
computer, capable of performing conditional execution soon
followed. Modern CNC machines are found anywhere from large automobile
factories (e.g. as sophisticated robots) to the homebrewed 3D printer
set up in the hobbyist's garage.

When applied to CNC machines, the terms ``rapid fabrication'' and
``rapid prototyping'' refer to their role in a design process. While
industrial CNC robots are engaged in large-scale production work,
rapid fabrication machines are used primarily by designers to explore
ideas and construction (preceding large-scale production, if it
happens at all). Because of their role in the design process it is
critical that these devices are faster and cheaper than using human
labor with manual machinery.

The type, quality, and composition of the output depends not only on
the machine, but also on the designer's skill in creating suitable
instructions. A CNC mill, for example, could be used to create a
customized office desk. But in order to build the parts comprising the
desk, somebody must give the machine a digital model that indicates
not only where the mill will cut, but how the cut will be made.

A mill may 3 or more axes: it can move on the x/y plane, up and down
(z-axis), and some models allow the tool head to rotate. The tool path
is therefore an important consideration when designing for most types
of computer controlled manufacturing machines. In many (but not all)
cases, software tools can compute tool paths automatically. Designers
must consider a machine's capabilities.

SIMI targets design for laser cutters. This tool was chosen for
several reasons. First, tool path concerns are nearly
non-existant. Second, a laser cutter executes quickly, giving me a
faster loop of coding, testing, and debugging. Cutting the parts for a
toothbrush holder, for example, takes about five minutes on the
machine in our lab, but our 3D printer would take several hours to
make a similar object. Last, Laser cutters are among the more popular
rapid fabrication machines.

They can be thought of as a very fast, strong, and precise
automated razor blade, cutting through flat material (paper, wood,
plastic, metal, etc.) from directly above. Many items can be made
entirely with a laser cutter, aside from the occasional screw or
glue. Figure~\ref{fig:laser-example} shows several examples of useful
items made with laser cutters. % TODO get good examples

The user places material on the laser cutter's \textit{cut
  bed}. Typical sizes for cut beds are in the range of 12''x 18'' to
48'' x 72'' (30.5 x 46 cm to 122 x 183 cm). A laser is directed
through a series of mirrors that are mounted on robotic arms that move
to allow the laser to reach any location on the cut bed. A 40 watt
machine can cut $\frac{1}{4}''$ (6mm) thick wood; a 100 watt machine
can cut up to 1'' (25mm) plywood.

\subsection{Current Design Tools for Laser Cutting}

Today, designers can choose among several modeling tools for laser
cutter projects. Adobe Illustrator, a general-purpose vector graphics
editor, is the most commonly used tool. Illustrator is full-featured
and has an interface new users find familiar. However, participants in
our formative study (see Section~\ref{sec:formative-interviews}) had a
hard time using Illustrator quickly and effectively because they spent
a great deal of effort looking for appropriate functions among the
many features that are irrelevant to laser cutters. Specialized CAD
tools like Rhino or SolidWorks are perhaps more appropriate for this
kind of modeling but they also have a substantial learning curve. If
rapid fabrication is to become common, appropriate modeling tools must
be made accessible to ordinary users~\cite{lipson-homefactory}.

\section{Design Sketching}

People commonly sketch when problem solving. Some sketches are
personal, others collaborative. Some help people make quick
calculations and are quickly forgotten while others serve longer-term
purposes. For professional designers, sketching is a \textit{process}
to think about problems. Just as importantly, a sketch (the result of
the process) is a \textit{record} that communicates ideas. 

Design can be seen as an iterative process of problem-framing and
exploring possible solutions within the current conception of the
problem. A sketch is not a contract: it is a proposal that can be
modified, erased, built upon. The rough look of hand-made sketches
suggests their provisional nature.

Some theories of cognition give the human mind two distinct tasks: to
perceive the world via our senses, and to reason about what our senses
provide. In contrast, the late psychologist Rudolf Arnheim argues that
perception and thinking are inseparable: ``Unless the stuff of the
senses remains present the mind has nothing to think
with''~\cite{arnheim-visthink}. Visual thinking is valuable in
evaluating what is and designing what might be. Sketching allows
people to give form to notions that are otherwise imaginary; the act
of seeing fuels the process of reasoning.

Sketching plays a crucial role in the practice of design. It helps
people think about problems and offers an inexpensive but effective
way to communicate ideas to others. The practice of sketching is
nearly ubiquitous: One recent study of interaction designers and HCI
practitioners found that 97\% of those surveyed began projects by
sketching~\cite{myers-behavior-design}. We must understand the purpose
and practice of sketching as it is done \textit{without} computation
if we hope to effectively support it \textit{with} computation.

% Karolina's project management and Shaun's floorplan diagram.
\begin{figure}
  \centering
  \begin{subfigure}[b]{0.42\textwidth}
    \includegraphics[width=\linewidth]{img/sketch-type-project-management.pdf}
    \caption{Project management diagram showing task precedence of two
      projects. Hastily drawn boxes and arrows represent abstract
      activities.}
    \label{fig:sketch-type-pm}
  \end{subfigure}
  \hspace{1cm}
  \begin{subfigure}[b]{0.42\textwidth}
    \includegraphics[width=\linewidth]{img/sketch-type-architecture.pdf}
    \caption{An architect's floor plan sketch. It includes text, 
      spatial information, and symbols representing household
      items like a piano or dining table.}
    \label{fig:sketch-type-architecture}
  \end{subfigure}
  
  \caption[Project Management and Architecture Sketches]{Sketches vary
    in domain and in the visual characteristics of marks.}
  \label{fig:types-of-sketches}

\end{figure}


While designing, we iteratively explore and refine the problem
definition and proposed solutions. Sketching supports this creative
search process. We set out on our design task with some high-level
goals. However, due to the
ill-structured~\cite{simon-ill-structured-problems} and ``wicked''
nature of design~\cite{rittel-wicked}, we encounter unforeseen
opportunities and constraints as designing progresses. Those
opportunities and constraints may be implicit in the original problem
description, but designers expose them as they explore. The
discoveries are incorporated into the understanding of the problem and
potential solutions. Design problems are ``not the sort of problems or
puzzles that provide all the necessary and sufficient information for
their solution~\cite{cross-nature-nurture}.'' So it goes with
sketching. We draw different views of our model, which allows us to
perceive the problem in new ways.

Designers engage in a sort of ``conversation'' with their sketches in
a tight cycle of drawing, understanding, and
interpreting~\cite{schon-kinds-of-seeing}. Goldschmidt describes this
as switching between two reasoning modes: ``seeing that'' and ``seeing
as''~\cite{goldschmidt-dialectics}. Seeing \textit{that} is the
process of recognizing the literal, descriptive properties of the
drawing.  Seeing \textit{as} is figurative and transformative,
allowing the designer to re-interpret parts of the sketch in different
ways. Care must be taken to support this conversation when developing
sketch based modeling tools. If the system interprets drawings too
aggressively or at the wrong time, it may prevent the human designer
from seeing alternative meanings; recognize too little and the
software is no better than paper.

\subsection{Prototyping and fidelity}
\label{sec:traditional-prototyping}

Newman and Landay's ethnographies of web designers focused how
designers use informal techniques~\cite{newman-web-designers}. They
found that designers always sketch at the beginning of a web design
project, exploring numerous high-level options. Frequently this early
sketching phase is accompanied with construction of low-fidelity
prototypes made on paper or with Microsoft PowerPoint. As the design
progresses and designers begin incrementally adding details, they move
to higher fidelity models. Client meetings are an important forcing
function in web design projects. When meeting with clients, designers
want to show polished prototypes produced with computer
software. Therefore designers used electronic tools earlier in the
process than they would otherwise have preferred.

Today most software tools support incremental refinement and
specification of details but do not adequately support idea generation
or exploration~\cite{terry-creative-ui}. Designers who begin using
software tools in the early phases of design tend to make superficial
explorations of possible solutions. Further, because tools are poor
for exploration but good for specifying details (font, line weight,
and color), designers tend to focus on nuances that are not yet
important. Observing that current tools are inadequate for creative
pursuits, researchers have developed calligraphic tools such as SILK
and DENIM, which aim to support the early phases of
design~\cite{landay-silk,lin-denim}.

Paper sketches dominate the early phases of design as people generate
new ideas, in a process Goel terms ``lateral
transformations''~\cite{goel-sketches-of-thought}. But as soon as the
web designer believes he or she will make incremental revisions (which
Goel calls ``vertical transformations'') they switch to a computer
tool.

\section{Sketches as a symbol system}

% Cartoon clouds or trees: Overloaded, Ambiguous
\begin{figure}
  \centering
  \begin{subfigure}[b]{0.3\textwidth}
    \centering
    \includegraphics[width=\textwidth]{img/cloud-1.pdf}  
    \caption{Overloaded semantics: The cloud and tree have similar
             shapes but different meanings due to context.}
    \label{fig:cloud-1} 
  \end{subfigure}
  \hspace{0.03\linewidth}
  \begin{subfigure}[b]{0.3\textwidth}
    \centering
    \includegraphics[width=\textwidth]{img/cloud-2.pdf}  
    \caption{Ambiguity: A small addition changes our
      interpretation. The object at left may be a cloud or a thought
      bubble.}
    \label{fig:cloud-2} 
  \end{subfigure}
  \hspace{0.03\linewidth}
  \begin{subfigure}[b]{0.3\textwidth}
    \centering
    \includegraphics[width=\textwidth]{img/cloud-3.pdf}  
    \caption{More information gives more confidence about object
      identity. Text in the cloud indicates a thought bubble.}
    \label{fig:cloud-3} 
  \end{subfigure}

  \caption[Overloaded semantics and ambiguity]{Overloaded semantics
    and ambiguity.}
  \label{fig:cloud}
\end{figure}


Goodman provides a comprehensive framework for analyzing the
properties of various symbol systems, including
sketches~\cite{goodman-symbols}. Goel places sketching in Goodman's
framework, noting that sketches have \textit{overloaded semantics},
they are \textit{ambiguous}, \textit{dense}, and
\textit{replete}~\cite{goel-sketches-of-thought}. These properties
describe one particular sense of sketching in which the drawer's marks
may be idiosyncratic. It is critical to understand these properties
when developing sketch-based design software.

Sketches have ``overloaded semantics'': The same symbol may mean
different things depending on context. Further, a sketched symbol may
be ``ambiguous'', meaning that the symbol affords more than one
plausible interpretation.  Figure~\ref{fig:cloud} illustrates these
properties. A lumpy shape can be used to indicate many things
including clouds, trees, or thought bubbles. We interpret the shape
differently depending on context.

Sketched symbols are ``dense'', indicating there is a continuous range
between instances of the same symbol. While there may be minute visual
discrepancies between symbol instances, Goel claims that such symbols
are also ``replete'': no aspect of the sketched symbol may be safely
ignored (Figure~\ref{fig:dense-replete}).

The pen strokes constituting a sketch serve various functions. Ink may
indicate abstract domain symbols (e.g. diode, treble clef), object
boundaries, actions (e.g. arrows indicating containment or movement),
dimensions and units, annotations, region texturing, and so on. Some
parts of a sketch are more dense and replete than others. For example
a diode's properties do not change if it is drawn with a slightly
larger triangle. However, subtle variations in how a desk lamp is
drawn might lead to substantially different aesthetic responses to it.

% Stick figures: dense and replete
\begin{figure}
\begin{center}
  \includegraphics[angle=0, origin=c]{img/dense-replete-stick-figures.pdf}

  \caption[Dense and replete sketches]{Different instances of the same
    stick figure vary along a continuum (\textit{dense}). However, the
    visual properties of individual symbols may (or may not)
    communicate additional information (\textit{replete}). Is the
    figure at the right waving? }

  \label{fig:dense-replete}
\end{center}
\end{figure}


Gross and Do discuss some properties of hand-drawn diagrams from the
perspective of building tools to support design drawing
activities~\cite{gross-ecn-uist}. The authors distinguish sketches
from diagrams, noting that diagrams are ``composed of primitive
elements chosen from a small universe of simple symbols---boxes,
circles, blobs, lines, arrows.'' This list is certainly not
exhaustive, but it does illustrate the general idea that diagrams have
a limited vocabulary. In practice, sketches and diagrams from various
dialects may be combined (e.g. mathematical notation on the same page
as circuit diagrams and hand written notes.)

Freehand diagrammatic drawings are abstract, ambiguous, and
imprecise. \textit{Abstract} symbols denote elements whose identities
or properties are not (yet) important or known. For example,
Figure~\ref{fig:sketch-type-pm} on page \pageref{fig:sketch-type-pm}
shows a project management diagram of two hypothetical projects. The
activities composing each project are abstract---they could represent
anything. The value of the sketch is that it shows the project's
network topology and does not draw attention to what the specific
activities are. 

An \textit{ambiguous} symbol has many plausible interpretations. The
floor plan sketch in Figure~\ref{fig:sketch-type-architecture} shows
several rectangles indicating rooms, furniture, shelves, or
counters. Human observers can confidently disambiguate the intended
meaning of some rectangles, but others remain unclear. The
bottom-right of the sketch shows two armchairs and a sofa with an
ambiguous rectangle in the middle that could plausibly represent
either a rug or a coffee table.

Last, freehand diagrams are \textit{imprecise}. Imprecision allows
designers to work with rough values (e.g. ``about two meters wide'')
and avoid premature commitment. Imprecision also indicates that the
design is by no means final.

The notational properties of sketches make them powerful tools for
supporting visual thinking. Designers may leverage ambiguities in
their sketches to see new meanings, for example. However, these same
properties present challenges for accurate software recognition.

The degree to which a drawing is ambiguous, imprecise, and abstract
varies among instances, and people might interpret them differently. A
rough sketch is useful to designers, especially for brainstorming and
incremental development of ideas. But in order for the sketch to be
transformed into a finished product (e.g. as a digital model supplied
to a rapid manufacturing device), it must be made unequivocal,
precise, and concrete. The process of moving from the informal sketch
to the formal specification involves drawings that are semi-ambiguous,
partially precise, and with some abstractions given definite
identities.

\subsection{Summary: traditional sketching and computation}
\label{ref:traditional-summary}

If we hope to effectively support sketching with computation, we must
first understand the practical aspects of traditional sketching. 

Sketching is an important---perhaps necessary---tool for doing
design. It gives us a way to quickly make provisional drawings, which
help us efficiently make sense of spatial, relational
information. Sketches let us make marks that are as vague or specific
as we need. Because sketched elements can easily be ambiguous, rough
drawings afford different interpretations. We may therefore reflect on
our sketches and see new meaning in existing marks. People sketch in
part because they don't know exactly what they are making---sketching
facilitates exploration.

Low-fidelity prototypes are especially important as tools to test
ideas during early design. This is because they are easy to make,
allowing designers to quickly expose problems before committing to
decisions. Sketching is a common method of creating such prototypes.

In order for computers to recognize sketches, we must develop
techniques to transform imprecisely made marks into discrete
symbols. However, some of the properties that make a sketch useful for
a human (overloaded semantics, ambiguous, imprecise, etc.) complicate
the task of computer recognition.

%% Be sure to also talking about pragmatic vs. epistemic actions, and
%% the tentative, explorational nature of design sketching.

\section{Computational Support for Sketching}

\subsection{Sketching Hardware: Tablets and Pens}

Researchers in sketch recognition and interaction typically use tablet
devices such as a Wacom Bamboo or Cintiq. Some, like the Bamboo,
simply sense stylus input but do not display feedback. These ``blind''
tablets require the user to look at one place (their computer display)
but draw on another surface. This can lead to hand-eye coordination
trouble. The Cintiq, and various Tablet PC computers, combine display
and sensing surfaces. In this case, the user's pen comes into contact
with a drawing plane that is separated from the display plane by only
a few millimeters. When the user views the display at an angle, this
parallax difference can be annoying, but it is certainly better than
the situation with blind tablets.

Input surfaces that are intended to be used with fingers or hands
offer different interaction experiences than pen-oriented drawing
surfaces. For example, users may trace shapes with a single finger,
use two fingers to zoom in or out, or use whole-hand gestures for
issuing other commands. These interaction techniques provide
opportunities for developing innovative sketching applications, as
demonstrated by work by Hinckley \textit{et. al} at Microsoft
Research~\cite{hinckley-pen-touch}.

Regardless of sensing technology, these devices allow users to provide
input in a way that is much closer to handwriting than a mouse
allows. Although pen and mouse input share many properties (both allow
users to interact with 2D displays) they have several key differences.

Mouse input affords \textit{motion} sensing while pen input
affords \textit{position} sensing~\cite{hinckley-input-technology}. In
other words, while mice produce the relative \textit{change} in
$(x,y)$ locations, pens directly provide absolute $(x,y)$ locations.
Users can configure tablets to report relative position, thereby
behaving like a mouse.

Form is also extremely important. A stylus affords people to use the
fine motor abilities of their fingers to control the tip of the pen,
whereas hand and forearm muscles dominate mouse usage. Fingers can be
used to move the mouse, but not with the same dexterity possible with
a pen. Depending on the type of work, a pen may be ergonomically
superior to a mouse, or the other way around.

\begin{figure}
  \centering 
  \begin{subfigure}[b]{6.0cm}
    \centering
    \includegraphics[origin=c, width=\textwidth]{img/button-force-mouse.pdf} 
    \caption{Force vector for mouse button press is perpendicular to
      the plane the device rests on. It will therefore not move much.}
    \label{fig:button-force-mouse} 
  \end{subfigure}
  \hspace{0.5cm} 
  \begin{subfigure}[b]{6.0cm}
    \centering
    \includegraphics[origin=c, width=\textwidth]{img/button-force-pen.pdf} 
    \caption{Pressing a stylus button is more likely to cause unwanted
      pen tip movement because it is at an acute angle to the drawing
      surface plane.}
    \label{fig:button-force-pen} 
  \end{subfigure}
  \caption[Pen vs. Mouse]{The force required to press a mouse button compared with a
    stylus button.}
  \label{fig:button-force}
\end{figure}


Some styluses have buttons. While buttons are an indispensable part of
a mouse, they are often difficult to use on the barrel of a
pen~\cite{plimmer-pen-usability}. The force of a \textit{mouse} button
click is orthogonal to the device's plane of use and has negligible
effect on target accuracy. However, pressing buttons on a
\textit{stylus} can move the tip of the pen, making it difficult to
press the button while pointing at particular objects (see
Figure~\ref{fig:button-force}). Further, pressing a button on a
computer stylus usually requires the user to adjust the pen in
hand. This action may be distracting and uncomfortable for long term
use.


%% Distinguish between ``faithful'' and ``faux'' sketching
%% systems. Faithful tools respect the tentativeness of physical
%% sketching where people don't have all the answers at the start. Faux
%% sketching systems are named so because the developers view their tools
%% as being fast, but they do not support sketching as on paper.

%% Within computational support for sketching there are a bunch of
%% technical areas to look at. Can crib a lot from the lit review. The
%% sections that are most important are:

%% \begin{itemize}
%% \item Ink Parsing (corner finding, segmentation)
%% \item Recognition
%% \item Domain modeling
%% \item Graphics (rectification, rendering)
%% \item Interaction techniques, interaction design
%% \end{itemize}

