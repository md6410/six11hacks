Designers typically sketch during concept development, and use
computer modeling software to finalize their ideas. Both
tools---pencil and computer---are well-suited for their
task. Sketching is fast, fluid, and lets people explore ideas
efficiently; computation lets users make structured, detailed digital
models. But in practice, design does not progress directly from idea
to final product. Instead, designers use paper, then software, return
to sketching to develop or change concepts, turn back to the computer
to implement changes, and so on. The transition between between
pencils and pixels is time-consuming.

Rapid fabrication and prototyping is a design area of increasing
importance. Machines like laser cutters and 3D printers are becoming
more common, and ordinary people are increasingly interested in these
machines to design and make things. Current design software is made
for professionals, not hobbyists. For avocational rapid prototyping
machine users, software is the bottleneck.

This thesis presents \textit{Sketch It, Make It (SIMI)}, a
sketch-based modeling tool that lets non-experts design precise items
for laser cutting by sketching. This removes the need to transition
between sketches and formal CAD models. SIMI users make line work,
issue commands, create geometric constraints, and produce ``cut
files'' for production on a laser cutter---entirely with a
stylus. There are no modes (like line mode, erase mode) in SIMI: the
meaning of user input is recognized by analyzing pen strokes and
context.

Researchers have long sought to infer user intention by looking at
sketches. The work presented in this thesis treats sketch recognition
largely as an interaction design problem, rather than an artificial
intelligence problem. Sketch-based techniques were developed to
provide efficient user experience in specific contexts, including the
laser cutter domain and the other sketch-based interaction techniques
found in SIMI. 

Two evaluations were used to measure SIMI's performance. First, a
workshop involving sixty undergraduate architecture students was
carried out. The students used SIMI and provided feedback on its
technical performance and their own attitude about the software. Next,
a task/tool analysis of SIMI and another common tool compares the
steps needed to perform a simple design task.
