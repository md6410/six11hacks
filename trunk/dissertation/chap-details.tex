\chapter{Sketch It, Make It: Details}

This is the details chapter. It describes the recognition strategy
involving ink parsing, and how it recognizes various things like
glyphs (rotation invariant pictograms) or gestures (grammar-based like
erase).

This will be a dry chapter but it should be the most useful to anybody
else who wants to work on this. Should be enough that a smart grad
student can implement things based on just what is here.

\section{Ink Parsing}

There are three stages of ink parsing: (1) processing while the pen is
down, (2) immediately when the pen is lifted, and (3) later when the
magic button is pressed or timeout is reached.

In stage 1, only two things are going one: the system records and
draws pen activity, and ``pen down'' recognizers are
invoked. Currently the only ``pen down'' recognizer is for erasure. It
has to be very efficient to avoid introducing lag.

The other two stages are more complex. On pen up:

\begin{itemize}
\item Find corners/segments, and associate that structured ink with raw ink. 
\item Apply recognizers to raw ink
\item Disambiguate results of raw ink recognizers
\item For all recognized items, take the appropriate action
  (e.g. latch) and remove related raw ink.
\item For remaining raw ink, add the raw ink to an 'unprocessed' list
  for later analysis.
\item If any actions were taken based on recognition, request a
  snapshot.
\end{itemize}

Stage 3 operates on the 'unprocessed' list made from stage 2. So
anything that was recognized in stage 2 does not make it this far.

\begin{itemize}
\item Get structured ink related to unprocessed ink.
\item Remove hooks
\item Add all new structured ink to the model
\item Apply post-hoc recognizers to new structured ink
\item Disambiguate any conflicting results using (a) context in the
  model and (b) preset rules like right-angle wins over same-length
  (filterRecognizedItems())
\item Apply remaining recognized items. This removes the related
  structured ink from the model.
\item Auto-latch remaining segments with each other and existing model
  segments.
\item Search for cutouts (don't call them stencils anymore)
\item Wake up constraint solver
\item Request state snapshot
\end{itemize}

\begin{itemize}
\item 
\end{itemize}

\begin{itemize}
\item 
\end{itemize}

\begin{itemize}
\item 
\end{itemize}

\section{Sketch Interaction Techniques}

\subsection{Latching}

Four kinds: automatic, endpoint, continuation, and t-junction.

\subsection{Erase}

Wrote this somewhere else: For example, the Erase gesture went through
a number of design iterations. Each version was only slightly better
than the last, and users were only able to successfully execute the
scribble gesture about half the time. When it failed, the scribble
would be interpreted as linework, which would then require users to
erase that. I then completely changed the erase gesture recognizer so
it would operate as the pen was down. When it found a scribble, a
colored 'X' appeared over the scribble to indicate that the user's
erasure will succeed. This lets users scribble until an X appears,
giving them confidence the system has understood. This reduced the
number of recognition errors substantially.


\subsection{Same length}

\subsection{Specific length}

\subsection{Right angle}

\subsection{Same angle}

\subsection{Flow selection}

\subsection{Undo and Redo}

\subsection{Camera Control: Pan and Zoom}

\section{Model and Constraint Solver}

Data structures are simple: we have points and low-level
constraints. List all the low-level constraints and give an example of
the anatomy of one, especially how it measures error and how it
`knows' how to offer advice to correct that error.

Next talk about the higher-level user constraints: a user constraint
for right angle is implemented with several low-level constraints. 

Now for the iterative, numeric solving algorithm. While explaining how
it works, can justify the randomness by showing graphs of total error
when the solver did not use much randomness.

Explain how the solver relates to the rest of the tool: in a separate
thread, sending out signals when it is done or after taking a step (to
redraw), and how it is capable of being paused when the user is doing
something. This makes the system responsive at all times.
