\chapter{Sketch It, Make It: Details}

The previous chapter gave an overview of SIMI's architecture. This
included an introduction of SIMI's recognition process.
(Section~\ref{sec:recognition-architecture}). This chapter gives
details on how each recognizer works. 

First, SIMI's corner finding and segmentation strategy is
described. This process is necessary to most recognition, and is what
produces geometric output like lines and arcs. Next, the three types
of recognizers are described: including Dynamic, Pen Up, and Delayed
recognizers. All sketch based interaction techniques are detailed in
these sections.

\section{Corner Finder and Segmentation}
\label{sec:corner-finder}




\begin{itemize}
\item Remove hooks
\item Get structured ink related to unprocessed ink.
\item Add all new structured ink to the model
\item Apply post-hoc recognizers to new structured ink
\item Disambiguate any conflicting results using (a) context in the
  model and (b) preset rules like right-angle wins over same-length
  (filterRecognizedItems())
\item Apply remaining recognized items. This removes the related
  structured ink from the model.
\item Auto-latch remaining segments with each other and existing model
  segments.
\item Search for cutouts (don't call them stencils anymore)
\item Wake up constraint solver
\item Request state snapshot
\end{itemize}

\section{Dynamic Recognizers}

\subsection{Erase}

What is it for? State obvious use quickly, and give detail on any
non-obvious uses.

Recognition process: What kind of data does it use? What is the
context-free recognition like? What context does it use?

What (if any) visual feedback is there?

What actions are taken if it is positively recognized and not
filtered?

Wrote this somewhere else: For example, the Erase gesture went through
a number of design iterations. Each version was only slightly better
than the last, and users were only able to successfully execute the
scribble gesture about half the time. When it failed, the scribble
would be interpreted as linework, which would then require users to
erase that. I then completely changed the erase gesture recognizer so
it would operate as the pen was down. When it found a scribble, a
colored 'X' appeared over the scribble to indicate that the user's
erasure will succeed. This lets users scribble until an X appears,
giving them confidence the system has understood. This reduced the
number of recognition errors substantially.

\subsection{Undo and Redo}

What is it for? State obvious use quickly, and give detail on any
non-obvious uses.

Recognition process: What kind of data does it use? What is the
context-free recognition like? What context does it use?

What (if any) visual feedback is there?

What actions are taken if it is positively recognized and not
filtered?

\subsection{Flow selection}

What is it for? State obvious use quickly, and give detail on any
non-obvious uses.

Recognition process: What kind of data does it use? What is the
context-free recognition like? What context does it use?

What (if any) visual feedback is there?

What actions are taken if it is positively recognized and not
filtered?

\section{Pen Up Recognizers}

\subsection{Latching}

What is it for? State obvious use quickly, and give detail on any
non-obvious uses.

Recognition process: What kind of data does it use? What is the
context-free recognition like? What context does it use?

What (if any) visual feedback is there?

What actions are taken if it is positively recognized and not
filtered?

Four kinds: automatic, endpoint, continuation, and t-junction.

\subsection{Pan and Zoom}

What is it for? State obvious use quickly, and give detail on any
non-obvious uses.

Recognition process: What kind of data does it use? What is the
context-free recognition like? What context does it use?

What (if any) visual feedback is there?

What actions are taken if it is positively recognized and not
filtered?

\subsection{Select point}

What is it for? State obvious use quickly, and give detail on any
non-obvious uses.

Recognition process: What kind of data does it use? What is the
context-free recognition like? What context does it use?

What (if any) visual feedback is there?

What actions are taken if it is positively recognized and not
filtered?

\section{Delayed Recognizers}

\subsection{Same length}

What is it for? State obvious use quickly, and give detail on any
non-obvious uses.

Recognition process: What kind of data does it use? What is the
context-free recognition like? What context does it use?

What (if any) visual feedback is there?

What actions are taken if it is positively recognized and not
filtered?

\subsection{Specific length}

What is it for? State obvious use quickly, and give detail on any
non-obvious uses.

Recognition process: What kind of data does it use? What is the
context-free recognition like? What context does it use?

What (if any) visual feedback is there?

What actions are taken if it is positively recognized and not
filtered?

\subsection{Right angle}

What is it for? State obvious use quickly, and give detail on any
non-obvious uses.

Recognition process: What kind of data does it use? What is the
context-free recognition like? What context does it use?

What (if any) visual feedback is there?

What actions are taken if it is positively recognized and not
filtered?

\subsection{Same angle}

What is it for? State obvious use quickly, and give detail on any
non-obvious uses.

Recognition process: What kind of data does it use? What is the
context-free recognition like? What context does it use?

What (if any) visual feedback is there?

What actions are taken if it is positively recognized and not
filtered?
