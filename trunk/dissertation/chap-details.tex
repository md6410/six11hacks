\chapter{Sketch It, Make It: Details}

This is the details chapter. It describes the recognition strategy
involving ink parsing, and how it recognizes various things like
glyphs (rotation invariant pictograms) or gestures (grammar-based like
erase).

This will be a dry chapter but it should be the most useful to anybody
else who wants to work on this. Should be enough that a smart grad
student can implement things based on just what is here.

\section{Corner Finder and Segmentation}
\label{sec:corner-finder}

Stage 3 operates on the 'unprocessed' list made from stage 2. So
anything that was recognized in stage 2 does not make it this far.

\begin{itemize}
\item Remove hooks
\item Get structured ink related to unprocessed ink.
\item Add all new structured ink to the model
\item Apply post-hoc recognizers to new structured ink
\item Disambiguate any conflicting results using (a) context in the
  model and (b) preset rules like right-angle wins over same-length
  (filterRecognizedItems())
\item Apply remaining recognized items. This removes the related
  structured ink from the model.
\item Auto-latch remaining segments with each other and existing model
  segments.
\item Search for cutouts (don't call them stencils anymore)
\item Wake up constraint solver
\item Request state snapshot
\end{itemize}

\section{Sketch Interaction Techniques}

\subsection{Erase}

Wrote this somewhere else: For example, the Erase gesture went through
a number of design iterations. Each version was only slightly better
than the last, and users were only able to successfully execute the
scribble gesture about half the time. When it failed, the scribble
would be interpreted as linework, which would then require users to
erase that. I then completely changed the erase gesture recognizer so
it would operate as the pen was down. When it found a scribble, a
colored 'X' appeared over the scribble to indicate that the user's
erasure will succeed. This lets users scribble until an X appears,
giving them confidence the system has understood. This reduced the
number of recognition errors substantially.

\subsection{Flow selection}

\subsection{Undo and Redo}

\subsection{Latching}

Four kinds: automatic, endpoint, continuation, and t-junction.


\subsection{Same length}

\subsection{Specific length}

\subsection{Right angle}

\subsection{Same angle}

\subsection{Camera Control: Pan and Zoom}

\section{Constraint Solver}
