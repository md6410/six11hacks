\chapter{Sketch It, Make It: Overview}

Chapter introduces rapid fab as a broad area of interest, laser
cutting as the narrow area. Describe the kind of users in this space:
ranging from students to hobbyists to professional designers. What is
the range of how the machines are used? What is made with them?


\section{Rapid Fabrication}

broad area

\section{Laser Cutting}

narrow area

Talk about the machines, what they cost, what they can cut, what kind
of output they provide, how long it takes to use, other operation
concerns like ventilation and inert gas.

Talk about the requirements imposed on the designer by the machine: 2D
cutfiles, material management, laser kerf issues. 

\section{Designer Workflow}

Then go into the typical process: design on paper, translate to CAD,
print it out, evaluate what to do next, then either go back to a prior
step or call it good enough. The bottleneck in this 'rapid' process is
often software due to how hard it is to learn and use.


\section{Sketch It Make It} 

People sketch like it is going out of style, until they are confronted
with telling a computer what's up. Then they have to switch into this
really strange way of 'designing' by using a keyboard and mouse,
because that's what is convenient for the computer.

I took all of this sketch recognition stuff that computery people have
been talking about for years and actually made something that
demonstrates (to a small but compelling degree) that we really can
design specific precise things by drawing them. It isn't exactly like
drawing on paper, but it is more like paper than it is like working
with nasty CAD.

This is an overview. make a story of somebody having a need, then step
through the process of designing, roughly at first, then with more
detail. No details on how the techniques work, but ok to say what they
are and what they look like. This is about behavior as a whole.

