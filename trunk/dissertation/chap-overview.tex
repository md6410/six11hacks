\chapter{Sketch It, Make It: Overview}

Chapter introduces rapid fab as a broad area of interest, laser
cutting as the narrow area. Describe the kind of users in this space:
ranging from students to hobbyists to professional designers. What is
the range of how the machines are used? What is made with them?

\section{Rapid Fabrication and Laser Cutting}

Talk about the machines, what they cost, what they can cut, what kind
of output they provide, how long it takes to use, other operation
concerns like ventilation and inert gas.

Talk about the requirements imposed on the designer by the machine: 2D
cutfiles, material management, laser kerf issues. 

Then go into the typical process: design on paper, translate to CAD,
print it out, evaluate what to do next, then either go back to a prior
step or call it good enough. The bottleneck in this 'rapid' process is
often software due to how hard it is to learn and use.

\section{Motivating Scenarios}

First a little one on making a triangular angle brace. This is a
simple example that illustrates the system's capabilities and how the
user interacts with it. It explains the UI layout, the modeless
interaction, sketch recognition, constraints, and cutfile generation.

\subsection{Pictureframe Holder Example}

The second example is cribbed from the video. It is of the user
designing a picture frame holder. It involves:

\begin{itemize}
\item Using a tablet, most likely a `blind' tablet with separate
  display. Introduces hand/eye coordination issues.
\item Having the idea in the first place
\item Drawing roughly
\item Mistakes: errors (the user or the system makes errors) vs. ``on
  second thought'' mistakes
\item Adding details
\item Imagining 3D construction. Admit the tool should support this
  directly
\item Fabrication: making the cutfile, laser cutting, and assembling
\item (maybe) another iteration
\end{itemize}

\subsection{Technical Challenges Met By SIMI}

The above section describes the user's needs, and how the user
interacts with the system. This section introduces broadly how the
system is built to support those needs and interaction. The technical
aspects are:

\begin{itemize}
\item Ink parsing (finding corners and segments, removing hooks)
\item Recognition (dynamic or post-hoc)
  \begin{itemize}
  \item Glyphs (akin to character recognition, like right angles)
  \item Gesture (recognizes grammatical patterns like erase or
    encircle)
  \end{itemize}
\item Disambiguation of recognized things
\item Data model and constraints
  \begin{itemize}
  \item User sketches things, system reates/maintains constraints
  \item Uses iterative, numerical relaxation method
  \item Only has X basic constraint types underlying everything
  \item High-level ``user constraints'' composed of low-level constraints
  \end{itemize}
\item Cutfile generation
  \begin{itemize}
  \item Basic typewriter algorithm. nothing fancy
  \item Scales things appropriately
  \item PDF output for Illustrator, SVG output for Ponoko (explain why
    format matters)
  \end{itemize}
\end{itemize}

From the user's perspective, SIMI gives a new kind of experience
because it:

\begin{itemize}
\item Is ``modeless''. Should describe the mode problem, why
  overcoming it is such an important thing, and why it is especially
  pertinent to sketch-based interaction. (because if we have modes, we
  are essentially on slippery-slope to MouseCAD)
\item Is incremental, rather than one-big-batch style recognition that
  is so popular
\item Uses ``more natural'' input. The pen is easier to manipulate
  than the mouse. Naturalness is a common term but is ultimately
  meaningless because all design tools are artificial. Could say it
  has less cognitive and ergonomic demands than modal, mouse-based
  tools
\item From a domain perspective it is not over featured, so it is
  clear how to proceed.
\end{itemize}
