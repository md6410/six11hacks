\begin{table}
\centering
\begin{tabular}{l | p{12cm}}
\textbf{Constraint Type} &\textbf{Remark} \\
\hline
\\

Same Length &

Constraints two or more segments to be the same length. On each step,
it calculates the mean segment length and expands or contracts
segments to conform to the mean.

\\

Specific Length &

Like the Same Length constraint, this can refer to two or more
segments. It uses a specific numerical value for the expected segment
length.

\\

Right Angle &

Constrain two line segments to form a 90 degree angle. As there are
two possible orientations (e.g., $\bot$ vs. $\top$), it chooses the
nearest solution when calculating error and change vectors. It rotates
each point about its line segment midpoint.

\\

Same Angle &

Like Right angle, this constraint rotates points about their line
segment midpoints. It operates on two or more angles, polling its
constituent angles to find a mean value to calculate error and change
vectors.

\\

Colinear Points &

Constrains three or more points to be on the same line. 

\\ \hline
\end{tabular}
\caption[Constraint Types]{Constraint Types in SIMI.}
\label{tab:constraints}
\end{table}
