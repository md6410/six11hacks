\chapter{Conclusion}

Rapid fabrication---3D printing, laser cutting, and many other
processes---can support designers to make in ways that were not
possible even ten years ago. But to realize this promise, machine
access is not enough. Users need effective and appropriate design
tools. 

Sketching is cited (often blithely) as a common and necessary part of
the design process in various domains. Beginning in 1963 with
Sutherland's Sketchpad system~\cite{sutherland-sketchpad}, researchers
have developed sketching systems in many such areas, like mechanical
engineering~\cite{lipson-correlation}, web site
design~\cite{lin-denim}, and furniture
production~\cite{oh-fab,saul-sketch-chair}. In this time there has
been a great deal of work on sketch recognition and on isolated sketch
based interaction techniques. But there has been little effort in
exploring how all of this fundamental work can be integrated and
presented as a coherent and useful whole to take on a real-world
design problem.

This thesis explored the role of sketch based interaction techniques
as a basis for rapid fabrication design tools. I bridged the gap
between an academic treatment of sketch based design and a useful and
useable system. This work has included a literature review in design
and computational support for sketching (Chapter~\ref{sec:rw} and the
survey with my
committee~\cite{johnson-sketch-review}). Chapter~\ref{sec:formative}
explored the domain of design for laser cutting by studying both
artist and the artifact. Last,
Chapters~\ref{sec:overview}~and~\ref{sec:details} detailed my
\textit{Sketch It, Make It} system that provides a solid example of
how a sketch based system can present a set of sketch recognizers and
(most importantly) interaction techniques to enable real people to do
real work in a way that was previously not actualized.

While I have made a prototype that many observers find compelling,
there is still more work that could be done. It raises interesting
ideas about what else is possible with this form of interaction,
and questions remain unresolved.

In this chapter I discuss how my work with SIMI can serve as a
starting point for further work. First I discuss the implications SIMI
has to research on sketch-based interaction and modeling. Next I will
describe the most obvious next steps that I would like to take.

\section{Implications to Related Areas}

%% \item Rapid (or even \textit{slow}) fab
%% \item Democratized design

Laser cutters, 3D printers, CNC mills and so forth are called
\textit{rapid} prototyping machines because they facilitate fast
machinework. Often, software design tools are the
bottleneck. Experienced professionals may have overcome the learning
curve and mastered the software to use it quickly, but the majority of
current (and certainly potential) users have not. Sketch based
interaction might be one part of a new kind democratized design
software that lets hobbyists and amateurs design and make with rapid
fabrication machines.

Inexpensive, widely available, and \textit{easy-to-use} rapid
fabrication hardware and software has interesting
consequences. Economically, it suggests a future where everyday people
are capable of simply ``printing'' a new object (such as a door knob)
when one is needed---which is obviously a topic of some concern for
the hardware store. Socially, it gives regular people the ability to
play an active role in the design, function, and production of
everyday things---which is concerning to professional designers and
engineers. While it is difficult to envision a world without hardware
stores (or designers and engineers), desktop design and fabrication
will have serious (and unpredictable) consequences.

Sketch based interaction may be paired with other input paradigms to
achieve unique results that would not be possible in isolation. While
sketching, the user's non-dominant hand is free to perform other
tasks. This can be leveraged to develop new ways to design. 

Touch input is a topic of importance these days, with touch-sensitive
devices like tablets and smart phones dominating the consumer
electronics market. Sketching, paired with touch, has recently been
shown to give compelling results~\cite{hinckley-pen-touch}. New
technologies, such as Microsoft's distance-sensing Kinect, offer
additional avenues that might be appropriate to mix with sketch-based
input. Virtual reality hardware---special glasses that simulate 3D
projection, volumetric displays, \textit{etc.}---could be used immerse
designers in augmented reality design tools that lets people draw
anything anywhere in physical space\cite{jung-lightpen}. These topics
have been covered to various degrees, but the sketch interaction
component has always been ad-hoc. I've given a solid example of one
way to support such interaction, so we might have something to gain by
revisiting some of these ideas. The input paradigms that pair
sketching with something else (touch, gesture, VR) have hardly been
explored enough to consider them known topics.

Currently, sketch interaction is limited to devices that explicitly
support pen interaction. But ongoing advances in sensing technology
(e.g. Touché~\cite{sato-touche}) may allow arbitrary surfaces to
accept sketch and touch input. Further, projection and display
technology continues to advance. When every available wall and
tabletop can display graphics and recognize sketch input, that has
serious consequences for the future of interaction and design.

\begin{itemize}

\item Sketch Interaction and SBIM. 
  \begin{itemize}
  \item Must move beyond isolated toy demonstrations
  \item Must move beyond ``works on my computer'' demos. This does not
    promote building on prior work due to technical difficulty in
    replicating other work
  \item Interaction is one of the glaring missing links
  \item SIMI gives and example of a useful and usable
    \textit{system}. Its value is its coherence.
  \item If we have a platform for research, a common starting point
    for building things, we explore what this mode of interaction can
    do. But we have to be brave and have vision. Have to move beyond
    slavish attendance to statistics and laboratory user studies.
  \end{itemize}
\end{itemize}

\section{Future Work}

% The most interesting research is that which leaves more questions
% than it answered. Now that I have made a useful and usable system,
% much of the feedback I receive is in the line of ``cool, but does it
% recognize X? Can I do Y now?'' Sketch based modeling is made
% possible by all the previous work on sketch recognition, but an
% exploration of proper integrative interaction design was necessary
% for it to move beyond the lab. This work allows SBIM to make that
% transition, and opens up a lot of interesting questions and new
% avenues for further work. 

\begin{itemize}
\item Commercialization is likely
\item Most immediate next steps would be:
  \begin{itemize}
  \item Support 'doodle' ink: not recognized but may be important
  \item Handwriting recog. for dimension values
  \item More constraint types (e.g. parallel, centered)
  \item Way to visualize how 2D parts compose to make 3D thing
  \end{itemize}
\item Machine learning
  \begin{itemize}
    \item Learning user preferences, drawing styles
    \item Learning ad-hoc syntax
    \item Learning organizational preferences (e.g. cultural norms in
      company/domain)
  \end{itemize}
\item Domain-specific improvements (for laser cutting)
  \begin{itemize}
  \item Kerf
  \item Have a library of common joint types
  \item Optimize use of material
  \item Versioning of parts
  \end{itemize}
\item Possibiliities beyond laser cutting:
  \begin{itemize}
  \item 3D modelling
  \item 2D graphic design (illustrations or cartooning)
  \item Other design domains: architecture, mechanical engineering,
    software development
  \item Classroom or academic uses: mathematical graphs, physics
    simulation, box-and-arrow diagrams
  \item Essentially any domain that makes use of structured or
    semi-structured diagrams
  \end{itemize}
\end{itemize}
