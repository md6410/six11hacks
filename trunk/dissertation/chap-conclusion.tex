\chapter{Conclusion}

Rapid fabrication---3D printing, laser cutting, and many other
processes---can support designers in ways that were not possible even
ten years ago. But to realize this promise, machine access is not
enough. Users need effective and appropriate design tools.

Sketching is often cited as a common and necessary part of the design
process in many domains. Beginning in the early 1960s with
Sutherland's Sketchpad system~\cite{sutherland-sketchpad}, researchers
have developed sketching systems in many areas, like mechanical
engineering~\cite{lipson-correlation}, web site
design~\cite{lin-denim}, and furniture
production~\cite{oh-fab,saul-sketch-chair}. During this time there has
been a great deal of work on sketch recognition and on isolated
sketch-based interaction techniques. But there has been little effort
in exploring how all this fundamental work can be integrated and
presented as a coherent and useful whole to address real-world design.

This thesis explored the role of sketch based interaction techniques
as a basis for rapid fabrication design tools. I bridged the gap
between an academic treatment of sketch-based design and a real-world
useful and usable system. This work has included a literature review
in design and computational support for sketching
(Chapter~\ref{sec:rw} and the survey with my
committee~\cite{johnson-sketch-review}). Chapter~\ref{sec:formative}
explored the domain of design for laser cutting by studying both
artist and the artifact. Last,
Chapters~\ref{sec:overview}~and~\ref{sec:details} detailed my
\textit{Sketch It, Make It} system that provides a solid example of
how a sketch based system can present a set of sketch recognizers and
(most importantly) interaction techniques to enable real people to do
real work in a way that was previously not actualized.

While I have made a prototype that many observers find compelling,
there is still more work to do. It raises interesting ideas about what
else is possible with this form of interaction, and questions remain
unresolved.

In this chapter I discuss how my work with SIMI can serve as a
starting point for further work. First I discuss the rationale for
providing the particular feature set embodied in this software. Next I
discuss the implications SIMI has on several areas of research. Last,
I describe the most obvious next steps that I would like to take.

\section{Justification of Feature Choices}

SIMI includes a set of interaction techniques that let users perform
actions to create, edit, and build precision models for laser
cutting. This particular collection chosen because it (a) provided a
minimal set of features to be useful, (b) it covered a range of
functions and (c) demonstrated a variety of sketch recognition types.

Some traditional software features were intentionally left out. For
example, \textit{Copy and Paste} is nearly ubiquitous in interactive
systems, but it is not found in SIMI. It is often beneficial to
discard long-standing assumptions of what is necessary to explore
alternate ways to proceed. By omitting Copy and Paste, I had to focus
on making it as easy as possible to create geometry and
constraints. It is possible to create a `copy' of geometry using
standard SIMI actions. This helps to keep the application simple.

\section{Implications to Related Areas}

This research has implications to several areas, including rapid
fabrication hardware and surrounding communities, interaction design,
and research on sketch-based interaction and modeling.

\subsection{Rapid Fabrication and Maker Communities}

Laser cutters, 3D printers, CNC mills and so forth are called
\textit{rapid} prototyping machines because they facilitate fast
machine-work. Often, software design tools are the bottleneck in a
designer's work/build process. While experienced professionals might
master the software to use rapid fabrication machines quickly, the
majority of current (and certainly potential) users have not. Sketch
based interaction might be one part of a new kind democratized design
software that lets hobbyists and amateurs design and make with rapid
fabrication machines.

Inexpensive, widely available, and \textit{easy-to-use} rapid
fabrication hardware and software has interesting
consequences. Economically, it suggests a future where people can
simply ``print'' new objects like door knobs as needed---obviously a
topic of concern for hardware stores and manufacturers. Socially, it
gives regular people the ability to play an active role in the design,
function, and production of everyday things---which is concerning to
professional designers and engineers. While it is difficult to
envision a world without hardware stores (or designers and engineers),
desktop design and fabrication will have serious, unpredictable, and
positive consequences.

\subsection{Interaction Design}

Sketch based interaction may be paired with other input paradigms to
achieve unique results that would not be possible in isolation. While
sketching, the user's non-dominant hand is free to perform other
tasks. This can be leveraged to develop new ways to design. 

Touch input is a current topic of importance, with touch-sensitive
devices like tablets and smart phones dominating the consumer
electronics market. Sketching, paired with touch, has recently been
shown to give compelling results~\cite{hinckley-pen-touch}. New
technologies, such as voice recognition or Microsoft's
distance-sensing Kinect, offer additional avenues that might be
appropriate to mix with sketch-based input. Virtual reality
hardware---special glasses that simulate 3D projection, volumetric
displays, \textit{etc.}---could be used to immerse designers in
augmented reality design tools that let people draw anywhere in
physical space~\cite{jung-lightpen}. These topics have been covered to
various degrees, but the sketch interaction component has always been
ad-hoc. This thesis gives a solid example of one way to support such
interaction. Input paradigms that pair sketching with something else
(touch, gesture, VR) have hardly been explored enough to consider them
known topics.

Currently, sketch interaction is limited to devices that explicitly
support pen interaction. But ongoing advances in sensing technology
(e.g. Touch\'e~\cite{sato-touche}) may allow arbitrary surfaces to
accept sketch and touch input. Further, projection and display
technology continues to advance. When every available wall and
tabletop can display graphics and recognize sketch input, that has
serious consequences for the future of interaction and design.

The recognition and resolution architecture from
Chapter~\ref{sec:overview} might be useful to researchers and
developers working on other kinds of recognition-based systems. For
example, multi-touch devices or Kinect-style hand or body recognition
could use the staged approach to simplify the recognition
process. However, because there are multiple contemporaneous sample
points to consider, future developers might have to extend the
architecture discussed here.

Perhaps most importantly, the staged recognition architecture lets
designers interact with computers using easy-to-make gestures that
richly specify \textit{an action}, \textit{to which objects}, and
\textit{in which way}. This seems to reduce users' cognitive load by
removing the need to enter persistent modes or the needs to set up
sequences of atomic operations to achieve a desired
outcome. Additional work on this topic is warranted.

\subsection{Sketch Based Interaction and Modeling}

This thesis claims that SIMI exemplifies sketch-based interaction
techniques that work well together. However, the particular set is
tuned to the laser cutter domain and to the features SIMI
provides. How can future researchers generalize this contribution?

The recognition and resolution architecture from
Section~\ref{sec:recognition-architecture} guides programmers and
interaction designers to develop new techniques and
applications. While a programming structure is helpful, there are
other important aspects to consider when designing new sketching
gestures.

I provide several heuristics for developing sketch-based interaction
techniques in other applications:

\textit{Leverage context}. Gesture syntax can be overloaded by letting
context determine meaning. For example, a circle drawn around two
unlatched end points is semantically different from a circle drawn
around a cutout. Overloading gestures means less work for the
programmer, and fewer gestures for the user to learn and
remember. However, care must be taken to ensure that the contexts are
separate enough that they will not be confused.

\textit{Substantial shape differences improve recognition and user
  experience}. Strive to develop gestures that are substantially
different from one another. It is tempting to create many gestures
that are different in subtle ways because it is easy to program the
related recognizers. However, users will have a difficult time
remembering and executing these subtly different gestures.

\textit{Rely on visual reinforcement and physical metaphors}. SIMI's
right angle constraint appearance echos the gesture used to create
it. When there is not a visual, the gesture should make physical
sense: the circle gesture used to latch points together feels like
tying the points together with a string; the erase gesture is based on
scratching over unwanted marks on physical paper.

\textit{Make use of the staged recognition architecture}. Some
gestures should be acted upon immediately, while others could (or
should) be deferred until later. For dynamic recognizers, effective
visual feedback is critical. While the staged architecture eases
programming, its true purpose is to improve user experience.

\section{Future Work}

The most interesting research leaves more questions than it
answered. Having built a useful and useful system for laser cut
designs, much of the feedback I receive focuses on additional
features. Observers commonly would like to see a tool like SIMI for
some other domain, such as for 3D modeling.

\subsection{Common Feature Requests}

It is likely that Sketch It, Make It (or something like it) will be
made into a commercial product soon. We have accumulated a long list
of feature requests that can serve as a starting point for the next
set of additional features.

Users like the paper-like feel of the tool, and would prefer to have
the ability to add ink that is not recognized. This enables designers
to explore freely, as they would on paper, while still having the
ability to use these unrecognized portions in future work.

Handwriting recognition is another commonly requested feature. This
was intentionally not included in the current prototype because it
would have required a substantial time investment without a
correspondingly substantial pay-out. Handmade numbers and text could
be applied to give dimensions and labels to parts of a drawing.

Users also request additional constraints, for example making two
lines parallel, or making an object appear at the midpoint between two
others. This would allow a wider and richer range of output.

Another feature that would likely make SIMI much more useful is the
ability to assemble 3D constructions from 2D parts. This would let
users see the relationship between the 2D parts they have designed and
the final 3D output. This can be useful, for example, to identify
stylistic or assembly problems before spending time and material on
the laser cutter.

\subsection{Machine Learning Improvements}

SIMI was built on the idea that it is generally better to give users
interaction techniques to state their intentions, rather than using
machine learning routines to deduce what the user wants. However,
there are a number of cases where machine learning could be used to
make the overall user experience better.

Each user has a drawing style that is different from any other. The
differences can be subtle, or might be substantial. In keeping with
the paper-like interface, an intelligent agent could watch the user
draw and learn their particular style and preferences. For example, if
a user consistently fails to make the erase gesture correctly, their
preferred erasure gesture could be learned and applied (without
asking). The user could also teach the system new gestures entirely,
providing new syntax directly.

Sometimes companies or other organizations adopt conventions. These
conventions may be common within that community but not typically
found outside. SIMI could learn these social norms, and understand
when to apply them.

\subsection{Domain Specific Improvements}

Laser cutters and similar machines (water jet, plasma, CNC mills)
share common properties. If the software could account for these
properties, the user could incorporate them into their designs. An
obvious property of laser cutters is the \textit{kerf}---the width of
the cut path. The laser cutter used throughout SIMI's development
typically leaves a kerf of approximately 0.4mm. When designing
press-fit notches for rigid material that is 3mm thick, the kerf size
is a significant concern. If the tool understood that certain geometry
was sensitive to kerf thickness, the software could save the user time
and material costs by making more accurate cuts.

The press-fit notch just mentioned is just one type of
joint. Currently SIMI does not include a concept of joints. If it did,
the user would be able to indicate where joints should exist, between
which parts, what the material properties are---and the software would
take care of the rest.

Using material is another important consideration, both because it
costs money, and because supplies are often limited. For example,
Ponoko sells 15''x15'' squares of acrylic for between \$8 and \$28
depending on thickness. To make more efficient use of material, the
software could perform intelligent cut file layout with a
shape-packing algorithm. If the system knows where a piece of material
has holes, the algorithm could also place new parts on used material,
saving a lot of money.

Assemblies often have a lot of parts, and it can frustrate users when
trying to determine which part goes where. The tool could etch part
numbers (or joint labels) into the material to aid assembly.

\subsection{Beyond Laser Cutting}

In this work, I set out to demonstrate how a set of interaction
techniques designed to be used together to make a useful and usable
system, and show that sketch-based interaction should be taken
seriously, particularly as a paradigm for doing design. To do this I
picked a specific domain---laser cutting---but there are other domains
that would have been appropriate as well. 

3D modeling is an obvious choice. Physical objects are necessarily
three dimensional, and designers often sketch them in
perspective. There are interesting challenges associated with ``3D
sketching'', and many others are working on these research topics.

2D applications will be easier to write based on SIMI's current
implementation. Graphic design applications for illustrations,
diagrams, logos, T-shirt designs, or cartooning are appropriate areas
for SIMI's interaction.

Teachers famously draw on chalk boards or whiteboards to illustrate
concepts in many areas, from physics to mathematics. A sketch-based
education system might be a useful tool for teaching STEM topics at
all levels---primary, secondary, and post-secondary education.

The techniques presented in this dissertation could potentially be used
in any domain that makes use of structured or semi-structured
diagrams. It could be used to think about ideas, generate concepts. Or
it could be a communication tool used to work with others in ongoing
design projects. Or sketch-based design could simply be used as a
fast, efficient method of specifying the designer's intent to create
finished output.

\section{Looking Forward}

Nearly everything in our environment was deliberately designed and
built by a human being. An artifact's users are not often the same
people that designed or built it. The recent introduction and success
of affordable, fast fabrication machinery suggests the gulf between
\textit{designers} and \textit{users} may narrow as desktop
manufacturing lets ordinary people design and build things on their
own. One remaining bottleneck is software: access to fabrication
machinery is not enough. Democratized fabrication requires
democratized design tools: People need design software that is easy to
use and gives the means to express ideas. But for ordinary people,
current CAD software is expensive, hard to learn, and hard to use.

The sketch-based interaction presented in this thesis explicitly
supports design for laser cut items. This is provided as an example of
a (potentially) much larger space of sketch-based modeling tools. Many
other application areas were mentioned above, such as STEM
applications for students, 3D modeling, animation, and business
diagrams. This line of research is important not simply because it
enables users to do one thing marginally better than before. It is
important because it will empower people to design and make many kinds
of things that were previously impossible. This is a powerful idea,
and the consequences are far-reaching.

