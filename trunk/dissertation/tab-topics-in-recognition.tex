\begin{table}
\begin{tabular}{p{5cm}| p{2cm} | p{8.5cm}}
\textbf{Topic} & \textbf{Section} & \textbf{Remark} \\
\hline \hline
Recognition accuracy &
\ref{sec:recognition-accuracy} & Discusses how to measure recognition 
accuracy, and what acceptable error rates are.
\\ \hline
When to recognize & 
\ref{sec:recognition-when} &
Recognition is powerful but may also distract users from their
task. 
\\ \hline
What to recognize &
\ref{sec:recognition-what} & Sketches may represent objects
(e.g. tables and chairs) and spatial or functional relationships
between those objects (e.g. chairs positioned around table perimeter).
\\ \hline
How to segment &
\ref{sec:recognition-segmentation} &
Sketches contain many different symbols that may overlap. Recognizers
must isolate groups of marks for consideration.
\\ \hline
How to recognize &

\ref{sec:recognition-techniques}, \ref{sec:recognition-hard-coded},
\ref{sec:recognition-library} &

Many recognition techniques exist, and rely on segmentation
methods. \\ \hline

Mediating recognition error &
\ref{sec:recognition-managing-ambiguity} &
Recognition often results in ambiguous results.
\\ \hline

\end{tabular}
\caption[Sketch Recognition Topics]{Topics in sketch recognition}
\label{tab:recognition-topics}
\end{table}
