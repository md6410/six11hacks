\chapter{Introduction}

Nearly everything around us has been \textit{designed} and
\textit{built}. With mass production, a single design can be
replicated many times. For example, a company might design a coat rack
and build 200,000 copies. This model of coat rack is likely to be good
enough for its users, and while they might prefer this or that to be
different, it lets them hang up their coats and get on with their
lives---all for less than \$30. But what if the user \textit{could}
indicate exactly what they want for the same price?

Machines like 3D printers, CNC mills, and laser cutters are beginning
to give us a glimpse of what the future might hold. Today, a
knowledgable person can design and ``print'' parts for a coat rack on
rapid prototyping machines. And if the designer finds they want to
change something, they can simply print a revised model. While this is
not as cost-effective as buying an ``almost good enough''
mass-produced coat rack at a store, it does enable the consumer to
directly engage themselves with the design and production of the item.

Rapid prototyping is a new technological phenomenon with economic and
social consequences. The machinery that enables regular people to
``print'' objects at home was unavailable ten years ago: either it was
too expensive, or it had not been invented yet. The current crop of
rapid prototyping machinery is still too expensive for most people to
buy for personal use. And today's machines produce adequate (but not
typically compelling) output. Happily, this is quickly changing as
price decreases while quality improves. It is interesting to see a new
technology that enables new activities---some that we can predict, but
others that we can't yet clearly see. It is conceivable that we are
witnessing the beginning of a shift from an economy entirely based on
\textit{mass production} to one that includes \textit{mass
  customization}.

Rapid fabrication gives many people the opportunity to design and
build things when there was no opportunity before. In the mass
production model, people are merely consumers. Rapid fabrication
enables people to play an active role in designing and constructing
the world around them. But the machines, alone, are insufficient: a
human must tell them what to make. To support this, people need
adequate design tools.

This dissertation addresses the observation that people need adequate
design tools if they are to effectively use rapid fabrication
machines. Current modeling software targets users who go to school to
learn how to design and use design software. However, most people can
not dedicate that much time to learn the intricacies and gotchas for
design software.

% TODO: get a higher rez version of this picture
\begin{figure}[t]
  \centering
  \includegraphics[width=0.9\linewidth]{img/simi-with-cintiq.png}
  \caption[SIMI on a Wacom Cintiq]{Using Sketch It, Make It on a Wacom
    Cintiq display tablet. The user sketches with their preferred
    hand, and uses a single button with their non-dominant hand.}
  \label{fig:simi-intro}
\end{figure}

While someone might a hard time using professional design software, it
is likely that person can \textit{draw} using pencil and paper to
communicate their ideas with other people. Even if the freehand sketch
is rough, it is an effective method to express ideas about the shape
of objects and how they relate to one another.

Researchers have tried to leverage sketching as a computational medium
for at least half a century, and despite many interesting prototypes,
we have yet to see this effort move beyond academic laboratories. I
believe the critical element missing from this body of work is the
lack of a \textit{system of interaction design} for sketch-based
tools.

In this dissertation, I fill this missing space by describing a
\textit{useful} and \textit{usable} modeling tool that exemplifies a
\textit{coherent set of sketch-based interaction techniques}. The
software, \textit{Sketch It, Make It (SIMI)}, lets people design
precise items for laser cutters. 

\section{Intended Audience}

This work is aimed primarily at researchers interested in sketch-based
modeling. This includes people working on topics traditionally found
in computer science (e.g. artificial intelligence or computer
graphics) as well as those interested in human-computer interaction
and interaction design. Beyond motivating the work (sketch-based
interaction) and domain (designing for laser cutters), a reader could
implement any of the techniques described herein. The target audience
may be in academia, but it is hoped that commercial software tools can
start to incorporate some of the novel interaction exemplified by my
tool.

Further, members of the physical hacking or tangible interaction
worlds may be interested in learning how their trade can benefit from
improved design tools.

\section{Motivation}

This work is motivated by academic and practical perspectives. From an
academic perspective, I am motivated to find a way to bring greater
coherence to the field of sketch based modeling by providing a
compelling example of a useful and usable sketching system that
exhibits a set of mutually harmonious interaction techniques. We have
dozens of compelling sketch-based systems that demonstrate recognition
algorithms, corner-finding methods, and interaction techniques, among
other topics. But as I have said, sketch-based tools don't exist
outside research labs. Because researchers do not share, or have
complete documentation or working examples to use as a starting point,
they must build their work from scratch. Consequently, researchers
interested in one topic (e.g. making new interactive widgets for
sketching) must spend a substantial amount of time implementing
functions that are not their research focus (e.g. ink parsing). While
it might on occasion be useful to re-invent the wheel, I view the
focus on side-topics as a largely needless distraction.

The practical motivation behind this work is based on the observation
(shared by many people) that current computer-based design tools are
hard to use, and that ``natural'' techniques like drawing, speaking,
and gesturing are potentially great methods to design with
computers. It would be a wonderful turn of events if such a natural,
easy-to-use tool were available to designers. I believe one of the
critical steps to making this a reality is to develop an appropriate
system of interaction. For example, an appropriate system to interact
with (drive) a car involves foot pedals, a steering wheel, and
dashboard indicators and knobs. It would be inappropriate (and
dangerous) to force automobile drivers to control their cars with a
standard PC setup: keyboard and mouse, complete with software updates
and dialog boxes. But this is the approach that many researchers take
when developing sketch based design tools. It is convenient, but
inappropriate. This dissertation offers an example of an appropriate
interaction for sketch based design tools.

\section{Thesis Structure}

This chapter introduced the domain of rapid fabrication and the
culture of hacking and making that is associated with it.  I discussed
the field of sketch based modeling and introduced my tool,
\textit{Sketch It, Make It (SIMI)}, a digital modeling tool for
designing precise items for laser cutters. The following chapters are
outlined as follows.

Chapter 2 covers background information about rapid fabrication and
laser cutting, and provides a brief survey of the literature on
computational support for sketch-based modeling in design. Next, in
Chapter 3 I describe several formative investigations that informed
the development of my tool: observations and interviews with
designers, and an analysis of some of the artifacts made with current
design tools and laser cutters.

I devote two chapters discussing SIMI. Chapter 4 introduces the tool
and presents usage scenarios: why and how somebody uses it. This is an
overview of the system. Chapter 5 details individual interaction
techniques and other technical aspects to the system likely to be
useful to others.

Chapter 6 presents several evaluations of the system, including
results of a quantitative evaluation involving 60 undergraduate
architecture students. Last, in Chapter 7, I discuss the implications
SIMI has on digital modeling tools and sketch based design. I also
describe some additional questions that warrant future work.
