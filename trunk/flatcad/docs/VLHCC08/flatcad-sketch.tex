% Keep these two lines in order to get typesetting to work w/ the funny fonts.
%!TEX TS-program = xelatex
%!TEX encoding = UTF-8 Unicode

\documentclass[times, 10pt, twocolumn]{article}
\usepackage{geometry} % See geometry.pdf for lots of layout options. 
\usepackage{graphicx}
\usepackage{amssymb}
\usepackage{hyphenat}
\usepackage{subfigure}

\usepackage{fontspec,xltxtra,xunicode}
\defaultfontfeatures{Mapping=tex-text}
\setromanfont[Mapping=tex-text]{Cochin}
%\setromanfont[Mapping=tex-text]{Times}
\setsansfont[Scale=MatchLowercase,Mapping=tex-text]{Gill Sans}
\setmonofont[Scale=MatchLowercase]{Andale Mono}

\newbox\subfigbox % Create a box to hold the subfigure. 
\makeatletter 
\newenvironment{subfloat}% % Create the new environment. 
{\def\caption##1{\gdef\subcapsave{\relax##1}}% 
\let\subcapsave=\@empty % Save the subcaption text. 
\let\sf@oldlabel=\label 
\def\label##1{\xdef\sublabsave{\noexpand\label{##1}}}% 
\let\sublabsave\relax % Save the label key. 
\setbox\subfigbox\hbox 
\bgroup}% % Open the box... 
{\egroup % ... close the box and call \subfigure. 
\let\label=\sf@oldlabel 
\subfigure[\subcapsave]{\box\subfigbox}}% 
\makeatother 

\title{Sketching for the Refinement Stage of Design}

\author{Gabe Johnson\\\texttt{johnsogg@cmu.edu}} 

\begin{document}

\maketitle

\begin{abstract}
This workshop paper summarizes the motivation for my forthcoming PhD
proposal on \textit{computational support for sketching during the
  refinement stages of design}. I will explore this space by (1)
building a tool empowering people to easily design artifacts by
concurrently sketching alongside traditional interaction methods, and
(2) evaluating interaction and engineering issues associated with
it. This work is preliminary, and I welcome constructive criticism in
shaping my proposal.
\end{abstract}

\section{Introduction}

Early phases of design can be characterized by idea generation and
exploration \cite{newman-web-designers,buxton-sketching}. Sketching
readily supports these activities. This is why many sketching systems
have focused on early phases of design. Refinement phases are
characterized by incremental revision and production---activities
traditional computer design tools support well. However, designers
continue to sketch on paper after they have begun revising computer
models. These refinement-phase sketches help people solve sub-problems
that were not apparent or relevant during earlier exploration.

However, current software design tools are unable to directly leverage
these refinement stage sketches. Instead, users manually translate
sketches to the computer model. Design tools could be made to
understand the user's informal sketching input by leveraging the
formal representation already present in the model. It is not
clear---from both technical and HCI perspectives---how a tool that
concurrently supports formal and informal input should be made. In
order to explore these areas, I propose to build such a tool.

First, I will produce a design environment called FlatCAD. The current
version of FlatCAD supports users in making models by programming in a
domain-specific language called FlatLang \cite{johnson-flatcad}. The
next version of FlatCAD will also support traditional GUI operations
and sketching input. 

The second contribution explores challenges associated with
concurrently using informal sketches with formal computer
representations. In particular I am interested in (a) software
engineering and (b) HCI aspects of these tools. In the following
sections I briefly describe FlatCAD, discuss the technical and
interaction challenges associated with adding support for sketching to
it.

\section{FlatCAD Use Scenario}

% Just summarize FlatCAD, don't go into the philosophy, nor should
% FlatLang be described in great depth... a 'LOGO turtle with a laser
% cutter for an ass' is sufficient. This section gives just enough
% background for a later discussion of the environment I will add
% sketching support for.

FlatCAD is an environment for designing objects made of flat material
for rapid prototyping. Currently, users define shapes by programming
in the LOGO-like FlatLang domain-specific language. The output of a
FlatLang program is a ``cutfile'' that is sent to a laser cutter. A
simple FlatLang program for drawing a square is listed in
Figure~\ref{fig:flatlang}.

\begin{figure}
  \centering 
  \begin{subfloat}% 
    \begin{minipage}{1.8in} 
      \small
\begin{verbatim}
define rectangle(width, height)
  repeat(2)
    forward(width)
    left(90)
    forward(height)
    left(90)
  done
done
\end{verbatim}
    \end{minipage}
  \end{subfloat} 

  \caption{Very simple \nohyphens{FlatLang} code that draws a
    rectangle.}
  \label{fig:flatlang}
\end{figure} 

FlatCAD has been used to design several classes of physical artifacts,
such as construction kits to mechanisms to household goods like soap
dishes and toothbrush holders. Here I detail the process I took when
making the soap dishes shown in Figure~\ref{fig:soapdish-sketch}.

I began by making quick drawings to help me think about aspects of
making a soap dish---its primary purpose is to hold a bar of soap, but
it also should prevent water from pooling, be easy to use, and be
stylish. 

\begin{figure}
  \centering
  \subfigure[] { 
    \label{fig:soapdish-gabe} 
    \includegraphics[width=5.0cm]{soapdish-gabe.jpg} 
  }
  \hspace{0.5cm}
  \subfigure[] {
    \label{fig:soapdish-tiller} 
    \includegraphics[width=5.0cm]{soapdish-tiller.jpg}
  }
  \caption{Two soap dishes made with FlatCAD.}
  \label{fig:soapdish}
\end{figure}

\begin{figure}
  \centering
  \subfigure[] { 
    \label{fig:soapdish-sketch-1} 
    \includegraphics[]{soapdish-01.jpg} 
  }
  \subfigure[] {
    \label{fig:soapdish-sketch-2} 
    \includegraphics[]{soapdish-02.jpg}
  }
  \subfigure[] {
    \label{fig:soapdish-sketch-3} 
    \includegraphics[]{soapdish-03.jpg}
  }
  \subfigure[] {
    \label{fig:soapdish-sketch-4} 
    \includegraphics[]{soapdish-04.jpg}
  }
  \subfigure[] {
    \label{fig:soapdish-sketch-5} 
    \includegraphics[]{soapdish-05.jpg}
  }
  \subfigure[] {
    \label{fig:soapdish-sketch-6} 
    \includegraphics[]{soapdish-06.jpg}
  }
  \subfigure[] {
    \label{fig:soapdish-sketch-7} 
    \includegraphics[]{soapdish-07.jpg}
  }
  \subfigure[] {
    \label{fig:soapdish-sketch-08ab} 
    \includegraphics[]{soapdish-08ab.jpg}
  }
  \subfigure[] {
    \label{fig:soapdish-sketch-08c} 
    \includegraphics[]{soapdish-08c.jpg}
  }
  \subfigure[] {
    \label{fig:soapdish-sketch-9} 
    \includegraphics[]{soapdish-09.jpg}
  }
  \subfigure[] {
    \label{fig:soapdish-sketch-10} 
    \includegraphics[]{soapdish-10.jpg}
  }
  \caption{Sketches made during idea generation and exploration.}
  \label{fig:soapdish-sketch}
\end{figure}

The first sketches in this sequence helped me to think about
functional requirements, while the last few sketches focused on
details of how the soap dish would be constructed and what particular
pieces would look like. The final two sketches include parameters and
the locations of notches where other pieces adjoin. After these
sketches were made, I recreated the sketches in FlatCAD by writing a
FlatLang program. Figure~\ref{fig:soapdish-slat} shows the program's
rendering of a single vertical slat.

\begin{figure}
   \centering
   \includegraphics[width=3.5cm]{soapdish-style-1-slat.png} 
   \caption{FlatCAD screen shot of a soap dish slat in progress.}
   \label{fig:soapdish-slat}
\end{figure}

As I worked I needed to make several more drawings
(Figure~\ref{fig:refine}). Some were made to help understand how
variables were related, others were used to add needed complexity to
the parts so they would fit together well. Even though I had a
perfectly good image of the model on my computer screen, I had to
re-draw it on paper before making progress. The details already
specified (lengths and angles) could not be transferred to the
sketch. After making the drawings, I had to translate what I had
learned from them back into FlatLang code.

It would be helpful if FlatCAD enabled users to sketch directly on the
model in order to modify it. The precise nature of the on-screen model
provides excellent contextual clues for supporting recognition. For
example, several sketches refer to variables that exist in the formal
model (e.g. \texttt{n.thickness} in Figure~\ref{fig:refine-01}) or
anonymous dimensions (e.g. the vertical edge indicated with the
arrowheaded line in Figure~\ref{fig:refine-03}).

% Note: \usepackage{subfigure}

\begin{figure}
  \centering
  \subfigure[] { 
    \label{fig:refine-01} 
    \includegraphics[width=4cm]{refine-01.jpg} 
  }
  \subfigure[] {
    \label{fig:refine-02} 
    \includegraphics[width=4cm]{refine-02.jpg}
  }
  \subfigure[] {
    \label{fig:refine-03} 
    \includegraphics[width=4cm]{refine-03.jpg}
  }
  \caption{Refinement phase sketches drawn on paper.}
  \label{fig:refine}
\end{figure}



\section{Interaction Techniques}

Currently, design tools support input done with a keyboard and mouse
in a structured user interface consisting of elements like buttons,
pick-lists, dialog boxes, text areas, and so on. Structured interfaces
are designed to minimize or eliminate ambiguity of user input.

Consider the simple task of enlarging a rectangle. A traditional
structured drawing program is likely to provide control handles, which
the user clicks and drags in order to change an object's size
(Figure~\ref{fig:object-handle-1}).

An alternate strategy is that of calligraphic interaction, wherein
users draw freehand, ambiguous input
\cite{jorge-calligraphic-introduction}. Users may sketch their
intention to resize a box in many ways: drawing arrows indicating what
should move, or re-drawing the entire
box. Figure~\ref{fig:object-sketch} illustrates a user changing a
rectangle by redrawing only the portion that must change. There are
currently no widely used conventions for such interaction, and little
is known about what is effective for ongoing usage.

\begin{figure}
  \centering
  \subfigure[Original rectangle] { 
    \label{fig:object-handle-1} 
    \includegraphics[origin=c,width=4cm]{object-handle-1.pdf} 
  }
  \hspace{0.5cm}
  \subfigure[Selecting object reveals `handles'] { 
    \label{fig:object-handle-2} 
    \includegraphics[origin=c,width=4cm]{object-handle-2.pdf} 
  }
  \hspace{0.5cm}
  \subfigure[Drag cursor, enlarge to 1/72 inch accuracy] { 
    \label{fig:object-handle-3} 
    \includegraphics[origin=c,width=4cm]{object-handle-3.pdf} 
  }
  \caption{Enlarging an object using traditional WIMP interaction.}
  \label{fig:object-handle}
\end{figure}


\begin{figure}
  \centering
  \subfigure[Original rectangle] { 
    \label{fig:object-sketch-1} 
    \includegraphics[origin=c,width=4cm]{object-sketch-1.pdf} 
  }
  \hspace{0.5cm}
  \subfigure[Draw new extents of rectangle, implicitly selecting the
    particular edges that will change] {
    \label{fig:object-sketch-2} 
    \includegraphics[origin=c,width=4cm]{object-sketch-2.pdf} 
  }
  \hspace{0.5cm}
  \subfigure[The exact size is not precisely provided] { 
    \label{fig:object-sketch-3} 
    \includegraphics[origin=c,width=4cm]{object-sketch-3.pdf} 
  }
  \caption{Enlarging an object using calligraphic interaction.}
  \label{fig:object-sketch}
\end{figure}

One of the primary challenges with interactive systems is the ``mode
problem'' \cite{tesler-mode-problem}: users often are unsure how to
enter a mode, or are unaware which mode they are in. Structured user
interfaces can arguably mitigate this problem more effectively than
recognition based interfaces.  There are several strategies for
handling this in sketch recognition user interfaces, including the
Inferred Mode Protocol \cite{saund-inferred-mode} and explicit mode
selection triggered by gestures
\cite{hinckley-scriboli,johnson-flow-selection}.

\section{Engineering}

Existing sketching tools produce models for refinement with other
design software. In these cases, the connection between the sketching
and refining tool has been one way \cite{lin-denim}. However, tools
that enable iterative sketching and refinement requires two-way
connections. For example, SimuSketch recognizes drawings of dynamic
systems \cite{kara-recognizer-uist}. The recognized model is passed to
the SimuLink program for display and editing. However, modifications
made in SimuLink are not reflected in the sketch.

FlatCAD will support such a two-way connection between informal and
formal modes of expressing intent. Informal sketch input may be
provided at one moment, formal commands given the next. The inherent
ambiguity of sketch input must be managed because the user may issue
formal instructions at any time. Say the user has drawn a shape that
may be a rectangle, but may also be an oval. Next the user selects a
'resize' tool and begins to enlarge the shape. What should happen?

This is partly an interaction question, but it is also an engineering
question. The drawing may provide enough context to disambiguate the
shape's identity. This context can come in many forms, such as domain
awareness \cite{alvarado-dynamic-bayes}, or knowledge of temporal
patterns of how people draw \cite{sezgin-multiscale-models}.

\section{Summary}

A proposed version of FlatCAD aims to support concurrent informal and
formal modes of input during the refinement phases of design. This
will serve as a useful means for exploring various human-computer
interaction topics related to sketch recognition interfaces. It also
provides a foundation to study engineering topics of how to identify,
model, and use context for the purpose of recognition. This work is
still very much in progress.

\bibliographystyle{plain}

\bibliography{sketch-bibliography}

\end{document}  

